\chapter{Key Ideas of the CP Compiler} \label{ch:key}

In this part, we propose an efficient compilation scheme that translates merges
in CP to extensible records, where types are used as record labels to perform
lookup on merges. We first introduce the key ideas under the hood of the CP
compiler and describe why and how to compile CP in this way. We also discuss the
major challenges that we had to overcome. Although our implementation targets
JavaScript, the design can be adapted to any other language that supports some
kind of extensible records. We refer the reader to \autoref{ch:calculi} for a
formal description of our compilation scheme and \autoref{ch:compilation} for
the details of our implementation targeting JavaScript. \autoref{ch:empirical}
conducts an empirical evaluation of the CP compiler.

\section{Dunfield's Elaboration Semantics} \label{sec:dunfield}

In previous work by \citet{dunfield2014elaborating} and its follow-up work by
\citet{oliveira2016disjoint}, the semantics of the merge operator is well
studied. According to the \emph{non-deterministic} operational semantics given
by \citeauthor{dunfield2014elaborating}, a merge $48  \bbcomma   \mathbf{true} $ may reduce to
$48$ or $ \mathbf{true} $; both are valid reductions. However, such reductions may
not preserve types. For instance, in a context like $\ottsym{(}  48  \bbcomma   \mathbf{true}   \ottsym{)}  \ottsym{-}  2$, the
merge should reduce to an integer. Alternatively,
\citeauthor{dunfield2014elaborating} proposes an elaboration semantics into a
target calculus with pairs, which is also used by
\citeauthor{oliveira2016disjoint} Within this framework, an intersection type
$\ottnt{A}  \, {\&} \,  \ottnt{B}$ is elaborated into a product type $ \ottnt{A}  \times  \ottnt{B} $, and a merge $\ottnt{e_{{\mathrm{1}}}}  \bbcomma  \ottnt{e_{{\mathrm{2}}}}$
is elaborated into a pair $ \langle  \ottnt{e_{{\mathrm{1}}}} ,  \ottnt{e_{{\mathrm{2}}}}  \rangle $. While an elaboration to pairs offers a
simple model for merges, it also imposes significant runtime overhead. We
identify three limitations in previous work. 

\paragraph{Indirect coercions.} \label{sec:coercive}
Following the elaboration model to pairs, $\ottsym{(}  48  \bbcomma   \mathbf{true}   \ottsym{)}  \ottsym{-}  2$ should be
elaborated into $  \langle  48 ,   \mathbf{true}   \rangle   . \mathsf{fst}   \ottsym{-}  2$. That is, we need to select the first
element from the elaborated pair to obtain a well-typed expression. Merges, due
to their flexible nature, do not have an explicit elimination form. Then how can
we determine where to insert ``\textsf{.fst}''? In a type-directed elaboration,
we can generate coercion functions according to subtyping judgments in the
typing derivation. A \rref{DTyp-Sub} can be found in previous work.
\begin{mathpar}
\ottdruleDTypXXSub{}
\and
\ottdruleElaXXSub{}
\end{mathpar}
The \rref{DTyp-Sub} means that if a source term $\ottnt{e}$ is inferred to have type
$\ottnt{A}$ and elaborated into a target term $\epsilon$, and the subtyping judgment
$ \ottnt{A}   \ottsym{<:}   \ottnt{B} $ implies the coercion function $c$, then $\ottnt{e}$ can be checked
against type $\ottnt{B}$ and elaborated into $ c\, \epsilon $. In the aforementioned
example, the merge of type $ \mathbf{Int}   \, {\&} \,   \mathbf{Bool} $ is cast to type $ \mathbf{Int} $. Therefore,
a coercion function should be implicitly inserted for the subtyping relation
$  \mathbf{Int}   \, {\&} \,   \mathbf{Bool}    \ottsym{<:}    \mathbf{Int}  $. Since the latter is the first half of the former, the
coercion $  \lambda \ottmv{x} .\, \ottmv{x}   . \mathsf{fst} $ is inserted.

A careful reader may notice that \rref{DTyp-Sub} does not produce
$  \langle  48 ,   \mathbf{true}   \rangle   . \mathsf{fst}   \ottsym{-}  2$ as we expect. Instead, it produces
$\ottsym{(}  \ottsym{(}    \lambda \ottmv{x} .\, \ottmv{x}   . \mathsf{fst}   \ottsym{)} \,  \langle  48 ,   \mathbf{true}   \rangle   \ottsym{)}  \ottsym{-}  2$, which is less efficient as it introduces a
spurious application. To fill the gap, we propose an alternative \rref{Ela-Sub}
in our work and a novel coercive subtyping judgment, which directly coerces
$\epsilon_{{\mathrm{1}}}$ into $\epsilon_{{\mathrm{2}}}$. In the aforementioned example, the subtyping relation
$  \mathbf{Int}   \, {\&} \,   \mathbf{Bool}    \ottsym{<:}    \mathbf{Int}  $ will coerce $ \langle  48 ,   \mathbf{true}   \rangle $ to produce the more efficient
$  \langle  48 ,   \mathbf{true}   \rangle   . \mathsf{fst} $. Although we only avoid one step of beta reduction in this
case, a more complicated subtyping judgment will lead to many coercion functions
composed together and introduce many spurious applications.

An alternative choice is to keep the original elaboration rules and let the
target language perform standard optimizations like function inlining. However,
this approach depends on the nature of the target language and brings
uncertainty to the performance of the generated code. In our implementation, we
choose to optimize it in the elaboration phase.

\paragraph{Linear merge lookup.}
A second important drawback of \citeauthor{dunfield2014elaborating}'s approach
is the representation of merges as nested pairs. The merge operator composes
expressions in a binary manner, so extracting one component from nested merges
of $n$ components requires $n-1$ projections in the worst case. For example,
when adding one more element to the previous merge, $48  \bbcomma   \mathbf{true}   \bbcomma   \text{`\texttt{a}'} $ for
example, one more projection must be added to the elaborated result as well:
$   \langle   \langle  48 ,   \mathbf{true}   \rangle  ,   \text{`\texttt{a}'}   \rangle   . \mathsf{fst}   . \mathsf{fst}   \ottsym{-}  2$. Note that we have simplified the coercion
application from $(\ottsym{(}    \lambda \ottmv{x} .\, \ottmv{x}   . \mathsf{fst}   \ottsym{)}\circ\ottsym{(}    \lambda \ottmv{x} .\, \ottmv{x}   . \mathsf{fst}   \ottsym{)})\, \langle   \langle  48 ,   \mathbf{true}   \rangle  ,   \text{`\texttt{a}'}   \rangle $ to
$   \langle   \langle  48 ,   \mathbf{true}   \rangle  ,   \text{`\texttt{a}'}   \rangle   . \mathsf{fst}   . \mathsf{fst} $. Compared with array access or dictionary lookup,
such projections are more expensive in terms of both code length and runtime
performance.

\paragraph{Pairs are order-sensitive.}
What is worse, a representation based on pairs has another disadvantage:
unnecessary coercions are never optimized. Consider two merges $48  \bbcomma   \mathbf{true} $
and $ \mathbf{true}   \bbcomma  48$. These two merges are equivalent in any context. Although
they lead to a different order in the elaborated pairs, permutation of
components does not matter as long as it is consistent with the projection. For
example, $  \langle  48 ,   \mathbf{true}   \rangle   . \mathsf{fst}   \ottsym{-}  2$ is the same as $  \langle   \mathbf{true}  ,  48  \rangle   . \mathsf{snd}   \ottsym{-}  2$. However,
permutation can lead to expensive coercions. To cast $48  \bbcomma   \mathbf{true}   \bbcomma   \text{`\texttt{a}'} $ to type
$ \mathbf{Char}   \, {\&} \,   \mathbf{Int}   \, {\&} \,   \mathbf{Bool} $, every single component needs to be extracted and
rearranged:
\[
\textbf{let } e =  \langle   \langle  48 ,   \mathbf{true}   \rangle  ,   \text{`\texttt{a}'}   \rangle  \textbf{ in }
 \langle   \langle   \ottnt{e}  . \mathsf{snd}  ,    \ottnt{e}  . \mathsf{fst}   . \mathsf{fst}   \rangle  ,    \ottnt{e}  . \mathsf{fst}   . \mathsf{snd}   \rangle 
\]
Thus, it is desirable to replace nested pairs with other representations that
support more efficient merge lookup and avoid conversions between equivalent
types.

\section{Our Representation of Merges}

\paragraph{Prologue: compiling overloaded functions.}
In programming languages that support function overloading, C++ for example, the
compiler generates different names for overloaded functions. This process is
usually called \emph{name mangling}. If we have a function \lstinline{f} with
two overloaded versions:
\begin{lstlisting}[language=C++]
void f(int x)  { ... }  // f -> __Z1fi
void f(bool x) { ... }  // f -> __Z1fb
\end{lstlisting}
Two different names are generated based on the parameter types: the postfix
\lstinline{i} in \lstinline{__Z1fi} is short for \lstinline[language=C++]{int}
and \lstinline{b} in \lstinline{__Z1fb} for \lstinline[language=C++]{bool}.
After name mangling, the overloaded versions are disambiguated, and the linker
can easily associate each call site with a specific version.

\paragraph{Key idea: compiling merges to type-indexed records.}
When it comes to merging, the situation is similar: a merge contains
``overloaded'' terms of different types. For example, the merge $48  \bbcomma   \mathbf{true} $
contains both an integer and a boolean value. When compiling the merge, we adopt
a similar technique to name mangling. We generate a unique name for every type,
which is used to look up the corresponding component. More specifically, a merge
is compiled to a record, and the components of the merge become its fields. For
example, $48  \bbcomma   \mathbf{true} $ will compile to $\ottsym{\{}   \mathtt{int}   \Mapsto  48  ; \,   \mathtt{bool}   \Mapsto   \mathbf{true}   \ottsym{\}}$. The
labels in the record, which we call \emph{type indices}, are generated from the
type of each term. As for nested merges, we also flatten them in one record.
Instead of the nested pairs $ \langle   \langle  48 ,   \mathbf{true}   \rangle  ,   \text{`\texttt{a}'}   \rangle $, $48  \bbcomma   \mathbf{true}   \bbcomma   \text{`\texttt{a}'} $ is translated
into a record of three fields: $\ottsym{\{}   \mathtt{int}   \Mapsto  48  ; \,   \mathtt{bool}   \Mapsto   \mathbf{true}   ; \,   \mathtt{char}   \Mapsto   \text{`\texttt{a}'}   \ottsym{\}}$. The
disjointness constraint on merging ensures that the components of a merge have
non-overlapping types, hence the fields of the elaborated record are
conflict-free (e.g. a merge cannot contain both \textsf{48} and \textsf{46}).
The idea of using labels based on types is similar to \emph{type-indexed
rows}~\citep{shields2001type}, though their type system does not involve
subtyping at all.

The record design significantly reduces the cost of projections. For
$48  \bbcomma   \mathbf{true}   \bbcomma   \text{`\texttt{a}'} $, we would not need to project twice to find the exact position
when selecting the integer. With a single projection, a component in an $n$-level
merge can be extracted. Besides, record fields are order-irrelevant, which
allows us to treat permuted intersection types equivalently. Using our approach,
coercing a term from type $ \mathbf{Int}   \, {\&} \,   \mathbf{Bool}   \, {\&} \,   \mathbf{Char} $ to type $ \mathbf{Char}   \, {\&} \,   \mathbf{Int}   \, {\&} \,   \mathbf{Bool} $ has
\emph{no cost}, because the elaborated record does not change. In other words,
$\ottsym{\{}   \mathtt{int}   \Mapsto  48  ; \,   \mathtt{bool}   \Mapsto   \mathbf{true}   ; \,   \mathtt{char}   \Mapsto   \text{`\texttt{a}'}   \ottsym{\}}$ and $\ottsym{\{}   \mathtt{char}   \Mapsto   \text{`\texttt{a}'}   ; \,   \mathtt{int}   \Mapsto  48  ; \,   \mathtt{bool}   \Mapsto   \mathbf{true}   \ottsym{\}}$ are equivalent. In CP, multi-field record types are also
represented as intersection types. For example, $\ottsym{\{}  \ell_{{\mathrm{1}}}  :   \mathbf{Int}   ; \,  \ell_{{\mathrm{2}}}  :   \mathbf{Int}   \ottsym{\}}$ is
syntactic sugar for $\ottsym{\{}  \ell_{{\mathrm{1}}}  :   \mathbf{Int}   \ottsym{\}}  \, {\&} \,  \ottsym{\{}  \ell_{{\mathrm{2}}}  :   \mathbf{Int}   \ottsym{\}}$. Therefore, the order of fields in a
record type does not matter either. We will develop a comprehensive theory that
accounts for type equivalence and handles all possible cases next.

\section{Reducing Coercions for Equivalent Types} \label{sec:eqty}

Coercive subtyping is inevitable in CP, so the performance penalties caused by
coercions cannot be neglected. Following the line of discussion above, an
important optimization that we identify is to avoid coercions for subtyping
between equivalent types, whose impact will be benchmarked in
\autoref{sec:optimization}. In our translation scheme, some syntactically
different types are translated to the same type index. These types that are
treated equivalently after compilation are called \emph{equivalent types}
(denoted by $\ottnt{A}  \fallingdotseq  \ottnt{B}$). The design of equivalent types is inherently determined
by the fact that we represent merges as records. We do not need to distinguish
two types after compilation if their terms are compiled to records of exactly
the same shape. The most interesting types in our compilation scheme are:
\begin{itemize}
\item \emph{Top-like types}~\citep{oliveira2016disjoint}, which correspond to
empty records because they do not convey any information.
\item \emph{Intersection types}, which correspond to multi-field records.
Generally speaking, records are order-irrelevant and contain no duplicate labels
(or duplicate labels are allowed but fields with the same label have equivalent
values).
\end{itemize}
Considering the characteristics of our record-based representation, we can first
derive that all top-like types are equivalent. In addition, two intersection
types are considered equivalent if and only if they are formed using any
combination of the following three criteria:
\begin{itemize}
\item They are permutations of the same set of types, or
\item They are equivalent after deduplicating type components, or
\item They are equivalent after removing top-like components.
\end{itemize}
The rules for other types are structural, ensuring that the type equivalence is
a congruence.

Although we work hard to reduce the number of coercions, coercions cannot be
fully eliminated. Next, we will explain the reason why they are still necessary
to CP.

\section{Necessity of Coercions}

In CP, our interpretation of subtyping is
\emph{coercive}~\citep{luo2013coercive}, in contrast to the inclusive (also
called subsumptive) view of subtyping. That is, a value of a subtype is not a
value of a supertype directly, but it contains sufficient information so that it
can be converted into a value of a supertype. Such conversions are generated by
subtyping derivations and are inserted by the subsumption rule during type
checking.

The need for coercive subtyping in CP mainly comes from the unambiguity
constraint on merging, for which the redundant information in expressions could
be harmful. For example,
\[
\textbf{let } \ottmv{x} = 48  \bbcomma   \mathbf{true}  \textbf{ in }  \mathbf{not}  \, \ottsym{(}  \ottmv{x}  :   \mathbf{Int}   \bbcomma   \mathbf{false}   \ottsym{)}
\]
can evaluate to both $ \mathbf{true} $ and $ \mathbf{false} $ if the boolean component in
$\ottmv{x}$ is kept. During typing, we use disjointness checks to ensure the static
types of the components to be merged ($ \mathbf{Int} $ and $ \mathbf{Bool} $ in this example)
do not overlap. But the soundness of such checks is based on the assumption that
any expression's dynamic type corresponds to its static type. That is,
$\ottmv{x}  :   \mathbf{Int} $ should contain nothing other than an integer at run time. So we have
to coerce $\ottmv{x}$ from $\ottsym{\{}   \mathtt{int}   \Mapsto  48  ; \,   \mathtt{bool}   \Mapsto   \mathbf{true}   \ottsym{\}}$ to a record that only
contains the integer field. With some simplification, the whole expression
should compile to:
\[
\textbf{let } \ottmv{x} = \ottsym{\{}   \mathtt{int}   \Mapsto  48  ; \,   \mathtt{bool}   \Mapsto   \mathbf{true}   \ottsym{\}} \textbf{ in }  \mathbf{not}  \, \ottsym{(}  \ottsym{\{}   \mathtt{int}   \Mapsto  \ottmv{x}  \ottsym{.}   \mathtt{int}   \ottsym{\}}  \,{+}\!{+}\,  \ottsym{\{}   \mathtt{bool}   \Mapsto   \mathbf{false}   \ottsym{\}}  \ottsym{)}  \ottsym{.}   \mathtt{bool} 
\]
where $ \,{+}\!{+}\, $ denotes runtime record concatenation, which is a key feature of
extensible records. In summary, there is a strong correspondence between the
value and its static type in CP. So we can directly tell from the declared type
how many fields the compiled record has and what the labels are. This design
resolves the issue of interaction between merging and subtyping in
\autoref{sec:merge} and is key to the type safety of dynamic trait inheritance.

\paragraph{Distributive subtyping.}
Normally, coercions are just removing redundant fields from a compiled record.
For example, we coerce $\ottmv{x}$ from $\ottsym{\{}   \mathtt{int}   \Mapsto  48  ; \,   \mathtt{bool}   \Mapsto   \mathbf{true}   \ottsym{\}}$ to
$\ottsym{\{}   \mathtt{int}   \Mapsto  48  \ottsym{\}}$ in the previous example. This is because a supertype of an
intersection type consists of part of the component types, so the compiled
record of the supertype contains a subset of the original fields. However, the
situation becomes complicated in the presence of \emph{distributive} subtyping.
For example, a function of type $\ottsym{(}   \top   \rightarrow   \mathbf{Int}   \ottsym{)}  \, {\&} \,  \ottsym{(}   \top   \rightarrow   \mathbf{Bool}   \ottsym{)}$ can be coerced to
type $ \top   \rightarrow   \mathbf{Int}   \, {\&} \,   \mathbf{Bool} $ because the former is a subtype of the latter via
distributivity. The coercion is not removing fields but merging two functions
into a single one.

Let us consider a more practical example based on the expression problem in
\autoref{sec:ep}. Here is a \emph{simplified} version of what happens to the
constructors for numeric literals when we compose the evaluation and
pretty-printing operations:
\begin{lstlisting}
ep = { Lit = \n -> { eval = n } } , { Lit = \n -> { print = toString n } };
-- : { Lit: Int -> Eval }         & { Lit: Int -> Print }
\end{lstlisting}
As the intersection type indicates, \lstinline{ep} should compile to a two-field
record: one field stores the constructor for \lstinline{Eval} and the other for
\lstinline{Print}. According to the subtyping relation, via distributivity, it
can be used as if it has type \lstinline|{ Lit: Int -> Eval&Print }|:
\begin{lstlisting}
ep.Lit 48  --> { eval = 48; print = "48" }
\end{lstlisting}
However, such usage expects that the compiled record from \lstinline{ep} only
has one field, whose label corresponds to \lstinline|{ Lit: Int -> Eval&Print }|.
Unfortunately, as we showed before, the compiled record actually contains two
different labels from the expected one, so the subtyping does not automatically
work. That is why we need to insert a coercion here to convert the two-field
record to a new one with one single field, which is similar to the previous
example of merging two functions.

Our compilation scheme is designed to avoid coercions as much as possible. The
aforementioned coercion is not inserted for \emph{direct} usage of record
projections or function applications. Instead, the compiled code will select the
two functions from the two fields for \lstinline{ep.Lit} and apply both to 48.
The results are then combined into a record so that both \lstinline{eval} and
\lstinline{print} fields are present.

\section{Implementation in JavaScript}

The extensible records that we have been mentioning are an abstract data type
that supports construction, concatenation, and projection. They do not imply any
concrete data structure in any particular programming language. They can be
implemented as hash tables, binary search trees, or even association lists, and
most mainstream languages have built-in and highly optimized support for these
data structures. In our implementation, extensible records are implemented as
\emph{JavaScript objects}, whose underlying data structure still varies among
JavaScript engines. Nevertheless, one thing we are certain of is that accessing
properties of an object, which corresponds to record projection in our
terminology, is highly optimized in the various engines.


The CP compiler supports modular type checking and separate compilation. In
other words, compiling a CP file does not require access to the source code of
the libraries that it depends on. What is needed is only the header files of the
libraries, which mainly contain type information. Separate compilation largely
decreases the rebuilding time since it avoids recompiling its dependencies, and
it allows closed-source distribution of libraries. More details about the
implementation of separate compilation can be found in \autoref{sec:separate}.

\paragraph{Type indices.} \label{sec:index}
In our implementation, type indices are represented by JavaScript strings
(hereinafter, \lstinline{"string"} is in violet and monospaced). Below is how we
represent different types:
\begin{itemize}
\item Primitive types are simply represented by their names,
      e.g. \lstinline{"int"} for $ \mathbf{Int} $.
\item Functions are represented by their return types,
      e.g. \lstinline{"func_int"} for $ \mathbf{String}   \rightarrow   \mathbf{Int} $.
\item Records are represented by both labels and field types,
      e.g. \lstinline{"rcd_l:int"} for $\ottsym{\{}  \ell  :   \mathbf{Int}   \ottsym{\}}$.
\item Intersections are represented by joining the representations of their
      components after alphabetical sorting, deduplication, and removal of
      top-like types, e.g. \lstinline{"(bool&int)"} for $ \mathbf{Int}   \, {\&} \,   \mathbf{Bool}   \, {\&} \,   \top   \, {\&} \,   \mathbf{Bool} $.
      Note that such type indices only occur when intersection types are nested
      within functions or records. A top-level intersection corresponds to a
      multi-field record, which has separate type indices for each component.
\end{itemize}
The representation for function types may be a bit surprising. It originates
from the disjointness rule for function types: two function types are disjoint
if and only if their return types are disjoint (\rref{D-ArrowArrow}). This rule
is derived from the specification of disjointness (\autoref{thm:disjoint}),
which basically means that two disjoint types do not overlap on any meaningful
types. For example, \lstinline{Int -> Int} and \lstinline{Bool -> Int} shares a
common supertype \lstinline{Int&Bool -> Int}, so these two types are \emph{not}
disjoint. If those types are considered to be disjoint, we could have the
following application:
\begin{lstlisting}
((\(x: Int) -> x + 1),(\(x: Bool) -> if x then 1 else 0)) (1,false)
\end{lstlisting}
Note that both functions can be selected, and we get either 2 or 0 depending on
which function we pick. The semantics would be ambiguous in this way. Thus,
allowing such merges is unsafe. That is why \lstinline{Int -> Int} and
\lstinline{Bool -> Int} are not disjoint, and \lstinline{"func_int"} cannot
occur twice. The disjointness checks in CP rule out the possibility of type
index conflicts between two functions in a merge. Our design that includes only
return types also avoids very long property names in JavaScript, which may lead
to performance issues.

\paragraph{Compiling parametric polymorphism.}
As we have discussed previously, dynamic inheritance and family polymorphism are
already difficult to handle. In those examples, parametric polymorphism also
plays an important role, yet we have not mentioned the difficulty of compiling
it. The reason why this feature is challenging to compile is a bit more
technical: it relates to when to build type indices, namely the labels of the
compiled records.

For non-polymorphic types, the labels remain fixed throughout the program
execution. However, for polymorphic types, we have to deal with \emph{type
instantiation}. For example, we may have a source type \lstinline|{ f: A -> A }|,
where the type \lstinline|A| is a type variable. After the instantiation of
\lstinline|A|, we may have the type \lstinline|{ f: Int -> Int }| or perhaps
the type \lstinline|{ f: Bool -> Bool }|. The problem is that different
instantiations of polymorphic type variables will produce different labels. So
for polymorphic types, the labels cannot be statically computed. To solve this
problem, first-class labels~\citep{leijen2004first} are needed so that
polymorphic instantiation can build a label at run time and propagate the label
that corresponds to the instantiated type. A more detailed explanation with
examples can be found in \autoref{sec:poly}.

\paragraph{Important optimizations.}
In our implementation, we have applied several optimizations to improve the
performance of the generated JavaScript code. Besides the elimination of
redundant coercions based on equivalent types in \autoref{sec:eqty}, some
important optimizations are:
\begin{enumerate}
\item Reducing intermediate objects using destination-passing
      style~\citep{shaikhha2017destination};
\item Reducing object copying by detecting whether the compiled term is part of
      a merge;
\item Limiting lazy evaluation to certain trait fields to improve performance;
\item Preventing primitive values from boxing/unboxing;
\item Avoiding the insertion of coercions for record projections.
\end{enumerate}
These optimizations will be elaborated with examples in
\autoref{ch:compilation}, and their impact on performance will be evaluated in
\autoref{sec:optimization}. Among the five optimizations, the last one (5\th) is
formalized.
