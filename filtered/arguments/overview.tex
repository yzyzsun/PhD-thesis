\section{Our Type-Safe Approach}

In this section, we informally present how we translate named and optional
arguments into a core language with intersection and union types, while
retaining type safety. We start by introducing the core language constructs that
we need. Then we illustrate our translation scheme by example and demonstrate
how it recovers type safety. After that, we showcase a practical example in the
CP language, which has incorporated our approach to support named and optional
arguments. Finally, we discuss how our translation scheme can be applied to
other languages.

\subsection{Core Language} \label{sec:core}

The core language features intersection and union types, which establish an
elegant duality in the type system. Intersection and union types correspond to
the logical conjunction and disjunction respectively. Intersections are used to
combine multiple types, while unions are used to express alternatives. Similar
calculi are widely
studied~\citep{barbanera1995intersection,frisch2008semantic,dunfield2014elaborating}
and provide a well-understood foundation for named and optional arguments.

\paragraph{Named Arguments as Intersections.}
Named arguments are translated to multi-field records. However, the core
language does not support multi-field records directly. There are only
single-field records in the core language, and multiple fields are represented
as intersections of single-field record types. For example,
$\ottsym{\{}  \ottmv{x}  \ottsym{:}   \mathbb{Z}   \ottsym{\}}  \land  \ottsym{\{}  \ottmv{y}  \ottsym{:}   \mathbb{Z}   \ottsym{\}}$ represents a record type with two integer fields $x$ and
$y$. With intersection types, width subtyping for record types comes for free,
and permutations of record fields are naturally
allowed~\citep{reynolds1997design}.

At the term level, a merge
operator~\citep{dunfield2014elaborating,rehman2023blend} is used to concatenate
multiple single-field records to form multi-field records, reminiscent of
Forsythe~\citep{reynolds1997design}. For example, $\ottsym{\{}  \ottmv{x}  \ottsym{=}  \ottsym{1}  \ottsym{\}}  \bbcomma  \ottsym{\{}  \ottmv{y}  \ottsym{=}  \ottsym{2}  \ottsym{\}}$ forms a
two-field record from two single-field records.

\paragraph{Optional Arguments as Unions.}
Optional arguments are translated to nullable types. A nullable type is not
implicit in the core language but is represented as a union with the null
type~\citep{nieto2020scala}. For example, an optional integer argument named $z$
is translated to $\ottsym{\{}  \ottmv{z}  \ottsym{:}   \mathbb{Z}   \lor  \ottkw{Null}  \ottsym{\}}$.

At the term level, a type-based switch
expression~\citep{frisch2008semantic,rehman2023blend} is used to scrutinize a
term of a union type, reminiscent of ALGOL~68~\citep{van1975revised}. For
example, $\ottkw{switch} \, \ottmv{z} \, \ottkw{case} \,  \mathbb{Z}   \Rightarrow  \ottnt{e_{{\mathrm{1}}}} \, \ottkw{case} \, \ottkw{Null}  \Rightarrow  \ottnt{e_{{\mathrm{2}}}}$ returns $\ottnt{e_{{\mathrm{1}}}}$ if $z$
is an integer or $\ottnt{e_{{\mathrm{2}}}}$ if $\ottkw{null}$.

\subsection{Translation by Example} \label{sec:example}

Let us review the previous Python function in \autoref{sec:mut} that appends an
element to a list, which defaults to an empty list:
\begin{lstlisting}[language={[3]Python}]
def append(x: int, xs: list[int] = []): ...
\end{lstlisting}
The function will be translated to a core function as follows:
\begin{align*}
\mathtt{append} =~&  \lambda \ottmv{args} \!:\! \ottsym{\{}  \ottmv{x}  \ottsym{:}   \mathbb{Z}   \ottsym{\}}  \land  \ottsym{\{}  \ottmv{xs}  \ottsym{:}  \ottsym{[}   \mathbb{Z}   \ottsym{]}  \lor  \ottkw{Null}  \ottsym{\}} .\;     \\
                  & \ottkw{let} \, \ottmv{x}  \ottsym{=}  \ottmv{args}  \ottsym{.}  \ottmv{x} \, \ottkw{in} \\
                  & \ottkw{let} \, \ottmv{xs}  \ottsym{=}  \ottkw{switch} \, \ottmv{args}  \ottsym{.}  \ottmv{xs} \, \ottkw{as} \, \ottmv{xs} \, \ottkw{case} \,  \mathbb{Z}   \Rightarrow  \ottmv{xs} \, \ottkw{case} \, \ottkw{Null}  \Rightarrow   [\,]  \, \ottkw{in} \\
                  & \cdots
\end{align*}
Here we can see that the default value (i.e. the empty list) is not shared
across different calls to \lstinline{append} because the default value is
evaluated within the function body. Therefore, calling \lstinline{append(x=1)}
will consistently return \lstinline{[1]} instead of surprisingly modifying the
default value. This design leads to less astonishment and more predictable
behavior.

Since we translate named parameter types to record types, we correspondingly
translate named arguments to records. For example, the function call
\lstinline{append(x=1, xs=[0])} will be translated to {\small
$\mathtt{append}\,\ottsym{(}  \ottsym{\{}  \ottmv{x}  \ottsym{=}  \ottsym{1}  \ottsym{\}}  \bbcomma  \ottsym{\{}  \ottmv{xs}  \ottsym{=}  \ottsym{[0]}  \ottsym{\}}  \ottsym{)}$}.

\paragraph{Rewriting Call Sites.}
More importantly, we also rewrite call sites to add null values for absent
optional arguments. For example, the function call \lstinline{append(x=1)} will
be rewritten and translated to {\small
$\mathtt{append}\,\ottsym{(}  \ottsym{\{}  \ottmv{x}  \ottsym{=}  \ottsym{1}  \ottsym{\}}  \bbcomma  \ottsym{\{}  \ottmv{y}  \ottsym{=}  \ottkw{null}  \ottsym{\}}  \ottsym{)}$}.

\paragraph{Dependent Default Values.}
Another advantage of our translation scheme is that it naturally allows default
values to depend on earlier arguments. Python and Ruby do not support dependent
default values, but this feature can be useful in some practical scenarios. For
example, when setting up I/O, we may want to output error messages to the same
stream as \lstinline{out} by default:
\begin{lstlisting}[language={[3]Python}]
def setIO(in_, out, err = out): ...
\end{lstlisting}
The variable \lstinline{out} can be used in the default value of \lstinline{err}
because it has been brought into scope by the previous let-in binding:
\begin{align*}
& \ottkw{let} \, \ottmv{out}  \ottsym{=}  \ottmv{args}  \ottsym{.}  \ottmv{out} \, \ottkw{in} \\
& \ottkw{let} \, \ottmv{err}  \ottsym{=}  \ottkw{switch} \, \ottmv{args}  \ottsym{.}  \ottmv{err} \, \ottkw{as} \, \ottmv{err} \, \ottkw{case} \, \ottkw{IO}  \Rightarrow  \ottmv{err} \, \ottkw{case} \, \ottkw{Null}  \Rightarrow  \ottmv{out} \, \ottkw{in}
\end{align*}

\subsection{Recovering Type Safety}

The type safety of our translation scheme is essentially guaranteed by call site
rewriting. Besides adding null values for absent optional arguments, we also
sanitize arguments to ensure that they are expected from the parameter list.
Since named arguments are first-class and can be passed as a variable, we may
not have literals like \lstinline{append(x=1, xs=[0])} but splats like
\lstinline{append(**args)}. So the matching between (formal) parameters and
(actual) arguments is performed based on their static types:
\begin{itemize}
\item If \lstinline{args} has type $\ottsym{\{}  \ottmv{x}  \ottsym{:}   \mathbb{Z}   \ottsym{\}}  \land  \ottsym{\{}  \ottmv{xs}  \ottsym{:}  \ottsym{[}   \mathbb{Z}   \ottsym{]}  \ottsym{\}}$, the call site will be
      rewritten to something equivalent to \lstinline{append(x=args.x, xs=args.xs)}.
\item If \lstinline{args} only has type $\ottsym{\{}  \ottmv{x}  \ottsym{:}   \mathbb{Z}   \ottsym{\}}$, the call site will be
      rewritten to something equivalent to \lstinline{append(x=args.x, xs=null)}.
\end{itemize}
For the \lstinline{append} function, no other cases can pass the sanitization process.

Let us review the previous type-unsafe Python example in \autoref{sec:mypy}:
\begin{lstlisting}[language={[3]Python}]
def f(args: In) -> Out: return args
args = f({ "host": "0.0.0.0", "port": 80, "debug": "Oops!" })
app.run(**args) #= app.run(host="0.0.0.0", port=80, debug="Oops!")
\end{lstlisting}
Recall that \lstinline{args} has type \lstinline{Out}, which is similar to
\lstinline|{ host: str, port: int }|. The \lstinline{debug} key is forgotten in
the static type but is still present at run time. It passes mypy's type checking
but raises a runtime error. In our translation scheme, the call site will be
rewritten to the following form based on the type of \lstinline{args} (i.e.
\lstinline{Out}):
\begin{lstlisting}[language={[3]Python}]
app.run(host=args.host, port=args.port, debug=null)
\end{lstlisting}
Therefore, type safety is recovered in our translation scheme.

\paragraph{Takeaways.}
There are two important observations from our translation scheme:
\begin{enumerate}
\item \lstinline|{ required: A, optional?: B }| is not equivalent to
      \lstinline|{ required: A }| because the former contains more information
      that prevents \lstinline{optional} from being associated with other types
      than \lstinline{B}. In other words, the \lstinline{optional} argument can
      be absent, but if it is present, it must have type \lstinline{B}.
\item Corresponding to the above observation at the type level, we explicitly
      pass a null value as an optional argument if it is statically missing. The
      null value fills the position of a potentially forgotten argument that may
      have a wrong type. In other words, we implement the splat operator as per
      the static type of named arguments.
\end{enumerate}

\subsection{Implementation in the CP Language}

Our approach to named and optional arguments has been implemented in the CP
language. CP supports not only intersection and union types but also the merge
operator and type-based switch expression. The implementation of named and
optional arguments in CP is a direct application of our translation scheme.

More interestingly, the sanitization process during call site rewriting comes
for free because CP employs a \emph{coercive} semantics for
subtyping~\citep{luo2013coercive}. For example, a subtyping relation between
\lstinline[language={[3]Python}]|{ host: str, port: int, debug: str }| and
\lstinline[language={[3]Python}]|{ host: str, port: int }| implies a coercion
function from subtype to supertype. In CP, such coercions are implicitly
inserted to remove the forgotten fields (e.g. \lstinline{debug} in this case).
Therefore, the only remaining work is to add null values for absent optional
arguments.

\begin{figure}[b!]
\begin{lstlisting}[language=CP]
-- from a SVG library in CP
SVG: { width: Int; height: Int } -> [Element] -> Graphic;  -- <svg>
Rect: { x: Int; y: Int; width: Int; height: Int
      ; rx?: Int; ry?: Int; color?: String } -> Element;   -- <rect>
......
-- client code
fractal { level = 4; x: Int; y: Int; width: Int; height: Int } =
  let args = { level = level-1; width = width/3; height = height/3 } in
  let center = Rect (args,{ x = x + width/3; y = y + height/3
                          ; color = "white" }) in
  if level == 0 then [center]
  else fractal (args,{ x = x;             y = y })
    ++ fractal (args,{ x = x + width/3;   y = y })
    ++ fractal (args,{ x = x + width*2/3; y = y })
    ++ fractal (args,{ x = x;             y = y + height/3 })
    ++ [center]
    ++ fractal (args,{ x = x + width*2/3; y = y + height/3 })
    ++ fractal (args,{ x = x;             y = y + height*2/3 })
    ++ fractal (args,{ x = x + width/3;   y = y + height*2/3 })
    ++ fractal (args,{ x = x + width*2/3; y = y + height*2/3 });
init = { x = 0; y = 0; width = 600; height = 600; color = "black" };
main = SVG init ([Rect init] ++ fractal init);
\end{lstlisting}
\caption{Sierpiński carpets implemented in the CP language.} \label{fig:fractal}
\end{figure}

To demonstrate the use of named and optional arguments in CP, we show a fractal
example in \autoref{fig:fractal}, which is adapted from code in
\autoref{sec:fractal}. The code makes use of named and optional arguments a lot,
including both the \lstinline{SVG}/\lstinline{Rect} constructors from the
library and the \lstinline{fractal} function defined by the client. For example,
\lstinline{fractal} has five named arguments (\lstinline{level}, \lstinline{x},
\lstinline{y}, \lstinline{width}, and \lstinline{height}), among which
\lstinline{level} is optional with a default value of \lstinline{4}.

It is worth noting that named arguments are used as first-class values in the CP
code. On the first line of the \lstinline{fractal} body, we store three fields
\lstinline{level}, \lstinline{width}, and \lstinline{height} in a variable
\lstinline{args}, which are shared arguments for later calls. When constructing
the center rectangle, we merge \lstinline{args} with three more fields
\lstinline{x}, \lstinline{y}, and \lstinline{color} to form a full set of named
arguments we need for the \lstinline{Rect} constructor. When recursively calling
\lstinline{fractal}, we pass \lstinline{args} merged with different
\lstinline{x} and \lstinline{y} values to draw the eight sub-copies. In the
\lstinline{main} function, we also use a variable \lstinline{init} to avoid
repeating the same set of arguments for \lstinline{SVG}, \lstinline{Rect}, and
\lstinline{fractal} calls. The \lstinline{**} operator is not needed in CP when
passing first-class named arguments. Note that the parameter lists of these
three constructors/functions are not completely the same, but we can still use a
larger set of named arguments to cover all the cases. This is possible because
CP allows subtyping for named arguments while retaining type safety.

\subsection{Applications to Other Languages}

Although we base our translation scheme on a core language with intersection and
union types for type-theoretic solidness and elegance, it can work for a wider
range of languages. We discuss the alternatives to intersections and unions
below.

\paragraph{Alternative to Intersections.}
Record types have existed long before intersection types were invented. In
practice, multi-field records are rarely represented as intersections of
single-field records. For example, \emph{Software Foundations}~\citep{plf}
demonstrates how to directly model multi-field records and define depth, width,
and permutation subtyping without intersections, though their formalization is
more complex than ours.

There is a merge operator in our translation scheme, but we only use it to
construct multi-field records statically. Although the merge operator can be
powerful if we want to construct first-class named arguments at run time like in
CP, its absence does not disable our translation scheme. In other words, we only
assume a simplified version that does not merge terms dynamically.

\paragraph{Alternative to Unions.}
Nullable types are rarely represented as unions with the null type too. For
example, C\#, Kotlin, and Dart support nullable types as a primitive data
structure. Putting a question mark behind any type makes it nullable in these
languages (e.g. \lstinline{int?}).

No matter how a nullable type is represented, there is usually some expression
that can check whether a nullable value is null or not. For example, C\#
provides the \lstinline[language={[Sharp]C}]{is} operator to examine the runtime
type, which is generally known as type introspection and is similar to the
type-based switch. C\# also provides the null coalescing operator \lstinline{??}
and simplifies the common pattern $\ottkw{switch} \, \ottnt{e} \, \ottkw{as} \, \ottmv{x} \, \ottkw{case} \, \ottnt{A}  \Rightarrow  \ottmv{x} \, \ottkw{case} \, \ottkw{Null}  \Rightarrow  \ottnt{d}$
as \lstinline{e??d} for nullable values.

\subsubsection{Dynamically Typed Languages.}
It may be surprising at first sight that dynamically typed languages can benefit
from our work with static typing, but recall that the type-safety issue in
\autoref{sec:mypy} was found in Python. Nowadays, popular dynamically typed
languages have been retrofitted with gradual typing. For example, Python has
type hints and mypy~\citep{mypy}, Ruby has RBS and Steep~\citep{steep},
JavaScript gets typed by TypeScript~\citep{typescript}, and Lua gets typed by
Luau~\citep{luau}. All of these typed versions support record-like and union
types, and all except Python also support intersection types. Our translation
scheme can almost directly apply to these languages. For a concrete example, we
show how the aforementioned \lstinline{exp} function can be encoded in
TypeScript:
\begin{lstlisting}[language=TypeScript]
function exp(args: { x: number } & { base: number|null }) {
  let x = args.x;
  let base = (typeof args.base === "number") ? args.base : Math.E;
  return Math.pow(base, x);
}
exp({ base: 2, x: 10 })     //= 1024
exp({ x: 10, base: null })  //= e^10
\end{lstlisting}
The code is almost the same as in \autoref{sec:example}. Note that the
\lstinline[language=TypeScript]{typeof} operator is the standard way to perform
type introspection in TypeScript, and the type of \lstinline{args.base} is
refined from the \lstinline[language=TypeScript]{number|null} to
\lstinline[language=TypeScript]{number} in the true-branch. We assume the call
sites have been rewritten in the code above. In this manner, named and optional
arguments can be added to TypeScript as syntactic sugar.

Although we have discussed several alternatives to intersection and union types,
we believe that if a language is designed from scratch, our approach is a good
choice. Intersection and union types not only subsume multi-field record and
nullable types but also provide a solid and elegant foundation for other
advanced features, such as function overloading and heterogeneous data
structures. The essay by \citet{castagna2023programming} is an excellent further
reading on the beauty of programming with intersection and union types.
