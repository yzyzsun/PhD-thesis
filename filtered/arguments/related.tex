\section{Discussion} \label{sec:related}

In this section, we first discuss OCaml, the only language we know of that has
well-studied support for named and optional arguments, though its mechanism does
not go well with higher-order functions. Then we briefly show how named
arguments are handled very differently in Scala and Racket. Finally, we
illustrate how named arguments can be encoded as records in Haskell while not
natively supported. We will also explain why all these approaches have
drawbacks.

\subsection{OCaml}

OCaml did not support named arguments originally. Nevertheless,
\citet{garrigue1994label} conducted research on the label-selective
$\lambda$-calculus and implemented it in OLabl~\citep{olabl}, which extends
OCaml with labeled and optional arguments, among others. All features of OLabl
were merged into OCaml 3, despite subtle
differences~\citep{garrigue2001labeled}.

Here is an example of the exponential function defined in a labeled style:

\begin{lstlisting}[language={[Objective]Caml}]
let exp ?(base = Float.exp 1.0) x = base ** x
(* val exp : ?base:float -> float -> float *)
exp 10.0              (*= e^10. *)
exp 10.0 ~base:2.0    (*= 1024. *)
(exp 10.0) ~base:2.0  (* TypeError! *)
\end{lstlisting}

\noindent In the definition of \lstinline{exp}, \lstinline{base} is an optional
labeled parameter while \lstinline{x} is a positional parameter. Changing
\lstinline{x} into a second labeled parameter will trigger an
unerasable-optional-argument warning because OCaml expects that there should be
a positional parameter after all optional parameters. This expectation is at the
heart of how OCaml resolves the ambiguity introduced by currying.

For example, consider the function application \lstinline{exp 10.0}. Is it a
partially applied function or a fully applied one using the default value of
\lstinline{base}? Both interpretations are possible, but OCaml considers it to
be a full application because the trailing positional argument \lstinline{x} is
given. The presence of the positional argument is used to indicate that the
optional arguments before it can be replaced by their default values. However,
this design may confuse users since \lstinline{(exp 10.0) ~base:2.0} will raise
a type error but \lstinline{exp 10.0 ~base:2.0} will not. Partial application
does not lead to an equivalent program in such situations.

\paragraph{Option types.}
In OCaml, an optional argument is internally represented as an \lstinline{option}
type, which comprises two constructors: \lstinline{None} and \lstinline{Some}.
Here is an equivalent definition for \lstinline{exp}:

\begin{lstlisting}[language={[Objective]Caml}]
let exp ?(base : float option) x =
  let base = match base with
             | None -> Float.exp 1.0
             | Some b -> b in
  base ** x
(* val exp : ?base:float -> float -> float *)
exp 10.0            (*> exp 10.0 ~base:None *)
exp 10.0 ~base:2.0  (*> exp 10.0 ~base:(Some 2.0) *)
\end{lstlisting}

\noindent This encoding is similar to union types, but it depends on the
\lstinline{option} type in the standard library. Unfortunately, this specific
kind of \lstinline{option} is not a built-in type in many mainstream languages,
especially in those languages that do not support algebraic data types.

\paragraph{Higher-order functions.}
A surprising gotcha in OCaml is that the commutativity breaks down when we pass
a function with labeled arguments to another function. \emph{Real World
OCaml}~\citep{madhavapeddy2022real} gives the following example:
\begin{lstlisting}[language={[Objective]Caml}]
let apply1 f (fst,snd) = f ~fst ~snd
(* val apply1 : (fst:'a -> snd:'b -> 'c) -> 'a * 'b -> 'c *)
let apply2 f (fst,snd) = f ~snd ~fst
(* val apply2 : (snd:'a -> fst:'b -> 'c) -> 'b * 'a -> 'c *)
let divide ~fst ~snd = fst / snd
(* val divide : fst:int -> snd:int -> int *)
apply1 divide (48,3)  (*= 16 *)
apply2 divide (48,3)
(* TypeError: "divide" has type fst:int -> snd:int -> int
       but was expected of type snd:'a -> fst:'b -> 'c *)
\end{lstlisting}
Normally, the order of named arguments does not matter in OCaml, so it
type-checks whether we call \lstinline{divide ~fst ~snd} or
\lstinline{divide ~snd ~fst}. However, order matters when we pass
\lstinline{divide} to a higher-order function. That is why
\lstinline{apply1 divide} type-checks while \lstinline{apply2 divide} does not.
It turns out that the OCaml way of handling labeled arguments does not go well
with other features like higher-order functions. Our approach scales better in
this regard and the commutativity still holds in higher-order contexts via
intersection subtyping.

In short, OCaml has a very powerful label-selective core calculus that
reconciles commutativity and currying, but it is quite complicated and may
hinder its integration with other language features. Another thing worth
mentioning is that labeled arguments in OCaml are not first-class values, so
they cannot be assigned to a variable or passed around by functions. In contrast
to OCaml, our approach supports first-class named arguments and targets a
minimal core calculus with intersection and union types, which is compatible
with many popular languages like Python, Ruby, JavaScript, etc.

\subsection{Scala}

\citet{rytz2010named} described the design of named and default arguments in
Scala. Like in Python, parameter names in a method definition are non-mandatory
keywords in Scala, and thus every argument can be passed with or without
keywords. Furthermore, the parameter names are not part of the public interface
of a method. This design is partly due to the backward compatibility with
earlier versions of Scala, so the addition of named arguments will not break any
existing code. As a result of the conservative treatment, named arguments are
not first-class values in Scala and cannot be defined in an anonymous function.
In short, named arguments are more like syntactic sugar in Scala and do not
interact with the type system.

Below we show an example in Scala. In order to let the default value of
\lstinline{c} depend on \lstinline{a} and \lstinline{b}, we make the function
partly curried:
\begin{lstlisting}[language=Scala]
def f(a: Int, b: Int)(c: Int = a+b) = c
f(b = 1+1, a = 1)()
\end{lstlisting}
The code will be translated to equivalent code without keywords or defaults:
\begin{lstlisting}[language=Scala]
def f(a: Int, b: Int)(c: Int) = c
def f$default$3(a: Int, b: Int): Int = a+b
{
  val x$1 = 1+1
  val x$2 = 1
  val x$3 = f$default$3(x$2, x$1)
  f(x$2, x$1)(x$3)
}
\end{lstlisting}
There are two things to note here. First, a new function \lstinline{f$default$3}
is generated for the default value of \lstinline{c}, taking two parameters
\lstinline{a} and \lstinline{b}. Second, the call site is translated to a series
of variable assignments for each argument and a keyword-free call to
\lstinline{f} with arguments reordered. The whole call site is wrapped in a
block to avoid polluting the namespace.

In conclusion, named arguments in Scala are handled in a very different way from
OCaml and our approach. The Scala way is more syntactic than type-theoretic, so
it is hard to do an apples-to-apples comparison with our approach.

\subsection{Racket}

\lstdefinelanguage{Racket}[]{Lisp}{
  morekeywords={define,lambda,keyword-apply,Number},
  literate={\#:a}{{\textbf{\color{darkgray}\#:a}}}1 {\#:b}{{\textbf{\color{darkgray}\#:b}}}1 {\#:c}{{\textbf{\color{darkgray}\#:c}}}1
           {->*}{{$\rightarrow^*$}}1
}

\citet{flatt2009keyword} introduced keyword and optional arguments into Racket,
which was known as PLT Scheme at that time. A keyword is prefixed with
{\color{gray}\lstinline{#:}} in syntax and is implemented as a new built-in type
in Racket. Keyword arguments are supported by replacing
\lstinline[language=Racket]{define}, \lstinline[language=Racket]{lambda}, and
the core application form with newly defined macros that recognize
keyword-argument forms. Here is an example of a function \lstinline{f} with
three keyword arguments \lstinline{a}, \lstinline{b}, and \lstinline{c}, among
which \lstinline{c} is optional and defaults to \lstinline{a+b}:
\begin{lstlisting}[language=Racket]
(define (f #:a a #:b b #:c [c (+ a b)]) c)
(f #:b (+ 1 1) #:a 1)
\end{lstlisting}
The function call with keywords seems hard to implement because it just lists
the function and arguments in juxtaposition. In fact, an application form in
Racket implicitly calls \lstinline{#%app} in its lexical scope, so the support
for keyword arguments is done by supplying an \lstinline{#%app} macro. A new
\lstinline[language=Racket]{keyword-apply} function is also defined to accept
keyword arguments as first-class values. For example, we can rewrite the
function call above as follows\footnote{\lstinline[language=Racket]{'} is
\lstinline{quote}, \lstinline[language=Racket]{`} is \lstinline{quasiquote}, and
\lstinline[language=Racket]{,} is \lstinline{unquote} in Racket.}:
\begin{lstlisting}[language=Racket]
(keyword-apply f '(#:a #:b) `(1 ,(+ 1 1)) '()) ; OK!
(keyword-apply f '(#:b #:a) `(,(+ 1 1) 1) '()) ; Contract violation!
\end{lstlisting}
Note that we need to separate keywords and corresponding arguments into two
lists. The third list is for positional arguments, so it is empty in this case.
We cannot list keywords in arbitrary order: a contract violation will be
signaled unless the keywords are sorted in alphabetical order. In other words,
commutativity is lost for first-class keyword arguments in Racket.

In Typed Racket~\citep{typedracket}, Racket's gradually typed sister language,
\lstinline{f} can be typed as
\lstinline[language=Racket]{(->* (#:a Number #:b Number) (#:c Number) Number)}.
The first list contains the required arguments
(\lstinline[language=Racket]{#:a} and \lstinline[language=Racket]{#:b}), and the
second list contains the optional ones (\lstinline[language=Racket]{#:c}). However,
Typed Racket does not provide a typed version of
\lstinline[language=Racket]{keyword-apply}, and it is unclear how to properly type it.

In conclusion, Racket supports keyword and optional arguments in a unique way
via its powerful macro system. However, the support for first-class keyword arguments
is very limited and cannot easily transfer to a type-safe setting.

\subsection{Haskell}

\begin{figure}
\begin{subfigure}{0.45\textwidth}
\begin{lstlisting}[language=Haskell]
data Settings = Settings
  { settingsPort :: Port
  , settingsHost :: HostPreference
  , settingsTimeOut :: Int
  , ...
  }
\end{lstlisting}
\caption{Record type.} \label{fig:settings}
\end{subfigure}
\hfill
\begin{subfigure}{0.35\textwidth}
\begin{lstlisting}[language=Haskell]
defaultSettings = Settings
  { settingsPort = 3000
  , settingsHost = "*4"
  , settingsTimeout = 30
  , ...
  }
\end{lstlisting}
\caption{Default values.} \label{fig:default}
\end{subfigure}
\par\bigskip
\begin{subfigure}{\textwidth}
\begin{lstlisting}[language=Haskell]
runSettings :: Settings -> Application -> IO ()
runSettings = ...

main :: IO ()
main = runSettings settings app
  where settings = defaultSettings { settingsPort = 4000, settingsHost = "*6" }
\end{lstlisting}
\caption{Updating some settings before running a server application.} \label{fig:update}
\end{subfigure}
\caption{Named arguments as records in Haskell.}
\end{figure}

Unlike the aforementioned languages, Haskell does not support named arguments
natively. However, the paradigm of \emph{named arguments as records} has long
existed in the Haskell community. Although we have to uncurry a function to have
all parameters labeled in a record, it is clearer and more human-readable,
especially when different parameters have the same type. For example, in the web
server library \emph{warp}~\citep{warp}, various server settings are bundled in
the data type \lstinline{Settings}, as shown in \autoref{fig:settings}. It is
obvious how named arguments correspond to record fields, but it needs some
thought on how to encode default values for optional arguments. The simplest
approach, also used by \emph{warp}, is to define a record
\lstinline{defaultSettings}, as shown in \autoref{fig:default}. Users can update
whatever fields they want to change while keeping others. For example, we update
\lstinline{settingsPort} and \lstinline{settingsHost} while keeping the rest
unchanged in \autoref{fig:update}. Finally, we call the library function
\lstinline{runSettings} with the updated \lstinline{settings} to run a server.

Such an approach works fine here but still has two drawbacks. The first issue is
the dependency on \lstinline{defaultSettings}. It is awkward for users to look
for a record containing particular default values, especially when there are a
few similar records in a library. A better solution is to change the parameter
of \lstinline{runSettings} from a complete \lstinline{Settings} to a function
that updates \lstinline{Settings}:

\begin{lstlisting}[language=Haskell]
runSettings' :: (Settings -> Settings) -> Application -> IO ()
runSettings' update = runSettings (update defaultSettings)

main :: IO ()
main = runSettings' update app
  where update settings = settings { settingsPort = 4000, settingsHost = "*6" }
\end{lstlisting}

\noindent With the new interface, users do not need to look for default values
anymore. However, this design still has a second drawback: all arguments must
have default values. Usually, we do not consider every argument to be optional.
For example, we may want to require users to fill in \lstinline{settingsPort}. A
workaround employed by \lstinline{SqlBackend} in the library
\emph{persistent}~\citep{persistent} is to have another function that asks for
required arguments and supplements default values for optional arguments:

\begin{lstlisting}[language=Haskell]
{-# language DuplicateRecordFields, RecordWildCards #-}

data ReqSettings = ReqSettings { settingsPort :: Port }

mkSettings :: ReqSettings -> Settings
mkSettings ReqSettings {..} =
              Settings { settingsHost = "*4", settingsTimeout = 30, .. }
\end{lstlisting}

\paragraph{Best practice.}
Although \lstinline{mkSettings} resolves the second issue, there is a regression
concerning the first issue: users have to look for \lstinline{mk*} functions
now. Fortunately, we can harmonize both design patterns to develop a third
approach:

\begin{lstlisting}[language=Haskell]
{-# language DuplicateRecordFields, RecordWildCards #-}

data OptSettings = OptSettings { settingsHost :: HostPreference
                               , settingsTimeOut :: Int }

runSettings'' :: (OptSettings -> Settings) -> Application -> IO ()
runSettings'' update = runSettings (update defaultSettings)
  where defaultSettings = OptSettings { settingsHost = "*4"
                                      , settingsTimeout = 30 }

main :: IO ()
main = runSettings'' update app
  where update OptSettings {..} =
                  Settings { settingsPort = 4000, settingsHost = "*6", .. }
\end{lstlisting}

\noindent This last approach is probably the best practice in Haskell, though it
is already quite complicated and requires two GHC language extensions. Of
course, there could be other approaches to encoding named and optional arguments
in Haskell. Users could get confused about the various available design
patterns. This is largely due to lack of language-level support. We believe it
is better for a language to natively support named and optional arguments.

\paragraph{Sidenote.}
The design pattern of \emph{named arguments as records} can also be found in
other functional languages like Standard ML, Elm, and PureScript, just to name a
few. It is worth mentioning that these languages have first-class support for
record types, so no separate type declarations are needed like in
\autoref{fig:settings}. However, they still suffer from lack of native support
for optional arguments.
