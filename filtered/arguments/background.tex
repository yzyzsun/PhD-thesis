\section{Named and Optional Arguments: The Bad Parts} \label{sec:bad}

Since named and optional arguments are not well studied in most languages, the
ad-hoc mechanisms employed in those languages may sometimes surprise programmers
or even cause safety issues.

\subsection{Gotcha! Mutable Default Arguments in Python} \label{sec:mut}

Let us consider a simple Python function that appends an element to a list. We
provide a default value for the list, which is an empty list:

\begin{multicols}{2}
\begin{lstlisting}[language={[3]Python}]
def append(x, xs=[]):
  xs.append(x)
  return xs
append(1) #= [1]
append(2) #= [1, 2]
\end{lstlisting}
\end{multicols}

\noindent
After calling \lstinline{append(1)} above, we get the expected result
\lstinline{[1]}. However, continuing to call \lstinline{append(2)} gives us
\lstinline{[1, 2]} instead of \lstinline{[2]}. This is because Python only
evaluates the default value once when the function is defined, so the same list
initialized for \lstinline{xs} is shared across different calls to
\lstinline{append}. When calling \lstinline{append(2)}, the default value for
\lstinline{xs} is no longer an empty list but the list that has been modified by
the previous call \lstinline{append(1)}.

This issue, while seemingly minor, highlights the importance of understanding
the semantics of default arguments. Our design strives to avoid such surprises,
following the principle of least astonishment, yet this is not our main focus.
We will discuss the more critical issue about type safety next.

\subsection{Caution! Type Safety with First-Class Named Arguments} \label{sec:mypy}

As we have shown in \autoref{fig:python-ruby}, quite a few languages, especially
dynamically typed ones like Python and Ruby, treat named arguments as
first-class values. This feature is particularly helpful for passing settings
because they are usually stored in a separate configuration file. We can read
the settings from the file and pass them as named arguments using the
\lstinline{**} operator. For example, we can find such code in Python to run a
web server:
\begin{lstlisting}[language={[3]Python}]
class App: # from a web server library
  def run(self, host: str, port: int, debug: bool = False):
    assert isinstance(debug, bool) # actual code omitted...

args = { "host": "0.0.0.0", "port": 80, "debug": True }
app.run(**args) #= app.run(host="0.0.0.0", port=80, debug=True)
\end{lstlisting}
Although Python is dynamically typed, there is continuous effort in the Python
community to improve the detection of type errors earlier in the development
process, primarily through static analysis. There is an optional static type
checker for Python called mypy~\citep{mypy}.\footnote{Our code is tested against
mypy 1.14.0 (released on 20 December 2024).} In the example above, we make use
of \emph{type hints}, introduced in Python 3.5, to specify the types of the
parameters and the return value of the \lstinline{run} method. The type hints
have no effect at run time but can be used by external tools like mypy to
statically check if the code is well-typed. Perhaps surprisingly, the code above
cannot pass mypy's type checking, because the type inferred for \lstinline{args}
(i.e. \lstinline{dict[str,object]}) is not precise enough. The type checker
needs to know what keys \lstinline{args} exactly has and what types the values
associated with those keys have, in order to make sure that \lstinline{**args}
is compatible with the parameters of \lstinline{app.run}.

Fortunately, \lstinline{TypedDict} is added in Python 3.8 to represent a
specific set of keys and their associated types. By default, every specified key
is required, except when it is marked as \lstinline{NotRequired}, which is a
type qualifier added later in Python 3.11. With \lstinline{TypedDict} and
\lstinline{NotRequired}, we can now define a precise dictionary type for
\lstinline{args} that passes mypy's type checking:
\begin{lstlisting}[language={[3]Python}]
class Args(TypedDict):
  host: str
  port: int
  debug: NotRequired[bool]

args0: Args = { "host": "0.0.0.0", "port": 80, "debug": True }
app.run(**args0) # type-checks in mypy

args1: Args = { "host": "0.0.0.0", "port": 80 }
app.run(**args1) # type-checks in mypy, too
\end{lstlisting}
The mypy type checker will raise an error if we provide an argument with an
incompatible type, such as a string for the \lstinline{debug} key:
\begin{lstlisting}[language={[3]Python}]
class In(TypedDict):
  host: str
  port: int
  debug: str

args2: In = { "host": "0.0.0.0", "port": 80, "debug": "Oops!" }
app.run(**args2) # TypeError: Argument "debug"
# has incompatible type "str"; expected "bool"  [arg-type]
\end{lstlisting}
However, mypy's type system is not completely type-safe. We can create a
function \lstinline{f} that takes a dictionary with three keys specified in type
\lstinline{In} and returns a dictionary with only two keys specified in type
\lstinline{Out}. The function type-checks in mypy because type \lstinline{In} is
compatible whenever type \lstinline{Out} is expected. Roughly speaking, it means
that \lstinline{In} is a subtype of \lstinline{Out}. Then we can use
\lstinline{f} to forget the \lstinline{debug} key in the static type:
\begin{lstlisting}[language={[3]Python}]
class Out(TypedDict):
  host: str
  port: int

def f(args: In) -> Out: return args

args3 = f(args2) # still contains { "debug": "Oops!" }
app.run(**args3) # type-checks in mypy, but has a runtime error!
\end{lstlisting}
Here \lstinline{args3} has type \lstinline{Out} without the \lstinline{debug}
key specified. From a static viewpoint, \lstinline{args3} only has two keys
\lstinline{host} and \lstinline{port}, which are compatible with the parameters
of \lstinline{app.run} since \lstinline{debug} is optional and has a default
value. That is why \lstinline{app.run(**args3)} type-checks in mypy. However, at
run time, the \lstinline{debug} key is still present in \lstinline{args3}, so
the string \lstinline{"Oops!"} is passed as a named argument to
\lstinline{app.run}, which originally expects a boolean value. This results in a
runtime error since there is an assertion in \lstinline{app.run} to ensure that
\lstinline{debug} is boolean.

This issue is not unique to Python and mypy. We have reproduced nearly the same
issue in Ruby with two popular type checkers, namely Steep~\citep{steep} and
Sorbet~\citep{sorbet}, which is illustrated in \autoref{sec:steep} and
\autoref{sec:sorbet}.\footnote{Our code is tested against Steep 1.9.2 (16
December 2024) and Sorbet 0.5.11708 (20 December 2024).}

In conclusion, subtyping can lead to a fundamental type-safety issue when
dealing with first-class named and optional arguments. In essence, the following
subsumption chain is questionable:
\begin{lstlisting}[language={[3]Python}]
   { host: str, port: int, debug: str }
<: { host: str, port: int }
<: { host: str, port: int, debug?: bool }
\end{lstlisting}
Following this chain bypasses mypy's type compatibility checking for the
\lstinline{debug} key. Next we will show how to break the chain and address the
type-safety issue.
