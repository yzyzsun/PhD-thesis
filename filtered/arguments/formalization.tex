\section{Formalization} \label{sec:iu-uaena}

In this section, we formalize the translation of named and optional arguments as
an elaboration semantics. The target of elaboration is called \lambdaiu, and the
source is called \uaena. We prove that the source language with named and
optional arguments is type-safe via (1) the type soundness of the target
calculus and (2) the type soundness of elaboration. All the theorems are
mechanically proven using the Coq proof assistant.

\subsection{The Target Calculus: \lambdaiu} \label{sec:lambdaiu}

\lambdaiu is an extension to the calculus in Chapter 5 of the dissertation by
\citet{rehman2023blend} with $\ottkw{null}$, single-field records, and let-in
bindings. The addition of let-in bindings is not essential because they can be
desugared into lambda abstractions and applications:
\begin{equation*}
  \ottkw{let} \, \ottmv{x}  \ottsym{=}  \ottnt{e_{{\mathrm{1}}}} \, \ottkw{in} \, \ottnt{e_{{\mathrm{2}}}} \quad \equiv \quad (\lambda x.\ e_2)\ e_1
\end{equation*}
However, we still have let-in bindings for the sake of readability, and this
form of let-in bindings simplifies the rules of parameter elaboration
(introduced later in \autoref{fig:elab}). Another difference is that the
original calculus uses the locally nameless
representation~\citep{chargueraud2012locally} while ours directly uses names for
bound variables.

Our changes to Rehman's calculus are relatively trivial, and we do not touch the
rules for intersection and union types. We will not discuss his design choices
in this thesis, because our focus is on the type soundness with the addition of
$\ottkw{Null}$ and record types. We have proven in Coq that these extensions
preserve type soundness.

\subsubsection{Syntax of \lambdaiu}
\begin{align*}
  &\text{Types}          &\ottnt{A},\ottnt{B} ::=&~  \top  ~|~  \bot  ~|~ \ottkw{Null} ~|~  \mathbb{Z}  ~|~ \ottnt{A}  \rightarrow  \ottnt{B} ~|~ \ottsym{\{}  \ell  \ottsym{:}  \ottnt{A}  \ottsym{\}} ~|~ \ottnt{A}  \land  \ottnt{B} ~|~ \ottnt{A}  \lor  \ottnt{B} \\
  &\text{Expressions}    &      \ottnt{e} ::=&~ \ottsym{\{\}} ~|~ \ottkw{null} ~|~  n  ~|~ \ottmv{x} ~|~  \lambda \ottmv{x} \!:\! \ottnt{A} .\; \ottnt{e} \!:\! \ottnt{B}  ~|~ \ottnt{e_{{\mathrm{1}}}} \, \ottnt{e_{{\mathrm{2}}}} ~|~ \ottsym{\{}  \ell  \ottsym{:}  \ottnt{A}  \ottsym{=}  \ottnt{e}  \ottsym{\}} ~|~ \ottnt{e}  \ottsym{.}  \ell \\
  &                      &              |&~ \ottnt{e_{{\mathrm{1}}}}  \bbcomma  \ottnt{e_{{\mathrm{2}}}} ~|~ \ottkw{switch} \, \ottnt{e_{{\mathrm{0}}}} \, \ottkw{as} \, \ottmv{x} \, \ottkw{case} \, \ottnt{A}  \Rightarrow  \ottnt{e_{{\mathrm{1}}}} \, \ottkw{case} \, \ottnt{B}  \Rightarrow  \ottnt{e_{{\mathrm{2}}}} ~|~ \ottnt{letin} \, \ottnt{e}
\end{align*}
The types include the top type $ \top $, the bottom type $ \bot $, the null
type $\ottkw{Null}$, the integer type $ \mathbb{Z} $, function types $\ottnt{A}  \rightarrow  \ottnt{B}$, record
types $\ottsym{\{}  \ell  \ottsym{:}  \ottnt{A}  \ottsym{\}}$, intersection types $\ottnt{A}  \land  \ottnt{B}$, and union types $\ottnt{A}  \lor  \ottnt{B}$.
$\ottkw{Null}$ is a unit type that has only one value $\ottkw{null}$.

The expressions include the empty record $\ottsym{\{\}}$, the null value $\ottkw{null}$,
integer literals $ n $, variables $\ottmv{x}$, lambda abstractions $ \lambda \ottmv{x} \!:\! \ottnt{A} .\; \ottnt{e} \!:\! \ottnt{B} $,
function applications $\ottnt{e_{{\mathrm{1}}}} \, \ottnt{e_{{\mathrm{2}}}}$, record literals $\ottsym{\{}  \ell  \ottsym{:}  \ottnt{A}  \ottsym{=}  \ottnt{e}  \ottsym{\}}$, record
projections $\ottnt{e}  \ottsym{.}  \ell$, merges $\ottnt{e_{{\mathrm{1}}}}  \bbcomma  \ottnt{e_{{\mathrm{2}}}}$, type-based switch expressions
$\ottkw{switch} \, \ottnt{e_{{\mathrm{0}}}} \, \ottkw{as} \, \ottmv{x} \, \ottkw{case} \, \ottnt{A}  \Rightarrow  \ottnt{e_{{\mathrm{1}}}} \, \ottkw{case} \, \ottnt{B}  \Rightarrow  \ottnt{e_{{\mathrm{2}}}}$, and let-in bindings $\ottnt{letin} \, \ottnt{e}$.
The syntax of $\ottnt{letin}$ is as follows:
\begin{equation*}
  \ottnt{letin} ::= \ottkw{let} \, \ottmv{x}  \ottsym{=}  \ottnt{e} \, \ottkw{in} ~|~  \ottnt{letin_{{\mathrm{1}}}} \circ \ottnt{letin_{{\mathrm{2}}}}  ~|~ \ottkw{id}
\end{equation*}
The composition of two let-in bindings is denoted by $ \ottnt{letin_{{\mathrm{1}}}} \circ \ottnt{letin_{{\mathrm{2}}}} $, and an
empty binding is denoted by $\ottkw{id}$.

\paragraph{Subtyping.}
\autoref{fig:sub} shows the subtyping rules of \lambdaiu. The rules are standard
for a type system with intersection and union types. \Rref{Sub-Top} shows that
the top type $ \top $ is a supertype of any type, and \rref{Sub-Bot} shows that
the bottom type $ \bot $ is a subtype of any type.
\Rref{Sub-And,Sub-AndL,Sub-AndR} handle the subtyping for intersection types,
while \rref{Sub-Or,Sub-OrL,Sub-OrR} are for union types. \Rref{Sub-Null,Sub-Rcd}
added by us are straightforward. We prove that the subtyping relation is
reflexive and transitive.

\begin{theorem}[Subtyping Reflexivity]
  $\forall A, \ottnt{A}  \ottsym{<:}  \ottnt{A}$.
\end{theorem}
\begin{theorem}[Subtyping Transitivity]
  If $\ottnt{A}  \ottsym{<:}  \ottnt{B}$ and $\ottnt{B}  \ottsym{<:}  \ottnt{C}$, then $\ottnt{A}  \ottsym{<:}  \ottnt{C}$.
\end{theorem}

\begin{figure}[b!]
\IUdefnsub{}
\caption{Subtyping of \lambdaiu.} \label{fig:sub}
\end{figure}

\begin{figure}
\begin{align*}
  &\text{Typing contexts}&\Gamma ::=&~  \cdot  ~|~ \Gamma  ,\,  \ottmv{x}  \ottsym{:}  \ottnt{A}
\end{align*}
\IUdefntyping{}
\IUdefnletbind{}
\caption{Typing of \lambdaiu.} \label{fig:typ}
\end{figure}

\paragraph{Typing.}
\autoref{fig:typ} shows the typing rules of \lambdaiu. The empty record $\ottsym{\{\}}$
has the top type $ \top $, as shown in \rref{Typ-Top}. \Rref{Typ-Merge} is the
introduction rule for intersection types. Merging two functions is used for
function overloading, and merging two records is used for record concatenation.
\Rref{Typ-Switch} is the elimination rule for union types. The type-based switch
expression scrutinizes an expression having a union of the two scrutinizing
types (i.e. $\ottnt{e_{{\mathrm{0}}}}:\ottnt{A}  \lor  \ottnt{B}$). This premise ensures the exhaustiveness of the
cases in the switch. The $\ottkw{as}$-variable $\ottmv{x}$ is refined to type
$\ottnt{A}$ in $\ottnt{e_{{\mathrm{1}}}}$ and to type $\ottnt{B}$ in $\ottnt{e_{{\mathrm{2}}}}$.
\Rref{Typ-Null,Typ-Rcd,Typ-Prj} added by us are straightforward. \Rref{Typ-Let}
uses an auxiliary judgment $\Gamma  \,\vdash\,  \ottnt{letin}  \,\dashv\,  \Gamma'$ to obtain the typing context
for the body of the let-in binding. For example, if $\ottnt{e_{{\mathrm{1}}}}$ has type $\ottnt{A}$,
then $\ottkw{let} \, \ottmv{x}  \ottsym{=}  \ottnt{e_{{\mathrm{1}}}} \, \ottkw{in} \, \ottnt{e_{{\mathrm{2}}}}$ adds $\ottmv{x}:\ottnt{A}$ to the typing context before
type-checking $\ottnt{e_{{\mathrm{2}}}}$.

\paragraph{Dynamic semantics.}
We have a small-step operational semantics for \lambdaiu. The judgment
$\ottnt{e}\longrightarrow\ottnt{e'}$ means that $\ottnt{e}$ reduces to $\ottnt{e'}$ in one step,
and $\ottnt{e}\longrightarrow^*\ottnt{e'}$ is for multi-step reduction. We extend the
original dynamic semantics by adding rules for records and projections.
Similarly to the applicative dispatch for function applications in the original
calculus, we add a relation called projective dispatch for record projections.
For example, $\ottsym{(}  \ottsym{\{}  \ottmv{x}  \ottsym{=}  \ottsym{1}  \ottsym{\}}  \bbcomma  \ottsym{\{}  \ottmv{y}  \ottsym{=}  \ottsym{2}  \ottsym{\}}  \ottsym{)}  \ottsym{.}  \ottmv{x}$ reduces to $\ottsym{\{}  \ottmv{x}  \ottsym{=}  \ottsym{1}  \ottsym{\}}  \ottsym{.}  \ottmv{x}$ via
projective dispatch to select the needed field.

Since the dynamic semantics of \lambdaiu is independent of the elaboration from
\uaena to \lambdaiu, we omit the rules here but leave them in
\autoref{sec:iu-dyn}. Note that the operational semantics is not commonplace in
that it is type-directed and non-deterministic. Please refer to the dissertation
by \citet{rehman2023blend} for detailed explanations.

\begin{theorem}[Progress]
  If $ \cdot   \,\vdash\,  \ottnt{e}  \ottsym{:}  \ottnt{A}$, then either $\ottnt{e}$ is a value or $\exists \ottnt{e'}, \ottnt{e} \longrightarrow \ottnt{e'}$.
\end{theorem}
\begin{theorem}[Preservation]
  If $ \cdot   \,\vdash\,  \ottnt{e}  \ottsym{:}  \ottnt{A}$ and $\ottnt{e} \longrightarrow \ottnt{e'}$, then $ \cdot   \,\vdash\,  \ottnt{e'}  \ottsym{:}  \ottnt{A}$.
\end{theorem}
Putting progress and preservation together, we conclude that \lambdaiu is type-sound:
a well-typed term can never reach a stuck state.
\begin{corollary}[Type Soundness]
  If $ \cdot   \,\vdash\,  \ottnt{e}  \ottsym{:}  \ottnt{A}$ and $\ottnt{e} \longrightarrow^* \ottnt{e'}$, then either $\ottnt{e'}$ is a value or $\exists \ottnt{e''}, \ottnt{e'} \longrightarrow \ottnt{e''}$.
\end{corollary}

\subsection{The Source Calculus: \uaena} \label{sec:uaena}

\uaena (\emph{Unnamed Arguments Extended with Named Arguments})
is a minimal calculus with named and optional arguments. Although the
calculus is small, named arguments are supported as first-class values and can
be passed to or returned by a function. Besides functions with named
arguments, \uaena also supports normal functions with positional
arguments. The two kinds of functions are distinguished in the syntax, as seen in
Ruby, Racket, OCaml, etc.

\subsubsection{Syntax of \uaena}
\begin{align*}
  &\text{Types}                 &\mathcal{A},\mathcal{B} ::=&~  \mathbb{Z}  ~|~  ( \mathcal{A} ) \rightarrow \mathcal{B}  ~|~ \ottsym{\{}  \mathcal{P}  \ottsym{\}}  \rightarrow  \mathcal{B} ~|~ \ottsym{\{}  \mathcal{K}  \ottsym{\}} \\
  &\text{Named parameter types} &        \mathcal{P} ::=&~  \cdot  ~|~ \mathcal{P}  ;\,  \ell  \ottsym{:}  \mathcal{A} ~|~ \mathcal{P}  ;\,  \ell  \ottsym{\mbox{?}}  \ottsym{:}  \mathcal{A} \\
  &\text{Named argument types}  &        \mathcal{K} ::=&~  \cdot  ~|~ \mathcal{K}  ;\,  \ell  \ottsym{:}  \mathcal{A} \\
  &\text{Expressions}           &       \epsilon ::=&~  n  ~|~ \ottmv{x} ~|~  \lambda ( \ottmv{x} \!:\! \mathcal{A} ).\; \epsilon  ~|~  \lambda \{ \rho \}.\; \epsilon  ~|~ \epsilon_{{\mathrm{1}}} \, \epsilon_{{\mathrm{2}}} ~|~ \ottsym{\{}  \kappa  \ottsym{\}} \\
  &\text{Named parameters}      &        \rho ::=&~  \cdot  ~|~ \rho  ;\,  \ell  \ottsym{:}  \mathcal{A} ~|~ \rho  ;\,  \ell  \ottsym{=}  \epsilon \\
  &\text{Named arguments}       &        \kappa ::=&~  \cdot  ~|~ \kappa  ;\,  \ell  \ottsym{=}  \epsilon
\end{align*}
The types include the integer type $ \mathbb{Z} $, normal function types $ ( \mathcal{A} ) \rightarrow \mathcal{B} $,
function types with named parameters $\ottsym{\{}  \mathcal{P}  \ottsym{\}}  \rightarrow  \mathcal{B}$, and (first-class)
named argument types $\ottsym{\{}  \mathcal{K}  \ottsym{\}}$. The expressions include integer literals
$ n $, variables $\ottmv{x}$, normal lambda abstractions $ \lambda ( \ottmv{x} \!:\! \mathcal{A} ).\; \epsilon $,
lambda abstractions with named parameters $ \lambda \{ \rho \}.\; \epsilon $, function applications
$\epsilon_{{\mathrm{1}}} \, \epsilon_{{\mathrm{2}}}$, and (first-class) named arguments $\ottsym{\{}  \kappa  \ottsym{\}}$.

A named parameter type $\mathcal{P}$ can be required ($\ell:\mathcal{A}$) or optional
($\ell?:\mathcal{A}$). If a named parameter is optional, its default value must be
provided in the function definition. For example,
$\lambda\{\ottmv{x}: \mathbb{Z} ;\;\ottmv{y}=0\}.\;\ottmv{x}  \ottsym{+}  \ottmv{y}$ has type
$\{\ottmv{x}: \mathbb{Z} ;\;\ottmv{y}?: \mathbb{Z} \} \rightarrow  \mathbb{Z} $. A function with named
parameters can only be applied to named arguments, which are basically a list of
key-value pairs. For example, the previous function can be applied to
$\{\ottmv{x}=1;\;\ottmv{y}=2\}$ or $\{\ottmv{x}=1\}$ or a variable having a compatible type.
The variable case demonstrates the first-class nature of named arguments in \uaena.

% manually break line for the conclusion
\renewcommand{\IUdrulePElaXXOptional}[1]{\ottdrule[#1]{%
\ottpremise{ \Delta \,\vdash\, _{\!\!\!\! \ottmv{x} }\:\: \rho \ottsym{:} \mathcal{P} \,\rightsquigarrow\, \ottnt{letin} \,\dashv\, \Delta' }%
\ottpremise{\Delta'  \,\vdash\,  \epsilon  \ottsym{:}  \mathcal{A}  \,\rightsquigarrow\,  \ottnt{e}}%
}{
\Delta\,\vdash_{\!\ottmv{x}}\,\ottsym{(}  \rho  ;\,  \ell  \ottsym{=}  \epsilon  \ottsym{)}:\ottsym{(}  \mathcal{P}  ;\,  \ell  \ottsym{\mbox{?}}  \ottsym{:}  \mathcal{A}  \ottsym{)}\\
 \,\rightsquigarrow\,  \ottnt{letin} \circ \ottkw{let} \, \ell  \ottsym{=}  \ottkw{switch} \, \ottmv{x}  \ottsym{.}  \ell \, \ottkw{as} \, \ottmv{y} \, \ottkw{case} \, |  \mathcal{A}  |  \Rightarrow  \ottmv{y} \, \ottkw{case} \, \ottkw{Null}  \Rightarrow  \ottnt{e} \, \ottkw{in}  \,\dashv\, \Delta'  ,\,  \ell  \ottsym{:}  \mathcal{A}}{%
{\ottdrulename{PEla\_Optional}}{}%
}}

\begin{figure}
\begin{align*}
  &\text{Typing contexts}&\Delta ::=&~  \cdot  ~|~ \Delta  ,\,  \ottmv{x}  \ottsym{:}  \mathcal{A}
\end{align*}
\IUdefnelab{}
\IUdefnpelab{}
\caption{Type-directed elaboration from \uaena to \lambdaiu.} \label{fig:elab}
\end{figure}

Careful readers may notice that a named argument type can also serve as the
parameter of a normal function. This also demonstrates the first-class nature of
named arguments. But note that a normal function that takes named
arguments is different from a function with named parameters. Consider the
following two functions, the former of which is a function with named parameters
and the latter is a normal function:
\begin{align*}
                    (\lambda\{\ottmv{x}: \mathbb{Z} ;\;\ottmv{y}=0\}.\;\ottmv{x}  \ottsym{+}  \ottmv{y}) \quad&:\quad \{\ottmv{x}: \mathbb{Z} ;\;\ottmv{y}?: \mathbb{Z} \} \rightarrow  \mathbb{Z}  \\
  (\lambda(\ottmv{args}:\{\ottmv{x}: \mathbb{Z} ;\;\ottmv{y}: \mathbb{Z} \}).\;\ottmv{args}) \quad&:\quad (\{\ottmv{x}: \mathbb{Z} ;\;\ottmv{y}: \mathbb{Z} \}) \rightarrow \{\ottmv{x}: \mathbb{Z} ;\;\ottmv{y}: \mathbb{Z} \}
\end{align*}
Although both functions can be applied to $\{\ottmv{x}=1;\;\ottmv{y}=2\}$, there are two
main differences between them. First, optional parameters cannot be defined in a
normal function. So we cannot provide $\ottmv{y}=0$ as a default value in the second
function. Second, $\ottmv{x}$ and $\ottmv{y}$ are not brought into the scope of the
function body in a normal function. So the only accessible variable is $\ottmv{args}$
in the second function.

\paragraph{Elaboration.}
The type-directed elaboration from \uaena to \lambdaiu is defined at the top of
\autoref{fig:elab}. $\Delta  \,\vdash\,  \epsilon  \ottsym{:}  \mathcal{A}  \,\rightsquigarrow\,  \ottnt{e}$ means that the source
expression $\epsilon$ has type $\mathcal{A}$ and elaborates to the target expression
$\ottnt{e}$ under the typing context $\Delta$. \Rref{Ela-Abs,Ela-App} for normal
functions are straightforward. In \rref{Ela-NAbs} for functions with named
parameters, besides inferring the type of the function body $\epsilon$ and
elaborating it to $\ottnt{e}$, we generate let-bindings for the named parameters,
which is delegated to the auxiliary judgment $ \Delta \,\vdash\, _{\!\!\!\! \ottmv{x} }\:\: \rho \ottsym{:} \mathcal{P} \,\rightsquigarrow\, \ottnt{letin} \,\dashv\, \Delta' $.
In \rref{Ela-NApp}, there is also an auxiliary judgment $ \Delta \,\vdash\, _{\!\!\!\! \ottnt{e} }\:\: \mathcal{P} \,\diamond\, \mathcal{K} \,\rightsquigarrow\, \ottnt{e'} $
that rewrites call sites according to the parameter and argument types.
\Rref{Ela-NEmpty,Ela-NField} are used to elaborate named arguments.

\paragraph{Named parameter elaboration.}
As shown at the bottom of \autoref{fig:elab},
$ \Delta \,\vdash\, _{\!\!\!\! \ottmv{x} }\:\: \rho \ottsym{:} \mathcal{P} \,\rightsquigarrow\, \ottnt{letin} \,\dashv\, \Delta' $ means that the named parameter
$\rho$ is inferred to have type $\mathcal{P}$ and elaborates to a series of let-in
bindings $\ottnt{letin}$, given that the named parameters correspond to the target
bound variable $\ottmv{x}$. In the meanwhile, the typing context $\Delta$ is extended
with the types of the named parameters to form $\Delta'$. $\Delta'$ is used for
typing the body of the function with named parameters. \Rref{PEla-Required}
simply generates $\ottkw{let} \, \ell  \ottsym{=}  \ottmv{x}  \ottsym{.}  \ell \, \ottkw{in}$, while \rref{PEla-Optional} generates
$\ottkw{let} \, \ell  \ottsym{=}  \ottkw{switch} \, \ottmv{x}  \ottsym{.}  \ell \, \ottkw{as} \, \ottmv{y} \, \ottkw{case} \, |  \mathcal{A}  |  \Rightarrow  \ottmv{y} \, \ottkw{case} \, \ottkw{Null}  \Rightarrow  \ottnt{e} \, \ottkw{in}$ to provide a
default value $\ottnt{e}$ for the $\ottkw{Null}$ case.

\begin{figure}
\IUdefnpmatch{}
\IUdefnlookup{}
\IUdefnlookdown{}
\caption{Type-directed call site rewriting in \uaena.} \label{fig:match}
\end{figure}

\paragraph{Call site rewriting.}
As shown in \autoref{fig:match},
$ \Delta \,\vdash\, _{\!\!\!\! \ottnt{e} }\:\: \mathcal{P} \,\diamond\, \mathcal{K} \,\rightsquigarrow\, \ottnt{e'} $ means that if the parameter type
$\mathcal{P}$ is compatible with the argument type $\mathcal{K}$, the target expression
$\ottnt{e}$, which corresponds to the named arguments, will be rewritten to
$\ottnt{e'}$. The compatibility check is based on the parameter type $\mathcal{P}$.
\Rref{PMat-Required} handles the case where the argument is required, while
\rref{PMat-Present,PMat-Absent} handle the cases where the optional argument
with a specific type is present and where the optional argument is absent,
respectively. The remaining case, where the optional argument is present but
associated with a wrong type, is prohibited and cannot elaborate to any term.
We have two more auxiliary judgments $ \mathcal{K} :: \ell \Rightarrow \mathcal{A} $ and
$ \mathcal{K} :: \ell \nRightarrow $ to indicate that the argument type $\mathcal{K}$ contains a field $\ell$
with type $\mathcal{A}$ or $\mathcal{K}$ does not contain $\ell$.

\paragraph{Type translation.}
As we have informally mentioned in \autoref{sec:core}, we translate named
parameters to intersection types and optional parameters to union types. The
rules for $|\cdot|$ can be found in \autoref{fig:trans-iu}. Having
defined the translation, we can prove the soundness of call site rewriting and
elaboration.

\begin{figure}
\framebox{$|  \mathcal{A}  |$} \quad Type translation
\begin{mathpar}
|   \mathbb{Z}   | \equiv  \mathbb{Z} 

|   ( \mathcal{A} ) \rightarrow \mathcal{B}   | \equiv |  \mathcal{A}  |  \rightarrow  |  \mathcal{B}  |

|  \ottsym{\{}  \mathcal{P}  \ottsym{\}}  \rightarrow  \mathcal{B}  | \equiv |  \mathcal{P}  |  \rightarrow  |  \mathcal{B}  |

|  \ottsym{\{}  \mathcal{K}  \ottsym{\}}  | \equiv  | \mathcal{K} | 
\end{mathpar}

\noindent
\framebox{$|  \mathcal{P}  |$} \quad Parameter type translation
\begin{mathpar}
|   \cdot   | \equiv  \top 

|  \mathcal{P}  ;\,  \ell  \ottsym{:}  \mathcal{A}  | \equiv |  \mathcal{P}  |  \land  \ottsym{\{}  \ell  \ottsym{:}  |  \mathcal{A}  |  \ottsym{\}}

|  \mathcal{P}  ;\,  \ell  \ottsym{\mbox{?}}  \ottsym{:}  \mathcal{A}  | \equiv |  \mathcal{P}  |  \land  \ottsym{\{}  \ell  \ottsym{:}  |  \mathcal{A}  |  \lor  \ottkw{Null}  \ottsym{\}}
\end{mathpar}

\noindent
\framebox{$ | \mathcal{K} | $} \quad Argument type translation
\begin{mathpar}
 |  \cdot  |  \equiv  \top 

 | \mathcal{K}  ;\,  \ell  \ottsym{:}  \mathcal{A} |  \equiv  | \mathcal{K} |   \land  \ottsym{\{}  \ell  \ottsym{:}  |  \mathcal{A}  |  \ottsym{\}}
\end{mathpar}

\noindent
\framebox{$|  \Delta  |$} \quad Typing context translation
\begin{mathpar}
|   \cdot   | \equiv  \cdot 

|  \Delta  ,\,  \ottmv{x}  \ottsym{:}  \mathcal{A}  | \equiv |  \Delta  |  ,\,  \ottmv{x}  \ottsym{:}  |  \mathcal{A}  |
\end{mathpar}
\caption{Type translation from \uaena to \lambdaiu.} \label{fig:trans-iu}
\end{figure}


\begin{theorem}[Soundness of Call Site Rewriting]
  If $ \Delta \,\vdash\, _{\!\!\!\! \ottnt{e} }\:\: \mathcal{P} \,\diamond\, \mathcal{K} \,\rightsquigarrow\, \ottnt{e'} $ and $|  \Delta  |  \,\vdash\,  \ottnt{e}  \ottsym{:}   | \mathcal{K} | $,
  then $|  \Delta  |  \,\vdash\,  \ottnt{e'}  \ottsym{:}  |  \mathcal{P}  |$.
\end{theorem}

\begin{theorem}[Soundness of Elaboration]
  If $\Delta  \,\vdash\,  \epsilon  \ottsym{:}  \mathcal{A}  \,\rightsquigarrow\,  \ottnt{e}$, then $|  \Delta  |  \,\vdash\,  \ottnt{e}  \ottsym{:}  |  \mathcal{A}  |$.
\end{theorem}

\noindent
With the two theorems above and the type soundness of \lambdaiu, we can
conclude that \uaena is type-safe.
