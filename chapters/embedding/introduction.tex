\section{Introduction}

A common approach to defining DSLs is via a direct embedding into a host
language. This approach is used in several programming languages, such as
Haskell, Scala, and Racket. In those languages, various DSLs -- including pretty
printers~\citep{hughes1995design,wadler2003prettier}, parser
combinators~\citep{leijen2001parsec}, and property-based testing
frameworks~\citep{claessen2000quickcheck} -- are defined as embedded DSLs. There
are a few techniques for such embeddings, including the well-known
\emph{shallow} and \emph{deep} embeddings~\citep{boulton1992experience}.

Unfortunately, shallow and deep embeddings come with various trade-offs in
existing programming languages. Such trade-offs have been widely discussed in
the literature~\citep{rompf2012scala,scherr2014implicit,gibbons2014folding}. On
the one hand, the strengths of shallow embeddings are in providing
\emph{linguistic reuse}~\citep{krishnamurthi2001linguistic}, exploiting
meta-language optimizations, and allowing the addition of new DSL constructs
easily. On the other hand, deep embeddings shine in enabling the definition of
complex semantic interpretations and optimizations over the abstract syntax tree
(AST) of the DSL, and they enable adding new semantic interpretations easily.
Regarding such trade-offs, \citet{svenningsson2015combining} made the following
striking comment:

\begin{quoting}
\noindent ``The holy grail of embedded language implementation is to be able to
combine the advantages of shallow and deep in a single implementation.''
\end{quoting}

While progress has been made in embedded language implementation, the holy grail
is still not fully achieved in existing programming languages. Owing to the
trade-offs between shallow and deep embeddings, many realistic embedded DSLs end
up using a mix of both approaches in practice or more advanced forms of
embeddings. For instance, there have been several
approaches~\citep{rompf2012scala,svenningsson2015combining,jovanovic2014yinyang}
promoting the use of shallow embeddings as the frontend of the DSL to enable
linguistic reuse, while deep embeddings are used as the backend for added
flexibility in defining semantic interpretations. While such approaches manage
to alleviate some of the trade-offs, they require translations between the two
embeddings, a substantial amount of code, and some advanced coding techniques.
Alternatively, there are more advanced embedding techniques, which are inspired
by work on extensible Church encodings of algebraic data
types~\citep{hinze2006generics,oliveira2006extensible,oliveira2009modular}. Such
techniques include \emph{tagless-final
embeddings}~\citep{carette2009finally,kiselyov2010typed}, \emph{polymorphic
embeddings}~\citep{hofer2008polymorphic}, and \emph{object
algebras}~\citep{oliveira2012extensibility}, and they are able to eliminate some
of the trade-offs too. In particular, those approaches eliminate the trade-offs
with respect to extensibility, facilitating both the addition of new DSL
constructs and semantic interpretations. However, being quite close to shallow
embeddings, those approaches lack some important capabilities, such as the
ability to define complex interpretations and the use of (nested) pattern
matching to express semantic interpretations and transformations easily and
modularly.

In this chapter, we show that the \emph{compositional programming}
paradigm~\citep{zhang2021compositional} and the CP language provide improved
programming language support for embedded DSLs. Compositional programming
provides an alternative way to define data structures compared to algebraic data
types in functional programming or class hierarchies in object-oriented
programming. Compositional interfaces and traits play a role similar to
algebraic data types and definitions by pattern matching in functional programs.
However, unlike algebraic data types and definitions by pattern matching,
compositional interfaces and traits are modularly extensible. \emph{Intersection
types}~\citep{oliveira2016disjoint} enable the composition of interfaces, while
\emph{nested composition}~\citep{bi2018essence} enables the composition of
traits with pattern matching definitions.  Thus, the CP language does not suffer
from the infamous \emph{expression problem}~\citep{wadler1998expression}. In
addition, CP comes with several mechanisms to express \emph{modular
dependencies}, allowing powerful forms of dependency injection and complex
semantic interpretations.

With those programming language features, we obtain a new form of embedding
called a \emph{compositional embedding}, with nearly all of the advantages of
both shallow and deep embeddings. On the one hand, compositional embeddings
enable various forms of linguistic reuse that are characteristic of shallow
embeddings, including the ability to reuse host-language optimizations in the
DSL and easily add new DSL constructs. On the other hand, similarly to deep
embeddings, compositional embeddings support definitions by pattern matching or
dynamic dispatching, including optimizations and transformations over the
abstract syntax of the DSL, as well as the ability to add new interpretations.
In short, we believe that compositional embeddings come very close to the holy
grail of embedded language implementation desired by
\citet{svenningsson2015combining}.

We reimplemented the CP language and made it available as an in-browser
interpreter. Using this new implementation of CP, we illustrate an instance of
compositional embeddings with a DSL for document authoring called \ExT
(\textbf{Ex}tensible \textbf{T}ypesetting). The DSL is highly flexible and
extensible, allowing users to create various non-trivial extensions. For
instance, \ExT supports extensions that enable the production of wiki-like
documents, \LaTeX{} documents, SVG charts, or computational graphics like
fractals. Our largest application is Minipedia: a mini version of Wikipedia.
Minipedia includes wiki pages for several states and cities and some tables that
list countries by population or area. The \ExT DSL, as well as its various
extensions and applications, is available online:
\begin{center}
  \url{https://plground.org}
\end{center}
