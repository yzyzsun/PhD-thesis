\section{Embeddings of DSLs} \label{sec:embed}

In this section, we give some background on existing approaches to embedded DSLs
and evaluate their strengths and drawbacks. We focus on \emph{shallow} and \emph{deep}
embeddings~\citep{boulton1992experience}, which are the two main alternative
forms of embeddings. We also discuss some other forms of embeddings near the end
of this section. To illustrate all these embeddings, we present programs in
Haskell, which is well known for its good support for embedded DSLs and has a
syntax close to the CP language.

\subsection{A Simple Region DSL}

Inspired by \citet{hudak1998modular} and \citet{hofer2008polymorphic}, we
consider a simple region DSL for plane geometry. To illustrate the challenges in
developing such a DSL, we consider five separate steps in the development, which
illustrate various desired features, such as linguistic reuse and the ease of
adding features for software evolution.

\begin{enumerate}
\item \textbf{Initial system.}
      We start with a small region language with five constructs:
      \lstinline{circle} for creating a circular region with a given radius,
      \lstinline{outside} for a complement to a certain region,
      two set operators \lstinline{union} and \lstinline{intersect}, and finally
      \lstinline{translate} for moving a region by a given vector.
      The initial interpretation is to simply calculate the syntactic size of a region.
\item \textbf{Linguistic reuse.}
      We create a region with structure sharing to assess the difficulty of
      reusing host-language optimizations.
\item \textbf{Extensibility with new constructs.}
      We extend the region language with three more constructs:
      \lstinline{univ} for the universal region that contains all points,
      \lstinline{empty} for the empty region that contains no points, and
      \lstinline{scale} for resizing a region by two scale factors in a vector.
\item \textbf{Extensibility with new operations.}
      We add a new interpretation that checks whether a given point resides in a region.
\item \textbf{Complex interpretations, transformations, and optimizations.}
      Last but not least, we discuss three kinds of more complex interpretations
      that illustrate dependent interpretations (i.e.~interpretations that need
      to call other interpretations in their implementations) and
      transformations over the AST of the region language. 
\end{enumerate}

\subsection{Initial System}

\begin{figure}
\begin{lstlisting}[language=Haskell,basicstyle=\ttfamily\footnotesize]
data Vector = Vector { x :: Double, y :: Double } deriving Show
\end{lstlisting}
\begin{subfigure}[b]{.5\textwidth}
\begin{lstlisting}[language=Haskell,deletekeywords={union,intersect},basicstyle=\ttfamily\footnotesize]
type Region = Int

circle :: Double -> Region
circle _  =  1

outside :: Region -> Region
outside a  =  a + 1

union :: Region -> Region -> Region
union a b  =  a + b + 1

intersect :: Region -> Region -> Region
intersect a b  =  a + b + 1

translate :: Vector -> Region -> Region
translate _ a  =  a + 1
\end{lstlisting}
\caption{A shallow embedding.}
\end{subfigure}%
\begin{subfigure}[b]{.5\textwidth}
\begin{lstlisting}[language=Haskell,basicstyle=\ttfamily\footnotesize]
data Region where
  Circle    :: Double -> Region
  Outside   :: Region -> Region
  Union     :: Region -> Region -> Region
  Intersect :: Region -> Region -> Region
  Translate :: Vector -> Region -> Region


size :: Region -> Int
size (Circle      _) = 1
size (Outside     a) = size a + 1
size (Union     a b) = size a + size b + 1
size (Intersect a b) = size a + size b + 1
size (Translate _ a) = size a + 1
\end{lstlisting}
\caption{A deep embedding.} \label{fig:pattern}
\end{subfigure}
\caption{The initial system for the region DSL.} \label{fig:initial}
\end{figure}

\noindent
\autoref{fig:initial} shows the definitions of the initial DSL, including the
language interface and a simple interpretation for shallow and deep embeddings.

\paragraph{Language interfaces.}
In the shallow embedding, \lstinline{Region} denotes the semantic domain. All
the five language constructs are implemented as functions (called denotation
functions), which implement the respective semantics of the language constructs.
Thus, the language interface is essentially the signature of the denotation
functions and is entangled with the definition of a particular semantic domain.
In the deep embedding, \lstinline{Region} is defined as an algebraic data
type.\footnote{We use the notation of \emph{generalized} algebraic data types to
make type annotations clear.} Algebraic data types only capture the abstract
syntax, thus enabling the language interface to be separated from any concrete
semantics.

\paragraph{Semantic interpretations.}
There can be many semantic interpretations, which associate language constructs
with their meanings. The initial semantics that we choose here is an abstract
interpretation, which calculates the syntactic size of a region (i.e.~the number
of AST nodes). We opt to have an interpretation that is slightly artificial, but
simple, so that we can focus on the key aspects of compositional embeddings. In
the shallow embedding, the semantics is defined together with the language
interface since we must specify some concrete type (\lstinline{Int} in this
case) for \lstinline{Region}. In contrast, to create a new semantic
interpretation in the deep embedding, we define a function \lstinline{size}
using pattern matching over the \lstinline{Region} data type. 

We can see that the two embeddings work quite differently: the shallow embedding
encodes semantics in region constructors, and the interpretation function from
\lstinline{Region} to \lstinline{Int} is merely the identity function; the deep
embedding does nothing concerning constructors but hands over the work to the
interpretation function \lstinline{size}.

\subsection{Linguistic Reuse and Meta-Language Optimizations} \label{sec:meta}

An important concern for embedded DSLs is \emph{linguistic
reuse}~\citep{krishnamurthi2001linguistic}. One of the key selling points of
embedded DSLs is that they can reuse much of the infrastructure of the host
language and inherit various optimizations that are available in the host
language. However, there are important differences between shallow and deep
embeddings with respect to linguistic reuse. As widely observed in previous
work~\citep{rompf2012scala,jovanovic2014yinyang,svenningsson2015combining},
shallow embeddings make linguistic reuse easier. To illustrate this, we can
create a region that contains a series of repeated subregions with horizontally
aligned circles:

\begin{lstlisting}[language=Haskell,deletekeywords={union,intersect}]
circles = go 20 (2 ** 18)
  where go :: Int -> Double -> Region
        go 0 offset = circle 1
        go n offset = let shared = go (n - 1) (offset / 2)
                      in union (translate Vector { x = -offset, y = 0 } shared)
                               (translate Vector { x =  offset, y = 0 } shared)
\end{lstlisting}

\noindent
The region \lstinline{circles} is defined via an auxiliary recursive function
\lstinline{go}. In the shallow embedding, \lstinline{shared} is of type
\lstinline{Int}, and using \lstinline{shared} twice in the \lstinline{let}-body
will avoid interpreting the region twice. The code above also works for the deep
embedding, assuming some smart constructors such as \lstinline{circle = Circle}.
However, in the deep embedding, \lstinline{shared} is an AST of type
\lstinline{Region}. In this case, the AST will be duplicated in the
\lstinline{let}-body, and later when we interpret the region, we need to
traverse the same AST twice, duplicating the size calculation. Since
\lstinline{go} is recursive, it will lead to a lot of sharing for the shallow
approach and a lot of repetition for the deep approach. As a result, the
evaluation of \lstinline{circles} is instantaneous in the shallow approach,
whereas in the deep approach, it basically does not terminate in a reasonable
amount of time. In short, in shallow embeddings, we are able to preserve sharing
by naturally exploiting the sharing optimizations present in the host language,
but sharing is lost in deep embeddings.

We should remark that this particular issue regarding sharing (which is just an
instance of linguistic reuse) is well
known~\citep{gill2009type,kiselyov2011implementing,oliveira2013abstract}. There
have been several proposed techniques that enable preserving sharing in deeply
embedded DSLs. However, all those techniques add extra complexity to the DSL
encoding, and they may even make the DSL harder to use. This is in contrast to
the shallow approach, where no extra work is required by the programmer and
host-language features are naturally reused.

\subsection{Adding More Language Constructs}

As part of evolving the DSL with additional features, we may decide to add more
constructs for regions. To evaluate how easy it is to add more language constructs,
we add three more
language constructs for the universal or empty region as well as a scaling
operator. Shallow embeddings make such extensions very easy. We only need
to add three new denotation functions:

\noindent
\begin{minipage}{0.25\textwidth}
\begin{lstlisting}[language=Haskell]
univ :: Region
univ = 1
\end{lstlisting}
\end{minipage}%
\begin{minipage}{0.25\textwidth}
\begin{lstlisting}[language=Haskell]
empty :: Region
empty = 1
\end{lstlisting}
\end{minipage}%
\begin{minipage}{0.5\textwidth}
\begin{lstlisting}[language=Haskell]
scale :: Vector -> Region -> Region
scale _ a  =  a + 1
\end{lstlisting}
\end{minipage}

\noindent
Notwithstanding the modular extension in shallow embeddings, it is awkward to
add language constructs in deep embeddings. We have to modify the algebraic data
type \lstinline{Region} and all related interpretation functions to add new
cases. This is not modular because we must change the existing code. Even if we
have access to previous definitions, it is inevitable to recompile all the code
that depends on \lstinline{Region}.

In essence, once we start adding new features, we are quickly faced with an
instance of the \emph{expression problem}~\citep{wadler1998expression}. Shallow
embeddings make adding constructs easy, whereas adding new constructs in deep
embeddings is more difficult and non-modular. As we shall see next, when adding
new semantic interpretations, we have a dual problem.

\subsection{Adding a New Interpretation} \label{sec:contains}

The syntactic size is too abstract to describe a region, so let us add a new
interpretation that checks whether a region contains a given point. For example,
a circular region of radius $r$ contains $(x,y)$ if and only if $x^2 + y^2 \le
r^2$. Although adding constructs is awkward in deep embeddings, adding a new
interpretation function is trivial:

\begin{lstlisting}[language=Haskell]
contains :: Region -> Vector -> Bool
Circle      r `contains` p  =  p.x ** 2 + p.y ** 2 <= r ** 2
Outside     a `contains` p  =  not (a `contains` p)
Union     a b `contains` p  =  a `contains` p || b `contains` p
Intersect a b `contains` p  =  a `contains` p && b `contains` p
Translate Vector {..} a `contains` p  =
                      a `contains` Vector { x = p.x - x, y = p.y - y }
\end{lstlisting}

\noindent
The code is mostly straightforward.\footnote{Two GHC language extensions are
required here: \emph{OverloadedRecordDot} and \emph{RecordWildCards}.} The only
minor remark is that we use a record wildcard (\lstinline|Vector {..}|) in the
case of \lstinline{Translate} to bring record fields (\lstinline{x} and
\lstinline{y}) into scope.

However, there is no easy modular way to support multiple interpretations in
shallow embeddings. We need to completely change the definition of
\lstinline{Region} and remap all the language constructs to a new semantic
domain. The issue of semantic extension in shallow embeddings is dual to that of
language extension in deep embeddings. To address the tension between the two
dimensions, some alternative embeddings are proposed, such as tagless-final
embeddings~\citep{carette2009finally,kiselyov2010typed} and polymorphic
embeddings~\citep{hofer2008polymorphic}, though they still have some significant
drawbacks, which will be revealed shortly.

\subsection{Dependencies, Transformations, and Complex Interpretations} \label{sec:transform}

One area where deeply embedded DSLs shine is in enabling more complex kinds of
interpretations. These interpretations may, for example, enable transformations
or rewritings on the AST, which are helpful for writing domain-specific
optimizations, among other things. For writing such complex forms of
interpretations, we often require multiple \emph{dependent} functions defined by
pattern matching, and sometimes we may even need nested pattern matching. Such
interpretations are very challenging for shallow DSLs and are often the reason
why DSL writers prefer deep embeddings.

\paragraph{Dependent interpretations.}
Let us start with a dependent interpretation that shows a text representation of
a region using a deep embedding:

\begin{lstlisting}[language=Haskell]
text :: Region -> String
text   (Circle      r)  =  "circular region of radius " ++ show r
text   (Outside     a)  =  "outside a region of size " ++ show (size a)
text s@(Union     _ _)  =
    "union of two regions of size " ++ show (size s) ++ " in total"
text s@(Intersect _ _)  =
    "intersection of two regions of size " ++ show (size s) ++ " in total"
text s@(Translate _ _)  =  "translated region of size " ++ show (size s)
\end{lstlisting}

\noindent
Note that the definition of \lstinline{text} \emph{depends} on the previously
defined \lstinline{size} function. Such a definition is challenging in shallow
embeddings. A workaround sometimes used in shallow embeddings is to use tuples
to define multiple interpretations simultaneously. For example, we can define
\lstinline{size} and \lstinline{text} together as:

\begin{lstlisting}[language=Haskell,deletekeywords={union,intersect}]
type Region = (Int, String) -- (size, text)
circle      r  =  (1, "circular region of radius " ++ show r)
outside     a  =  (fst a + 1, "outside a region of size " ++ show (fst a))
union     a b  =
    (size, "union of two regions of size " ++ show size ++ " in total")
  where size = fst a + fst b + 1
intersect a b  =
    (size, "intersection of two regions of size " ++ show size ++ " in total")
  where size = fst a + fst b + 1
translate _ a  =  (size, "translated region of size " ++ show size)
  where size = fst a + 1
\end{lstlisting}

\paragraph{Mutually dependent interpretations.} \label{sec:mutual}
A similar but more interesting example is two interpretations for checking
universality and emptiness:

\begin{lstlisting}[language=Haskell]
isUniv :: Region -> Bool
isUniv            Univ  =  True
isUniv (Outside     a)  =  isEmpty a
isUniv (Union     a b)  =  isUniv a || isUniv b
isUniv (Intersect a b)  =  isUniv a && isUniv b
isUniv (Translate _ a)  =  isUniv a
isUniv (Scale     _ a)  =  isUniv a
isUniv               _  =  False

isEmpty :: Region -> Bool
isEmpty           Empty  =  True
isEmpty (Outside     a)  =  isUniv a
isEmpty (Union     a b)  =  isEmpty a && isEmpty b
isEmpty (Intersect a b)  =  isEmpty a || isEmpty b
isEmpty (Translate _ a)  =  isEmpty a
isEmpty (Scale     _ a)  =  isEmpty a
isEmpty               _  =  False
\end{lstlisting}

\noindent
Unlike in the previous example, where only \lstinline{text} depends on
\lstinline{size}, the two definitions are \emph{mutually recursive}, depending
on each other. If we want to rewrite them using shallow embeddings, they also
have to be encoded as a pair to call each other via
\lstinline[language=Haskell]{fst} and \lstinline[language=Haskell]{snd}:

\begin{lstlisting}[language=Haskell,deletekeywords={union,intersect}]
type Region = (Bool, Bool) -- (isUniv, isEmpty)

univ           =  (True,           False)
empty          =  (False,          True)
circle      _  =  (False,          False)
outside     a  =  (snd a,          fst a)
union     a b  =  (fst a || fst b, snd a && snd b)
intersect a b  =  (fst a && fst b, snd a || snd b)
translate _ a  =  (fst a,          snd a)
scale     _ a  =  (fst a,          snd a)
\end{lstlisting}

\noindent 
Generally speaking, using tuples to deal with dependencies is non-modular and
awkward to use in that interpretations become tightly coupled. Adding more
interpretations with dependencies, for example, would require a complete rewrite
of the code: dependent interpretations have to be defined together with the
interpretations that they depend on. More advanced encodings, such as
tagless-final embeddings~\citep{kiselyov2010typed} or polymorphic
embeddings~\citep{hofer2008polymorphic}, provide ways to modularize
\emph{independent} interpretations into reusable units, but they cannot scale
well to \emph{dependent} interpretations and often have to resort to similar
techniques using tuples like the code above. In \autoref{sec:tagless}, we encode
the two examples above in tagless-final embeddings and illustrate that they
suffer from the same problems as shallow embeddings with respect to
dependencies, both mutual and non-mutual.

\paragraph{Nested pattern matching.} \label{sec:nested}
Another strength of deep embeddings is their ability to perform nested pattern
matching, which is rather useful to inspect smaller constructs. Here we take
an optimization that eliminates double negation for example. Nested pattern
matching is used to check if an \lstinline{Outside} is directly wrapped around
an outer \lstinline{Outside}. If so, both of them are eliminated; otherwise,
\lstinline{eliminate} is recursively called with inner constructs:

\begin{lstlisting}[language=Haskell]
eliminate :: Region -> Region
eliminate (Outside (Outside a))  =  eliminate a
eliminate (Outside           a)  =  Outside (eliminate a)
eliminate (Union           a b)  =  Union (eliminate a) (eliminate b)
eliminate (Intersect       a b)  =  Intersect (eliminate a) (eliminate b)
eliminate (Translate       v a)  =  Translate v (eliminate a)
eliminate (Scale           v a)  =  Scale v (eliminate a)
eliminate                     a  =  a
\end{lstlisting}

\noindent
The common use of nested pattern matching poses a second challenge to
domain-specific optimizations. \citet{kiselyov2010typed} discusses the
``\emph{seemingly impossible pattern-matching}'' problem with nested patterns,
which appear not directly encodable in tagless-final embeddings. Instead, he
proposes that such algorithms can often be implemented in a tagless-final
interpreter as context-dependent interpretations by essentially changing the
algorithm. An extra parameter is added so the semantic type looks like
\lstinline{Ctx -> repr}. However, in the general case, it is not obvious how to
convert an algorithm with nested pattern matching into one based on contexts. 
