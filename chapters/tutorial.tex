\chapter{A Crash Course in the CP Language} \label{ch:cp}

CP, short for \emph{compositional programming}, is a statically typed
programming language that emphasizes modularity and extensibility. It
incorporates a number of novel features in line with recent research on
compositional programming at the University of Hong Kong. These features
include:
\begin{itemize}
\item Disjoint intersection types~\citep{oliveira2016disjoint} and disjoint
      polymorphism~\citep{alpuim2017disjoint,xie2020row};
\item Distributive intersection subtyping~\citep{bi2018essence,bi2019distributive}
      and its interaction with mutable references~\citep{ye2024imperative};
\item Compositional interfaces, method patterns, and modular
      dependencies~\citep{zhang2021compositional};
\item Iso-recursive types with nominal unfolding~\citep{zhou2022calculus};
\item Generalized record operations with type difference~\citep{xu2023making};
\item Compositional embeddings of domain-specific languages
      (\autoref{ch:embedding});
\item Dynamic inheritance (\autoref{ch:inheritance}) of first-class
      traits~\citep{bi2018typed};
\item Named and optional arguments (\autoref{ch:arguments}) via a blend of
      intersection and union types~\citep{rehman2023blend}.
\end{itemize}
Before diving into the novel features, we first introduce the basics of CP.

\section{Overview}

A CP program consists of a sequence of definitions, either for types or for
terms, and ends with a main expression to be evaluated. For example, the
following program defines a type alias \lstinline{I} and a term variable
\lstinline{i} of type \lstinline{I}, and finally evaluates \lstinline{i + 1}:
\begin{lstlisting}[xleftmargin=.3\textwidth]
type I = Int;  -- type definition
i: I = 1;      -- term definition
i + 1          -- main expression
\end{lstlisting}
Every definition must end with a semicolon.
Anything after \lstinline{--} (or within \lstinline|{- block -}|) is a comment.
Note that a type must start with an uppercase letter, while a term variable must
start with a lowercase letter. The type annotation (\lstinline{: I}) is not
required for term definitions as long as the type can be inferred.

\begin{table}[b!]
\caption{List of term operators by precedence in CP.} \label{tab:op}
\centering
\begin{tabular}{*{4}{c}}
\toprule
\thead{Operators}        & \thead{Operands}                      & \thead{Arity} & \thead{Description} \\
\midrule \midrule
\texttt{-}               & \lstinline{Int} or \lstinline{Double} & Unary         & Neg. \\
\midrule
\texttt{\textsurd}       & \lstinline{Double}                    & Unary         & Sqrt. \\
\midrule
\texttt{\#}              & Arrays                                & Unary         & Length \\
\midrule
\texttt{!}               & References                            & Unary         & Deref. \\
\midrule
\texttt{!!}              & Arrays (lhs), \lstinline{Int} (rhs)   & Binary        & Index \\
\midrule
\texttt{* / \%}          & \lstinline{Int} or \lstinline{Double} & Binary        & Mul. \& Div. \& Mod. \\
\midrule
\texttt{+ -}             & \lstinline{Int} or \lstinline{Double} & Binary        & Add. \& Sub. \\
\midrule
\texttt{++}              & \lstinline{String} or Arrays          & Binary        & Concat. \\
\midrule
\makecell{\texttt{== !=} \\ \texttt{< <= > >=}} & \makecell{\lstinline{Int} or \lstinline{Double} or \lstinline{String} or \lstinline{Bool} \\
                           or References}                        & Binary        & Comparison \\
                           \midrule
\texttt{\&\&}            & \lstinline{Bool}                      & Binary        & Conjunction \\
\midrule
\texttt{||}              & \lstinline{Bool}                      & Binary        & Disjunction \\
\midrule
\texttt{\^}              & Traits (lhs)                          & Binary        & Forwarding \\
\midrule
\texttt{,}               & \emph{Disjoint}                       & Binary        & Merging \\
\midrule
\texttt{\textbackslash-} & \emph{Subtractable}                   & Binary        & Diff. \\
\midrule
\texttt{:=}              & References (lhs)                      & Binary        & Assignment \\
\midrule
\texttt{>{}>}            & Unit (lhs)                            & Binary        & Sequencing \\
\midrule
\lstinline{if then else} & \lstinline{Bool} (1st operand)        & Ternary       & Conditional \\
\bottomrule
\end{tabular}
\vskip 1ex
{\footnotesize
  lhs: left-hand side; \quad rhs: right-hand side.
}
\end{table}

\section{Primitive Types}

CP supports five primitive types and their literal forms: 
\begin{itemize}
\item \lstinline{Int}: integers, e.g. \lstinline{48}, \lstinline{-2}, \lstinline{0o77} (octal), and \lstinline{0xFF} (hexadecimal);
\item \lstinline{Double}: floating-point numbers, e.g. \lstinline{48.0} and \lstinline{-1e8} (scientific notation);
\item \lstinline{String}: strings, e.g. \lstinline{"foo \n bar"};
\item \lstinline{Bool}: booleans, either \lstinline{true} or \lstinline{false};
\item \lstinline{()}: unit. The unit type has only one value, also denoted by
      \lstinline{()}. It is used as \lstinline{null} in \autoref{ch:arguments}
      and as the return value of variable assignment.
\end{itemize}
Primitive operations on these types can be found in \autoref{tab:op}, which also
lists other operations to be introduced later.

There are also two special types originating from subtyping: \lstinline{Top} for
the top type and \lstinline{Bot} for the bottom type. They are the maximal and
minimal types in the type lattice respectively. The top type is a supertype of
any type, and the bottom type is a subtype of any type. In other words, all
terms can be assigned the top type, while the bottom type is uninhabited
theoretically. However, for practical convenience, \lstinline{undefined} can be
assigned any type, including the bottom type.

\section{Compound Types}

Compound types in CP include references, arrays, records, intersections, and
unions.

\paragraph{References.}
A reference is similar to a pointer in an imperative language. A reference type
is denoted by \lstinline{Ref T}, where \lstinline{T} is the type of the value
stored in the reference cell. Note that reference types are \emph{invariant},
meaning that \lstinline{Ref Int} is neither a subtype nor a supertype of
\lstinline{Ref Top}, in order to ensure type safety.

A reference cell can be created using the \lstinline{ref} keyword. For example,
\lstinline{ref 48} creates a cell of type \lstinline{Ref Int} with the initial
value \lstinline{48}. The value in a reference cell can be accessed using the
\lstinline{!} operator and updated using the \lstinline{:=} operator. For
example, \lstinline{r := !r - 2} updates the value in \lstinline{r} by
subtracting 2. The assignment is an expression that returns \lstinline{()}, i.e.
the unit value. Finally, \lstinline{e1 >> e2} denotes sequential composition,
which means that \lstinline{e2} is returned after \lstinline{e1} (required to
have the unit type) is executed.

\paragraph{Arrays.}
An array type is denoted by \lstinline{[T]}, where \lstinline{T} is the element
type. Since arrays are \emph{immutable} in CP, array types are treated
\emph{covariantly}. For example, \lstinline{[Int]} is a subtype of
\lstinline{[Top]} because \lstinline{Int} is a subtype of \lstinline{Top}.

\begin{tipblock}
A mutable array can be represented as an array of references, i.e.
\lstinline{[Ref T]}. Note that \lstinline{[Ref Int]} is not a subtype of
\lstinline{[Ref Top]} due to the invariance of reference types.
\end{tipblock}

\noindent
An array literal can be conveniently written as \lstinline{[e1; e2]}. Note that
we use semicolons to separate elements instead of commas because
\lstinline{e1,e2} is parsed as a merge. Two arrays can be concatenated using the
\lstinline{++} operator. The length of an array can be obtained like
\lstinline{#arr}, and the \lstinline{i}-th element can be accessed like
\lstinline{arr!!i}.

\paragraph{Records.}
A record type has the form \lstinline|{ l1: T1; l2: T2 }|, where \lstinline{l1}
and \lstinline{l2} are labels, and \lstinline{T1} and \lstinline{T2} are field
types. A record literal has the form \lstinline|{ l1 = e1; l2 = e2 }|, where
\lstinline{e1} and \lstinline{e2} are field expressions. Since there is no tuple
type in CP, a pair can be represented as a record with two fields. Record fields
can be accessed using the dot notation like \lstinline{rcd.l1}.

\begin{tipblock}
An empty record \lstinline|{}| is also allowed, which is inferred to have the
top type \lstinline{Top}. Note that \lstinline|{}| is different from the unit
value \lstinline|()|, which has the unit type \lstinline|()|.
\end{tipblock}

\begin{table}[b!]
\caption{List of operations on records or traits in CP.} \label{tab:rcd}
\centering
\begin{tabular}{ccl}
\toprule
\thead{Subtraction by Label} & \lstinline{e \ l}              & $\triangleq$\quad\lstinline|e \\ { l: Bot }| \\
\midrule
\thead{Subtraction by Type}  & \lstinline{e \\ T}             & $\triangleq$\quad\lstinline{e : T0\T}\quad if \lstinline{e} has type \lstinline{T0} \\
\midrule
\thead{Subtraction by Term}  & \lstinline{e1 \- e2}           & $\triangleq$\quad\lstinline{e1 : T1\T2}\quad if \lstinline{e}$_i$ has type \lstinline{T}$_i$ \\
\midrule
\thead{Field Updating}       & \lstinline|{ e1 with l = e2 }| & $\triangleq$\quad\lstinline|e1\l , { l = e2 }| \\
\midrule
\thead{Label Renaming}       & \lstinline{e [l1 <- l2]}       & $\triangleq$\quad\lstinline|e\l1 , { l2 = e.l1 }| \\
\midrule
\thead{Leftist Merge}        & \lstinline{e1 +, e2}           & $\triangleq$\quad\lstinline{e1 , (e2 \- e1)} \\
\midrule
\thead{Rightist Merge}       & \lstinline{e1 ,+ e2}           & $\triangleq$\quad\lstinline{(e1 \- e2) , e2} \\
\bottomrule
\end{tabular}
\end{table}

\noindent
Two records can be concatenated using the merge operator like \lstinline{e1,e2}.
Many other operations on records are supported via merging and type difference,
as shown in \autoref{tab:rcd}. Subtracting type \lstinline{T1} by type
\lstinline{T2} is denoted by \lstinline{T1\T2}. Taking the difference of record
types as an example, \lstinline|{ l1: Int; l2: Bool } \ { l1: Int }| is equal to
\lstinline|{ l2: Bool }|.

\paragraph{Intersections.}
An intersection of types \lstinline{T1} and \lstinline{T2} is denoted by
\lstinline{T1&T2}. Intersection types have an explicit introduction form in CP,
which is called a merge and written as \lstinline{e1,e2}, where \lstinline{e1}
is an expression of type \lstinline{T1} and \lstinline{e2} is of type
\lstinline{T2}. Intersection types have no explicit elimination form, but they
can be implicitly eliminated via subtyping. For example, \lstinline{e1,e2} can
be used as if it has type \lstinline{T1} or \lstinline{T2}, because
\lstinline{T1} and \lstinline{T2} are subtypes of \lstinline{T1&T2}.

\begin{tipblock}
A multi-field record is encoded as an intersection of single-field records in
CP. For example, a record type with two fields \lstinline|{ l1: Int; l2: Bool }|
is encoded as \lstinline|{ l1: Int } & { l2: Bool }|. Similarly, at the term
level, a two-field record \lstinline|{ l1 = 48; l2 = true }| is encoded as
\lstinline|{ l1 = 48 } , { l2 = true }| using the merge operator.
\end{tipblock}

\paragraph{Unions.}
A union of types \lstinline{T1} and \lstinline{T2} is denoted by
\lstinline{T1|T2}. Union types have an explicit elimination form in CP, written
as \lstinline{switch e0 as x case T1 => e1 case T2 => e2}. Here \lstinline{x} is
a variable bound to the value of \lstinline{e0}, and it is refined to type
\lstinline{T1} in \lstinline{e1} and to \lstinline{T2} in \lstinline{e2}. Union
types have no explicit introduction form, but they can be implicitly introduced
via subtyping, e.g. \lstinline{48 : Int|()} and \lstinline{() : Int|()}.

\section{Functions}

Functions are first-class values in CP, just like in other functional languages.
A function type is written as \lstinline{T1 -> T2}, where \lstinline{T1} is the
parameter type and \lstinline{T2} is the return type. A function literal is
written as \lstinline|\(x: T1) -> e|, where \lstinline{x} is the parameter
variable and \lstinline{e} is the body expression. CP employs curried functions
to achieve multiple arguments. For example, a function of type
\lstinline{Int -> Int -> Int} can be regarded as a function that takes two
integers. For example, an addition function can be defined as follows:
\begin{lstlisting}
add (x: Int) (y: Int) = x + y;
-- is equivalent to:
add = \(x: Int) -> \(y: Int) -> x + y;
\end{lstlisting}
Higher-order functions like function composition can be defined as follows:
\begin{lstlisting}
compose (f: Int -> Int) (g: Int -> Int) (x: Int) = f (g x);
compose (\(x: Int) -> x + 1) (\(x: Int) -> x * 2) 3  --> 7
\end{lstlisting}
Unfortunately, CP does not support type inference for parameter types, so they
must be explicitly annotated as shown above. However, the return type can be
inferred for non-recursive functions. For recursive functions, the return type
must also be annotated. For example, a factorial function can be defined as
follows:
\begin{lstlisting}
fact (n: Int): Int = if n == 0 then 1 else n * fact (n - 1);
-- is equivalent to:
fact = fix fact: Int -> Int.
        \(n: Int) -> if n == 0 then 1 else n * fact (n - 1);
\end{lstlisting}
Recursive functions can be made anonymous using the \lstinline{fix} operator as
shown above.

If a local recursive function is desired rather than a top-level definition, the
\lstinline{letrec} expression can be used. For example, a tail-recursive
version of factorial can be defined as follows:
\begin{lstlisting}
fact = letrec fact' (acc: Int) (n: Int): Int =
         if n == 0 then acc else fact' (n * acc) (n - 1) in
       fact' 1;
\end{lstlisting}
Note that changing \lstinline{letrec} into \lstinline{let} will cause an error
complaining that \lstinline{fact'} is undefined, because \lstinline{let} does
not bind the variable being defined in its own definition. Nevertheless, this
behavior of \lstinline{let} can sometimes be useful for refining an existing
definition:
\begin{lstlisting}
letrec fact (acc: Int) (n: Int): Int =
  if n == 0 then acc else fact (n * acc) (n - 1) in
let fact = fact 1 in
...
\end{lstlisting}
Note that \lstinline{let fact = fact 1} is not a recursive definition. Instead,
\lstinline{fact 1} is partially applying the previously defined \lstinline{fact}
to \lstinline{1}, and the previous \lstinline{fact} is shadowed by the new
\lstinline{fact}.

\paragraph{Named and optional arguments.}
CP natively supports named and optional arguments. The syntax of function
definitions with named arguments are similar to record patterns in ML-like
languages, though CP does not support pattern matching in general. Below is an
exponential function with two named arguments:
\begin{lstlisting}
exp { x: Int; base = 10 }: Int =
  if x == 0 then 1 else base * exp { x = x - 1; base = base };
exp { base = 2; x = 15 }  --> 2^15 = 32768
exp { x = 4 }  --> 10^4 = 10000
\end{lstlisting}
The function can be called with named arguments in any order. Here
\lstinline{base} is an optional argument with a default value \lstinline{10}.
Therefore, the second call \lstinline|exp { x = 4 }| is equivalent to
\lstinline|exp { x = 4; base = 10 }| since \lstinline{base} is absent.

\section{Parametric Polymorphism}

Unlike implicit polymorphism in ML-like languages, CP employs explicit
polymorphism in System-F style. A polymorphic type is written as
\lstinline{forall A. T}, where \lstinline{A} is the type variable (must start
with an uppercase letter). A polymorphic term is written as \lstinline|/\A. e|,
and instantiating a polymorphic term \lstinline{poly} with type \lstinline{Int}
is written as \lstinline{poly @Int}.

For example, a polymorphic identity function can be defined as follows:
\begin{lstlisting}
id A (x: A) = x;
-- is equivalent to:
id = /\A. \(x: A) -> x;
\end{lstlisting}
Calling \lstinline{id @Int 48} will return \lstinline{48} as is. A more
intriguing example is a polymorphic function with \emph{disjoint
quantification}:
\begin{lstlisting}
concat R (r: R) = r , { l = 48 };  -- Type Error!
concat' (R * { l: Int }) (r: R) = r , { l = 48 };  -- OK!
\end{lstlisting}
When concatenating \lstinline{r} with a new field \lstinline|{ l = 48 }|, we
need to check whether it conflicts with any existing field in \lstinline{r}.
However, we do not know anything about \lstinline{r} if it has a totally
abstract type \lstinline{R}. That is why the first definition above is rejected
by the type checker, preventing potential field conflicts. The second definition
type-checks because the quantification \lstinline|(R * { l: Int })| indicates
that the quantified type variable \lstinline{R} is disjoint (will not overlap)
with \lstinline|{ l: Int }|, so \lstinline{r} will never have a field that
conflicts with \lstinline|{ l = 48 }|. The disjointness constraint will be
checked at the point of instantiation:
\begin{lstlisting}
concat' @{ l: Int; l': Int } { l = 46; l' = 0 }  -- Type Error!
concat' @{ l': Int } { l' = 0 }  --> { l' = 0; l = 48 }
\end{lstlisting}
The first instantiation is rejected because \lstinline|{ l: Int; l': Int }| is
not disjoint (overlaps) with \lstinline|{ l: Int }|. The second one type-checks
since the two labels \lstinline{l} and \lstinline{l'} are distinct.

\section{Traits}

CP-flavored traits are composable reuse units and are similar to classes (not
interfaces!) in traditional object-oriented languages. A trait implements an
interface by providing concrete implementations or inheriting from other traits.
A trait expression is anonymous and has the basic form \lstinline{trait => e},
where \lstinline{e} is the body expression, usually a record literal. A simple
example is shown below:
\begin{lstlisting}
type Point = { x: Int; y: Int };       -- interface (a type definition)
\end{lstlisting}
\vspace{-1ex}
\begin{lstlisting}
point (x: Int) (y: Int) =  -- trait with parameters (a term definition)
  trait implements Point => { x = x; y = y };  -- has type Trait<Point>
\end{lstlisting}
Note that traits in CP are \emph{structurally} typed, so the trait has type
\lstinline{Trait<Point>} even without declaring \lstinline{implements Point}.
The \lstinline{implements}-clause is mainly a subtyping check, ensuring that the
trait has implemented all the fields required by the interface, so we recommend
always writing it.

\paragraph{Inheritance.}
The \lstinline{Point} interface above can be extended using intersection types,
and an extended trait can be defined by inheriting from \lstinline{point x y}:
\begin{lstlisting}
type ColorPoint = Point & { color: String };
\end{lstlisting}
\vspace{-1ex}
\begin{lstlisting}
colorPoint (x: Int) (y: Int) (c: String) =
  trait implements ColorPoint inherits point x y => { color = c };
\end{lstlisting}
Instantiating a trait is done by the \lstinline{new} operator (the precedence of
\lstinline{new} is lower than function application):
\begin{lstlisting}
blackPoint = colorPoint 1 2 "black";  -- has type Trait<ColorPoint>
new blackPoint  --> { x = 1; y = 2; color = "black" } : ColorPoint
\end{lstlisting}

\paragraph{Overriding.}
Trait inheritance in CP prevents unexpected overriding by requiring the
\lstinline{override} keyword. One can access fields in the super-trait using the
\lstinline{super} keyword. For example, below we define a point by reflecting
\lstinline{blackPoint}:
\begin{lstlisting}
reflected = trait inherits blackPoint => {
  override x = -super.x;
  override y = -super.y;
};
\end{lstlisting}
The two fields \lstinline{x} and \lstinline{y} are overridden while
\lstinline{color} is inherited. Removing the \lstinline{override} keyword will
cause a type error complaining that disjointness is violated.

\paragraph{Multiple inheritance.}
Following the trait approach to multiple inheritance, there cannot be field
conflicts between multiple super-traits. In other words, one must explicitly
resolve the conflict, for example, by using the restriction operators, renaming,
or biased merging in \autoref{tab:rcd}. Below we show a fine-grained resolution
with restriction operators:
\begin{lstlisting}
mixedPoint = trait [self] inherits blackPoint\color , reflected\\Point => {
  override x = super.x + (reflected^self).x;
};
new mixedPoint  --> { x = 0; y = 2; color = "black" }
\end{lstlisting}
Here \lstinline{blackPoint\color} removes the \lstinline{color} field, while
\lstinline{reflected\\Point} removes all fields in type \lstinline{Point} (i.e.
\lstinline{x} and \lstinline{y}). In the trait body, we intentionally override
\lstinline{x} to show that the excluded \lstinline{x} field in
\lstinline{reflected} can still be accessed using the forwarding operator
\lstinline{^}. Basically, \lstinline{t^self} is the way to access all fields in
another trait \lstinline{t}. We will explain what \lstinline{self} means soon.

\paragraph{Dynamic inheritance.}
Traits in CP are first-class and allow dynamic inheritance, where
a trait can inherit from a trait expression, whose result is determined
dynamically. For example, we can decide whether the inherited trait is reflected
at run time:
\begin{lstlisting}
reflectedOrNot (b: Bool) =
  trait inherits if b then reflected else blackPoint => { amazing = true };
\end{lstlisting}
When combined with parametric polymorphism, dynamic inheritance may require
certain disjointness constraints to ensure that there is no field conflict:
\begin{lstlisting}
mixin (T * { amazing: Bool }) (base: Trait<T>) =
  trait [self: T] inherits base => { amazing = true };
\end{lstlisting}

\section{Recursive Types}

Recursive types are needed if a method refers to its enclosing trait. For
example, let us consider adding a method to the \lstinline{Point} type, which
compares two points for equality:
\begin{lstlisting}
type EqPoint = { x: Int; y: Int; eq: EqPoint -> Bool };
\end{lstlisting}
Now \lstinline{EqPoint} has to refer to itself in the type of the \lstinline{eq}
method. However, the definition above does not work because \lstinline{EqPoint} is
not yet defined when the type of \lstinline{eq} is being checked. To solve this
problem, the \lstinline{interface} keyword can be used:
\begin{lstlisting}
interface EqPoint { x: Int; y: Int; eq: EqPoint -> Bool };
-- is equivalent to:
type EqPoint = fix EqPoint. { x: Int; y: Int; eq: EqPoint -> Bool };
\end{lstlisting}
The form of recursive types implemented in CP is called \emph{iso}-recursive,
meaning that \lstinline{EqPoint} is \emph{not} equivalent to
\lstinline|{ x: Int; y: Int; eq: EqPoint -> Bool }|. Instead, they are isomorphic
up to folding/unfolding. For example, \lstinline{unfold @EqPoint p} converts
\lstinline{p} from type \lstinline{EqPoint} to the record type, while
\lstinline|fold @EqPoint { ... }| does the opposite:
\begin{lstlisting}
getX (p: EqPoint) = (unfold @EqPoint p).x;
getY (p: EqPoint) = (unfold @EqPoint p).y;
eqPoint (x: Int) (y: Int) = fold @EqPoint {
  x = x;  y = y;
  eq (p: EqPoint)  =  getX p == x && getY p == y
};
\end{lstlisting}

\paragraph{Implicit folding/unfolding.}
Nevertheless, CP simplifies typical uses of recursive types by implicitly
inserting \lstinline{fold} and \lstinline{unfold} when necessary. For example,
\lstinline{fold} is automatically inserted for a trait if the declared interface
is a recursive type, and \lstinline{unfold} is inserted for record projection.
The previous example can be rewritten as follows:
\begin{lstlisting}
eqPoint (x: Int) (y: Int) = trait implements EqPoint => {
  x = x;  y = y;
  eq p  =  p.x == x && p.y == y;
};
\end{lstlisting}
Note that we no longer need to specify the type of the method parameter
\lstinline{p} in \lstinline{eq p}, because the type can be inferred from the
declared interface \lstinline{EqPoint}.

\begin{tipblock}
Due to type-theoretic difficulties, a trait implementing a recursive type cannot
be inherited or merged with other traits in CP. There is ongoing research on how
to address this problem.
\end{tipblock}

\section{Self-References}

A more intriguing example is traits with self-type annotations:
\begin{lstlisting}
type Even = { isEven: Int -> Bool };
type Odd  = { isOdd:  Int -> Bool };
\end{lstlisting}
\vspace{-1ex}
\begin{lstlisting}
even = trait [self: Odd] implements Even => {
  isEven n  =  if n == 0 then true else self.isOdd (n - 1);
};  -- has type Trait<Odd => Even>
\end{lstlisting}
\vspace{-1ex}
\begin{lstlisting}
odd = trait [self: Even] implements Odd => {
  isOdd n  =  if n == 0 then false else self.isEven (n - 1);
};  -- has type Trait<Even => Odd>
\end{lstlisting}
Here the annotations \lstinline{[self: Odd]} and \lstinline{[self: Even]} allow
the trait bodies to refer to late-bound self-references by \lstinline{self},
which is declared to have types \lstinline{Odd} and \lstinline{Even}
respectively. By this means, \lstinline{even} and \lstinline{odd} are not
coupled with each other, allowing them to be composed with other
implementations. Self-type annotations play a crucial role in modular
dependencies in CP.

The trait type becomes a bit more complicated with a self-type annotation,
written as \lstinline{Trait<In => Out>}, where \lstinline{In} is the self-type
and \lstinline{Out} is the inferred type for the trait body (a subtype of, but
not necessarily the same as, the declared interface).

\begin{tipblock}
If we review the type \lstinline{Trait<Point>} in last section, we will find
that it is not the most precise type for the trait \lstinline{point x y}.
\lstinline{Trait<Point>} is a shorthand for \lstinline{Trait<Point => Point>} in
the latest version of CP, which is a supertype of \lstinline{Trait<Top => Point>}.
The latter is more precise because it indicates that the self-reference is not
used in the trait body (\lstinline{Top} contains no information).
\end{tipblock}

\noindent
If we try to instantiate the traits separately by \lstinline{new even} or
\lstinline{new odd}, a type error will be triggered saying that the required
interface is not satisfied (the input type is not supertype of the output type).
However, we can instantiate them together since they satisfy each other's
requirement:
\begin{lstlisting}
new even,odd  --> { isEven = ...; isOdd = ... }
\end{lstlisting}
Here \lstinline{new} is effectively obtaining a fixpoint of the merge of the two
trait, allowing late binding of the self-reference. The merge
\lstinline{even,odd} works because \lstinline{Trait<Odd=>Even> & Trait<Even=>Odd>}
can be used as \lstinline{Trait<Odd&Even>}.

\section{Compositional Interfaces and Method Patterns}

Although CP does not support algebraic data types or pattern matching in a
strict sense, similar functionalities can be achieved using compositional
interfaces and method patterns.

\emph{Compositional interfaces} refer to a kind of special interfaces that
contain constructors. A compositional interface is parametrized by a
\emph{sort}, which serves as the return type for the constructors. For example,
an abstract syntax tree for a language with integer literals and addition can be
defined as follows, with a Haskell counterpart (using the notation of
generalized algebraic data types) on the right for comparison:
\begin{lstlisting}
type ExpSig<Exp> = {             -- data Exp where
  Lit: Int -> Exp;               --   Lit :: Int -> Exp
  Add: Exp -> Exp -> Exp;        --   Add :: Exp -> Exp -> Exp
};
\end{lstlisting}
Here \lstinline{Exp} is a sort, which is delimited by angle brackets to
distinguish it from a normal type parameter. \lstinline{ExpSig} contains two
constructors: \lstinline{Lit} takes an integer and \lstinline{Add} takes two
subexpressions, both returning the sort \lstinline{Exp}. Note that constructors
must start with an uppercase letter to distinguish them from normal fields.

\lstinline{ExpSig} can be implemented by providing a method \lstinline{eval}:
\begin{lstlisting}
type Eval = { eval: Int };              -- eval :: Exp -> Int
evalExp = trait implements ExpSig<Eval> => {
  (Lit     n).eval = n;                 -- eval (Lit     n) = n
  (Add e1 e2).eval = e1.eval + e2.eval; -- eval (Add e1 e2) = eval e1 + eval e2
};
\end{lstlisting}
This syntax is similar to pattern matching in Haskell on the right and is thus
called \emph{method patterns}. The sort is instantiated with a concrete type
\lstinline{Eval} in \lstinline{ExpSig<Eval>}, which represents a method
\lstinline{eval} defined for both constructors \lstinline{Lit} and
\lstinline{Add}.

\begin{tipblock}
Different from algebraic data types, which are closed in Haskell, compositional
interfaces and traits are open to extension in CP. As we have seen, interfaces
can be extended via intersection types, and traits can be inherited or extended
via merging.
\end{tipblock}

\noindent
Furthermore, a wildcard pattern is also supported in CP. For example, the trait
below evaluates all \lstinline{Lit} and \lstinline{Add} expressions to \lstinline{0}:
\begin{lstlisting}
evalDummy = trait implements ExpSig<Eval> => { _.eval = 0 };  -- eval _ = 0
\end{lstlisting}

\section{Modular Dependencies}

A merit of compositional interfaces is that they can modularly handle
dependencies between different methods for the constructors. Consider a funny
pretty-printer below:
\begin{lstlisting}
type Print = { print: String };
printExp = trait implements ExpSig<Eval => Print> => {
  (Lit     n).print = toString n;
  (Add e1 e2).print = if e2.eval == 0 then e1.print
                      else e1.print ++ " + " ++ e2.print;
};
\end{lstlisting}
Here \lstinline{(Add e1 e2).print} depends on \lstinline{e2.eval}, but there is
no implementation of \lstinline{eval} in \lstinline{printExp}. To make such a
dependency work, instead of \lstinline{ExpSig<Print>}, we declare the interface
as \lstinline{ExpSig<Eval => Print>}. This notation means that \lstinline{e1}
and \lstinline{e2} (and other subexpressions in constructors, if any) has type
\lstinline{Eval&Print}. They only have type \lstinline{Print} with
\lstinline{ExpSig<Print>}.

It is useful to compare this example with a simpler approach based on
inheritance:
\begin{lstlisting}
printChild = trait implements ExpSig<Eval&Print> inherits evalExp => {
  (Lit     n).print = toString n;
  (Add e1 e2).print = if e2.eval == 0 then e1.print
                      else e1.print ++ " + " ++ e2.print;
};
\end{lstlisting}
The body of \lstinline{printChild} is exactly the same as that of
\lstinline{printExp}. The difference is that \lstinline{printChild} implements
\lstinline{ExpSig<Eval&Print>} by inheriting from \lstinline{evalExp},
essentially containing both \lstinline{eval} and \lstinline{print}. Although
both approaches work, \lstinline{printChild} is tightly coupled with
\lstinline{evalExp}, while \lstinline{printExp} can be composed with any other
trait that implements \lstinline{eval}. By the way, the former is similar to
directly writing the following code in Haskell:
\begin{lstlisting}
print (Add e1 e2) = if eval e2 == 0 then ...
\end{lstlisting}
The definition of \lstinline{print} is tightly coupled with a concrete function
\lstinline{eval}. We could use type classes to decouple dependencies in Haskell,
as seen in tagless-final embeddings, but this would require planning in advance
for \lstinline{eval} and \lstinline{print}. In CP, the natural way to implement
\lstinline{eval} and \lstinline{print} is already modular and extensible,
requiring no pre-planning.

\paragraph{Combination with self-references.}
As introduced previously, self-type annotations are also a powerful mechanism
for modular dependencies. In fact, we can combine the two mechanisms together
to achieve more flexible and modular designs. For example, we can define a
funnier pretty-printer like this:
\begin{lstlisting}
printSelf = trait implements ExpSig<Eval => Print> => {
         (Lit     n).print = toString n;
  [self]@(Add e1 e2).print = if self.eval == 0 then "0"
                             else e1.print ++ " + " ++ e2.print;
};
\end{lstlisting}
Now \lstinline{(Add e1 e2).print} depends on \lstinline{self.eval} rather than
\lstinline{e2.eval}. Therefore, we add a self-type annotation in
\lstinline{[self]@(Add e1 e2)}. The type of \lstinline{self} is inferred to be
\lstinline{Eval&Print} according to the interface \lstinline{ExpSig<Eval => Print>},
but it can be explicitly declared as \lstinline{[self: Eval]@(Add e1 e2)} if needed.

\paragraph{Virtual constructors.}
Before concluding this section, we briefly introduce how to use the traits to
construct an expression and pretty-print it. A lightweight way is: first merging
\lstinline{evalExp} into \lstinline{printSelf} to form a self-contained trait,
and then using the \lstinline{open} keyword to bring into scope both
constructors in \lstinline{factory}:
\begin{lstlisting}
factory = new evalExp,printSelf : ExpSig<Eval&Print>;
let e1 = open factory in Add (Lit 4) (Lit 8) in e1.print  --> "4 + 8"
let e2 = open factory in Add (Lit 0) (Lit 0) in e2.print  --> "0"
\end{lstlisting}
Note that we do not need to add \lstinline{new} before \lstinline{Add} and
\lstinline{Lit}, because \lstinline{new} is implicitly inserted for constructors
(syntactically starting with an uppercase letter). In case \lstinline{new} is
not desired, \lstinline{$Add} and \lstinline{$Lit} can be used instead. In other
words, \lstinline{Lit 0} is equivalent to \lstinline{new $Lit 0}.

Another way is to define a \lstinline{repo} trait with a self-type annotation:
\begin{lstlisting}
repo Exp = trait [self: ExpSig<Exp>] => open self in {
  e1 = Add (Lit 4) (Lit 8);
  e2 = Add (Lit 0) (Lit 0);
};
\end{lstlisting}
This trait allows us to use constructors \emph{virtually} to construct
expressions without committing to a concrete implementation of \lstinline{ExpSig}.
Then we can, for example, apply \lstinline{repo} to type \lstinline{Eval&Print}
and merge it with \lstinline{evalExp} and \lstinline{printSelf}:
\begin{lstlisting}
exp = new evalExp , printSelf , repo @(Eval&Print);
exp.e1.print  --> "4 + 8"
exp.e2.print  --> "0"
\end{lstlisting}

\section{Implementations}

There are two implementations of the CP language:
\begin{enumerate}
\item A reference interpreter bundled with the original CP
      paper~\citep{zhang2021compositional}, which is no longer maintained and is
      archived at \url{https://github.com/wxzh/CP}.
\item A new implementation including both an interpreter and a compiler to
      JavaScript, which is under active development at
      \url{https://github.com/yzyzsun/CP-next}.
\end{enumerate}
Both implementations first desugar CP source code into a core calculus called
\fiplus, and then interpret or compile the \fiplus code. The main difference
between the two interpreters is that the original one further elaborates \fiplus
to a simpler calculus called \fco~\citep{bi2019distributive} before evaluation,
while the new interpreter directly evaluates \fiplus code according to
type-directed operational semantics~\citep{huang2021taming,fan2022direct}. The
internals of the new compiler will be introduced in \autoref{pt:compile}.
