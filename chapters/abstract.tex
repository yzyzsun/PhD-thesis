Compositionality is an important principle in software engineering, meaning that
a complex system can be built by composing simpler parts. However, achieving
compositionality in practice remains challenging, as exemplified by the
expression problem, which highlights the tension between modularity and
extensibility. CP, a new statically typed programming language, naturally solves
such challenges through language-level support for compositional programming.

This thesis mainly studies the practical aspects of CP. We start with a crash
course in CP and showcase why compositional programming matters with two
applications. First, regarding object-oriented programming, CP innovates with a
trait-based model with merging that prevents implicit overriding in inheritance.
By comparing with type-unsafe code in TypeScript, we show that CP achieves
dynamic multiple inheritance and family polymorphism without sacrificing type
safety. Second, regarding domain-specific languages, CP enables a novel
embedding technique that combines the advantages of shallow and deep embeddings
and surpasses other techniques like tagless-final embeddings by supporting
modular dependencies.

Next, we detail the design and implementation of the CP compiler. With novel
language features for compositionality, the efficient compilation of CP code is
non-trivial, especially when separate compilation is desired. Our key innovation
is to compile merges to type-indexed records, which outperforms prior theoretic
work based on nested pairs. To maintain type safety in trait inheritance, CP's
type system employs coercive subtyping, which incurs a performance penalty in
compiled code. We mitigate the issue with several optimizations, including
eliminating coercions for equivalent types. We evaluate the impact of these
optimizations using benchmarks and show that the optimized compiler targeting
JavaScript can be orders of magnitude faster than a naive compilation scheme,
obtaining performance on a par with class-based JavaScript programs.

Finally, we extend CP with union types, complementing ubiquitous intersection
types, in order to provide a solid foundation for named and optional arguments.
Our approach resolves a critical type-safety issue found in popular static type
checkers for Python and Ruby, particularly in handling first-class named
arguments in the presence of subtyping. A detailed comparative analysis of named
and optional arguments in existing languages shows that CP's design achieves a
good balance of simplicity and effectiveness.

Both the compilation scheme for CP and the encoding of named and optional
arguments are formalized in Coq and proven to be type-safe.

\vspace{1.5\baselineskip}

\noindent\makebox[\linewidth]{\rule{0.7\textwidth}{0.4pt}}

\begin{center}
\textbf{An abstract of 373 words}
\end{center}
