\begin{lstlisting}
{-# LANGUAGE DuplicateRecordFields, NamedFieldPuns, OverloadedRecordDot, RecordWildCards #-}
\end{lstlisting}

\subsubsection{Modular Interpretations in Tagless-Final Embeddings}

\begin{lstlisting}[language=Haskell]
data Vector = Vector { x :: Double, y :: Double } deriving Show
\end{lstlisting}
\noindent
A tagless-final embedding defines region constructors in a type class,
instead of a closed algebraic data type:

\begin{lstlisting}[language=Haskell,deletekeywords={union,intersect}]
class RegionHudak repr where
  circle    :: Double -> repr
  outside   :: repr -> repr
  union     :: repr -> repr -> repr
  intersect :: repr -> repr -> repr
  translate :: Vector -> repr -> repr
\end{lstlisting}
\noindent
\lstinline!Size! can be modularly defined as an instance
of the type class since it has no dependency:

\begin{lstlisting}[language=Haskell,deletekeywords={union,intersect}]
newtype Size = S { size :: Int }

instance RegionHudak Size where
  circle      _ = S 1
  outside     a = S $ a.size + 1
  union     a b = S $ a.size + b.size + 1
  intersect a b = S $ a.size + b.size + 1
  translate _ a = S $ a.size + 1
\end{lstlisting}
\noindent
But how about \lstinline!Text!? As a first try, we might
write:

\begin{lstlisting}[language=Haskell]
newtype Text = T { text :: String }

instance RegionHudak Text where
   circle  r = T { text = "circular region of radius " ++ show r }
-- outside a = T { text = "outside a region of size " ++ show a.size }
-- ......
\end{lstlisting}
\noindent
We will get a type error concerning \lstinline!a.size! if
we uncomment the line above, because \lstinline!a! has
type \lstinline!Text! and thus does not contain a field
named \lstinline!size!. So the problem is that, once we
need operations that have some dependencies on other operations, we get
into trouble! But programs with dependencies are common in practice.
This is a serious limitation.

\subsubsection{A Non-Modular Workaround for Dependencies}

A simple workaround is to pack the two operations together and duplicate
the code of size calculation. We have already described this approach
for shallow embeddings, and the same workaround works for tagless-final
embeddings:

\begin{lstlisting}[language=Haskell,deletekeywords={union,intersect}]
data SizeAndText = ST { size :: Int, text :: String }

instance RegionHudak SizeAndText where
  circle  r = ST { size = 1, text = "circular region of radius " ++ show r }
  outside a = ST { size = a.size + 1, text = "outside a region of size "
                                           ++ show a.size }
  union a b = ST { size, text = "union of two regions of size "
                              ++ show size ++ " in total" }
    where size = a.size + b.size + 1
  intersect a b = ST { size, text = "intersection of two regions of size "
                                  ++ show size ++ " in total" }
    where size = a.size + b.size + 1
  translate v a = ST { size, text = "translated region of size " ++ show size }
    where size = a.size + 1
\end{lstlisting}
\noindent
However, this is not modular since we have to duplicate code for size
calculation. If we have another operation that depends on
\lstinline!size!, we have to repeat the same code even
again. This workaround is an \emph{anti-pattern}, which violates
basic principles of software engineering.

\subsubsection{Modular Language Constructs}

Of course, new region constructors can be modularly added in a
tagless-final embedding:

\begin{lstlisting}[language=Haskell]
class RegionHofer repr where
  univ  :: repr
  empty :: repr
  scale :: Vector -> repr -> repr
\end{lstlisting}
\noindent
We have to resort to the workaround again when we encounter mutual
recursion:

\begin{lstlisting}[language=Haskell,deletekeywords={union,intersect}]
data UE = UE { isUniv :: Bool, isEmpty :: Bool }

instance RegionHudak UE where
  circle      _ = UE { isUniv = False,     isEmpty = False }
  outside     a = UE { isUniv = a.isEmpty, isEmpty = a.isUniv }
  union     a b = UE { isUniv = a.isUniv || b.isUniv
                     , isEmpty = a.isEmpty && b.isEmpty }
  intersect a b = UE { isUniv = a.isUniv && b.isUniv
                     , isEmpty = a.isEmpty || b.isEmpty }
  translate _ a = UE { isUniv = a.isUniv,  isEmpty = a.isEmpty }

instance RegionHofer UE where
  univ      = UE { isUniv = True,     isEmpty = False }
  empty     = UE { isUniv = False,    isEmpty = True }
  scale _ a = UE { isUniv = a.isUniv, isEmpty = a.isEmpty }
\end{lstlisting}

\subsubsection{Use Case}

We can use all constructors from \lstinline!RegionHudak!
and \lstinline!RegionHofer! to create a region, as long as
\lstinline!UE! implements both type classes:

\begin{lstlisting}[language=Haskell,deletekeywords={union,intersect}]
region :: UE
region = outside empty `union` circle 1

main :: IO ()
main = do putStrLn $ "Univ:  " ++ show region.isUniv
          putStrLn $ "Empty: " ++ show region.isEmpty
\end{lstlisting}
