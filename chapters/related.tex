\chapter{Related Work} \label{ch:related}

\section{Solutions to the Expression Problem}

\section{Embedded Domain-Specific Languages}

\paragraph{Modular embeddings.}
There are quite a few existing approaches to modular embeddings, which
essentially exploit solutions to the \emph{expression
problem}~\citep{wadler1998expression} in existing languages. Such approaches
include, for example, \emph{tagless-final embeddings}~\citep{carette2009finally}
and \emph{data types à la carte}~\citep{swierstra2008data} in functional
programming, \emph{polymorphic embeddings}~\citep{hofer2008polymorphic} and
\emph{object algebras}~\citep{oliveira2012extensibility} in object-oriented
languages, and solutions to the \emph{expression compatibility
problem}~\citep{haeri2016expression,haeri2019solving}, just to name a few.
\autoref{sec:comparison} already compares those approaches with compositional
embeddings in detail. In essence, the lack of sufficient programming language
support for modularity makes it hard to model modular dependencies in those
approaches. Compositional embeddings offer an elegant solution to many issues by
exploiting the support for Compositional Programming in the CP language.

\paragraph{Hybrid embeddings.}
\citet{svenningsson2015combining} propose combining shallow and deep embeddings
by translating a shallowly embedded interface to a deeply embedded core
language. In this way, language interfaces can be shallowly extended, and
multiple interpretations are possible in the deep core. In addition, some
features in the host language can be exploited in the shallow part, which is
called \emph{deep linguistic reuse} in Scala-Virtualized~\citep{rompf2012scala}.
\citet{jovanovic2014yinyang} further propose an automatic translation between
shallow and deep embeddings using Scala macros instead of a manual translation.
Their Yin-Yang framework targets lightweight modular
staging~\citep{rompf2010lightweight} as the deep embedding backend but
completely conceals internal encodings from the users. A similar but slightly
different approach called \emph{implicit staging} is proposed by
\citet{scherr2014implicit}. Implicit staging extracts an intermediate
representation from a shallowly embedded DSL and reintegrates it after some
transformations. Their research prototype works in Java through load-time
reflection. Compared to such hybrid approaches, compositional embeddings do not
require the use of two different embeddings and the translations between them
while supporting most features from both shallow and deep embeddings.

\paragraph{Generative programming.}
As stated in \autoref{sec:dsl}, \ExT performs a lightweight desugaring during
parsing to generate compositionally embedded fragments in CP. This is an
\emph{ad-hoc} generative technique. Some generative approaches in other
languages, such as Racket macros~\citep{ballantyne2020macros} and Template
Haskell~\citep{sheard2002template}, provide more flexible mechanisms compared to
ours. Nevertheless, in this work, our goal is not to develop generative
programming in CP. Instead, we try to keep the DSL as close to the underlying
compositional embeddings as possible so that there is an obvious one-to-one
mapping between \ExT and CP. Rather than syntactic extensibility (i.e.~the
ability to extend the syntax of the source language), our focus is on semantic
support for DSLs. Our approach is different from some generative frameworks like
EVF~\citep{zhang2017evf} and Castor~\citep{zhang2020castor}, which generate
extensible visitors from annotations via meta-programming. Although they also
enable modular and extensible programming language components, their code uses
various non-standard annotations, whose semantics may be unclear to programmers.
What is worse, type checking is delayed in the aforementioned generative
frameworks so error messages are reported in terms of the generated code. This
can be quite confusing without detailed knowledge of how the generated code
works. In contrast, in our implementation, type checking is done in the source
language, and error messages are reported in terms of the original \ExT code.

\paragraph{Language workbenches.}
JetBrains MPS~\citep{mps}, Xtext~\citep{xtext},
MontiCore~\citep{krahn2010monticore}, and Spoofax~\citep{kats2010spoofax}, just
to name a few, are popular \emph{language
workbenches}~\citep{fowler2005language} that enable easier DSL development.
\citet{erdweg2015evaluating} have done a comprehensive evaluation about them.
Notably, in addition to textual DSLs, JetBrains MPS supports tables,
mathematical symbols, and diagrammatic notations, which significantly enhance
user experience. Modern language workbenches also provide editor support for
user-defined DSLs. Similar to compositional embeddings, they often allow
language components to be modularly composed, but their focus is mainly on
syntactic extensibility instead of semantic extensibility. In contrast to direct
embeddings into host languages, external tools are required by language
workbenches to generate DSL implementations from specifications. This is the
essential difference from compositional embeddings, where composition is
directly built in the programming language.

\paragraph{Document DSLs.}
There are quite a few existing DSLs for document authoring. \emph{\LaTeX} is one
of the most commonly used document languages, and \ExT's syntax is inspired by
it. In contrast to \ExT, \LaTeX{} is an external DSL and does not have a static
type system. \LaTeX{} supports arbitrary computation, but it is not easy to
write meta-programs over the AST of \LaTeX. Moreover, \LaTeX{} is not intended
to render web documents, which is an important goal of \ExT.
\emph{Wikitext}~\citep{wikitext} is widely used across wiki websites, including
Wikipedia and Fandom. Wikitext has some design limitations like the absence of
type safety, data consistency, and general-purpose computation, which have been
elaborated in \autoref{sec:minipedia}. \emph{Skribe}~\citep{gallesio2005skribe}
is a document DSL built on top of the Scheme programming language and extends
Scheme's syntax from S-expressions to \emph{Sk-expressions} for easier text
writing. Skribe makes use of Scheme macros and functions to construct documents.
This idea is inherited and popularized by
\emph{Scribble}~\citep{flatt2009scribble} in the Racket community, and DrRacket
offers very good tool support for Scribble. Regarding syntax, a new
\emph{@-notation} is designed for Scribble, which is desugared to normal
S-expressions. \ExT's syntax is similar to Scribble's but is more flexible.
Based on the infrastructure provided by Racket, Scribble supports arbitrary
computation and linguistic reuse like \ExT. However, Scribble has a fixed
document structure~\citep{scribble}, and its rendering function performs
conventional pattern matching on it. Thus, it is not easy to extend their core
document constructs and functions without modifying the original source code.
Furthermore, there is no static typing in Scribble, though it uses Racket's
\emph{contract} system to perform run-time type checking.

\section{Compilation of Inheritance}

In his excellent survey on inheritance, \citet{taivalsaari1996notion}
distinguishes two strategies for implementing inheritance: \emph{delegation} and
\emph{concatenation}. Most prototype-based languages, such as
\textsc{Self}~\citep{ungar1987self} and JavaScript, implement inheritance via
delegation, where an object contains a reference to its prototype (e.g.
\lstinline{__proto__} in JavaScript), and methods that are not found in the
current object will be delegated to its ancestors in the prototype chain. In
contrast, CP implements inheritance via concatenation (a.k.a. \emph{merging}
throughout the paper), where a trait is self-contained and itself contains all
the methods of its ancestors. Although some copying is involved, the
concatenation strategy is more efficient than delegation in terms of method
lookup.

To improve the performance of method lookup, newer implementations of the
\textsc{Self} language cache all lookup results for a polymorphic call site in a
\emph{polymorphic inline cache} (PIC)~\citep{holzle1991optimizing}. The methods
cached in a PIC will be inlined into the caller to further reduce the overhead
of method calls. Since a PIC is empty until a method is called for the first
time, dynamic recompilation is required to optimize the code at run time.
Moreover, the presence of \emph{dynamic inheritance} may lead to a full method
lookup in \textsc{Self}~\citep{chambers1992design}. Modern JavaScript engines,
such as V8 used in Node.js, utilize similar PIC-based techniques to optimize
method calls. Though the CP compiler does not implement inlining at all, which
is definitely a useful optimization, it is still efficient in terms of method
lookup, and dynamic inheritance \emph{never} causes a slower lookup.

Typical compilers for mainstream class-based languages, such as C++ and Java,
add a \emph{virtual method table} (vtable)~\citep{driesen1995message} to each
object to avoid searching for methods in the inheritance hierarchy at run time.
A vtable is basically an array of function pointers, associating each method
name (and parameter types if overloaded) with its implementation. Similarly, CP
compiles an object to a type-indexed record, which also associates each method
name and type with the corresponding implementation, among other fields. What is
more, CP allows for first-class classes (traits) and dynamic inheritance, which
are not supported by most mainstream languages. This is one of the key
differences of our work compared to other OOP language compilers.

Another significant difference from mainstream OOP languages is that our
compilation of inheritance is based on the denotational model by
\citet{cook1989denotational}. In this model, classes (traits) are encoded as
functions, and inheritance is essentially merging functions, which is
illustrated in \autoref{sec:inheritance}. That is why the source language of the
compilation scheme (\lambdaiplus) does not contain any notion of classes or
objects. Such encodings are common in the literature on foundations for
statically typed
OOP~\citep{bruce1999comparing,bruce2002foundations,pierce2002types}, and they
largely simplify the formalization of compilation and its metatheory.

Since objects are encoded as multi-field records, which are essentially merges
of single-field records in \lambdaiplus, the compilation of objects is closely
related to the elaboration of merges. Next, we discuss the elaboration of
intersection types and the merge operator.

\paragraph{Elaboration of intersection types and the merge operator.}
\citet{dunfield2014elaborating} shows that \emph{unrestricted} intersection
types and a term-level merge operator~\citep{reynolds1997design} can encode
various features like overloading and multi-field records, and they can be
elaborated into product types and pairs. However, her approach lacks the
critical property of coherence, i.e. the property that ensures the result of a
merge is unambiguous. In the follow-up work on \emph{disjoint} intersection
types~\citep{oliveira2016disjoint}, the merged components are required to be
disjoint with each other to avoid the semantic ambiguity.
\citet{alpuim2017disjoint} added parametric polymorphism to the calculus.
\citet{bi2018essence,bi2019distributive} further enhanced the intersection
subtyping with distributivity, enabling more novel features like nested
composition and family polymorphism. In other words, only
\citeauthor{bi2019distributive}'s \fiplus calculus fully covers the topics
mentioned in \autoref{sec:overview}. All the aforementioned work employs
elaboration semantics with standard $\lambda$-calculi serving as targets. They
use nested pairs as the target of elaboration, and consequently, the time
complexity of extracting a component by type can degenerate to linear in the
worst case. In addition, extensible records require fewer coercions than nested
pairs because some different source terms compile to equivalent records. These
differences from our CP compiler have been discussed in detail in
\autoref{sec:dunfield}. In short, they do not consider more efficient runtime
representations or eliminating redundant coercions, nor do they have benchmarks
to evaluate performance. Instead, their focus is on proving the type safety and
coherence of the elaboration. Furthermore, none of the aforementioned work
develops a language with separate compilation units.

The compilation of merges in our work has similarities to the compilation of
\emph{type-indexed rows}~\citep{shields2001type}, where record labels are
discarded and record fields are sorted by their types. However, the work on
type-indexed rows does not consider subtyping, which eliminates many of the
issues that we had to deal with. For instance, they do not need to apply
coercions to ensure that information statically hidden by subtyping is also
hidden at run time.

\paragraph{Compilation of extensible records.}
\citet{ohori1995polymorphic} investigates a polymorphic record calculus and
introduces an efficient type-directed compilation method for records. Following
the type-inference stage, records are converted into vectors with explicit
indexing. However, his records are not extensible, and his method has
difficulties to handle subtyping. Subtyping for records frequently enables field
hiding and reordering, rendering it impossible to determine a label's offset
statically. \citet{gaster1996polymorphic} propose a compilation technique for
polymorphic extensible records that utilizes qualified
types~\citep{jones1994theory}. During the compilation process to the target
language, supplementary parameters are introduced to determine suitable offsets.
This approach is integrated into Hugs, a well-known implementation of Haskell,
as an extension. Their system is later generalized by type-indexed
rows~\citep{shields2001type}. In summary, subtyping and record concatenation (or
merges) pose significant challenges to the compilation of extensible records.
Our work takes pragmatic considerations into account, including targeting widely
used dynamic languages such as JavaScript. As a result, we rely on the primitive
support of objects and object extension in our target language and do not delve
into low-level representations of extensible records, for which a comprehensive
summary can be found in the paper by \citet{leijen2005extensible}.

\begin{comment}
\paragraph{Compilation of feature-oriented programming.}
\emph{Feature-oriented programming}
(FOP)~\citep{prehofer1997feature,apel2009overview} is a programming paradigm
that aims to modularize features in software product
lines~\citep{apel2013feature}. There is a debate on what modularity exactly
means, and \citet{kastner2011road} mention two notions of modularity:
\emph{cohesion} and \emph{information hiding}. The majority of FOP
work~\citep{batory2004scaling,apel2005featurec++,apel2013language} focuses on
the notion of cohesion and basically does source-to-source transformations,
which hinders modular type checking and separate compilation. There is some
other work, such as Jiazzi~\citep{mcdirmid2001jiazzi} and
Scala~\citep{odersky2005scalable}, leveraging information hiding instead and
supporting modular type checking and separate compilation. However, FOP is
usually achieved via verbose design patterns or metaprogramming in those
languages. For instance, some precursor work of compositional programming, done
in Scala, employs design patterns based on object
algebras~\citep{oliveira2012extensibility} to achieve
FOP~\citep{oliveira2013feature,rendel2014object}. Since merging is not directly
supported in Scala, specialized composition operators are required to simulate
merges for different object algebra interfaces. The creation and use of those
composition operators cause a significant burden for developing programs. In
contrast, CP natively supports merges, eliminating the need for such specialized
composition operators and supporting FOP more directly.

\emph{Delta-oriented programming}
(DOP)~\citep{schaefer2010delta} is an extension of FOP, which features delta
modules that can add, remove, or modify classes. A feature module is a delta
module without the remove operation. DOP supports compositional type
checking~\citep{bettini2013compositional} at the level of source code. More
recently, a core calculus for \emph{dynamic} DOP is proposed by
\citet{damiani2018core} to support runtime
variability~\citep{hallsteinsen2008dynamic} and is proven to be type-safe.
However, dynamic DOP is not yet implemented, and its separation compilation is
unexplored.
\end{comment}

\section{Multiple, Dynamic Inheritance and Virtual Classes}

\emph{Multiple inheritance} is a well-known troublemaker in OOP languages,
bringing the diamond problem and method conflicts, among other issues.
Alternative notions like \emph{mixins}~\citep{bracha1990mixin} and
\emph{traits}~\citep{ducasse2006traits} are proposed to alleviate the issues. A
core difference between mixins and traits is how they handle conflicts when the
same method name occurs in multiple ancestors. Mixins resolve conflicts
implicitly by linearization (e.g. C3
linearization~\citep{barrett1996monotonic}). However, the implicit resolution of
conflicts may conceal accidental conflicts and lead to subtle bugs. Traits, on
the other hand, require the programmer to resolve conflicts explicitly. CP
adopts the trait model and imposes the disjointness constraint on merging (and
trait inheritance). Note that the disjointness constraint does not only consider
the method names but also takes into account the types of the methods, so the
methods with the same name but different return types are considered disjoint
and do not conflict with each other. By this means, CP tries to reach a balance
between safety and flexibility.

\paragraph{Dynamic inheritance and first-class classes.} \label{sec:first-class}
While various forms of multiple inheritance are well studied and implemented in
some popular languages, such as C++, Ruby, and Scala, dynamic inheritance is
more challenging and involved, especially in terms of static typing. In the
literature of OOP, dynamic inheritance is often discussed in a broader context
of \emph{first-class classes}~\citep{strickland2013contracts}, where inherited
classes can be determined at run time, among other dynamic features. There are
only a few statically typed languages that support first-class classes. To the
best of our knowledge, they are \texttt{gbeta}~\citep{ernst2000gbeta},
TypeScript~\citep{typescript}, Typed Racket~\citep{takikawa2012gradual},
Wyvern~\citep{lee2015theory}, and most recently,
CP~\citep{zhang2021compositional}. As elaborated in \autoref{sec:background},
the most popular one, TypeScript, has significant type-safety issues when
dealing with dynamic inheritance.

Typed Racket is gradually typed and uses row polymorphism to represent class
types. Similarly to the disjointness constraints in CP, there are constraints on
row variables to express absence, and thus the \emph{inexact superclass problem}
that TypeScript suffers from is resolved in Typed Racket. However, the absence
constraint on a row variable only includes the method name but not the type, so
the dynamically inherited class is more restricted than in CP. Moreover,
\citet{xie2020row} formally prove that CP's disjoint polymorphism is more
powerful than similar forms of row polymorphism. Furthermore, unlike Typed
Racket, CP can model virtual classes and family polymorphism. 

Wyvern is a language for design-driven assurance, and \citet{lee2015theory}
explored a foundational account of first-class classes based on
tagging~\citep{glew1999type}. Similarly to our formalization, they give an
elaboration semantics of an OOP language. However, their theory is very
different from ours, and they target a more sophisticated calculus with
hierarchical tagging and dependent types. In contrast, our target language is a
standard record calculus. Furthermore, their calculus cannot model multiple
inheritance or family polymorphism, and their implementation is an interpreter
rather than a compiler.

The \texttt{gbeta} language is the most interesting one and is the closest to CP
because it supports dynamic multiple inheritance and family polymorphism.
However, separate compilation was not supported at the time when
\citet{ernst2000gbeta} wrote his dissertation because of some technical issues
with the Mjølner BETA persistence support. If this factor is disregarded,
separate compilation can still be accomplished, but at the cost of efficient
attribute lookup. Since an object in \texttt{gbeta} may have more mixin
instances at run time than what is statically known, and the mixin instances may
occur in a different order, the offset of an attribute cannot be determined
statically. As a result, \texttt{gbeta} has to perform a linear search through
super-mixins to look up inherited attributes. In contrast, attribute lookup in
CP, even with dynamic inheritance, is more efficient, and no linear search is
needed.

The notion of \emph{patterns} in \texttt{gbeta} unifies classes and methods, and
patterns can be composed using the combination operator `\&', which is similar
to the merge operator `$\bbcomma$' in CP. Though dynamic multiple inheritance
can be achieved using `\&', only a subset of dynamic combinations is safe, where
at least one of the classes being composed must be created by single
inheritance~\citep{ernst2002safe}. Otherwise, the C3 linearization algorithm
used by \texttt{gbeta} may fail at run time. In contrast, dynamic inheritance
via `$\bbcomma$' is completely type-safe, because CP utilizes disjointness to
avoid conflicts, and no linearization is needed. This difference has been
summarized earlier as \emph{mixins versus traits}. Some other notable
consequences of this difference are:
\begin{itemize}
\item `$\bbcomma$' is commutative, while `\&' is not;
\item `$\bbcomma$' supports mutual dependencies between traits, while `\&'
rejects such cycles.
\end{itemize}

\paragraph{Virtual classes and family polymorphism.}
Virtual classes~\citep{madsen1989virtual}, similarly to virtual methods, are
nested classes that can be overridden in subclasses. Virtual classes enable
family polymorphism~\citep{ernst2001family}, which can naturally solve the
expression problem~\citep{ernst2004expression}. The idea of virtual classes was
initially introduced in the BETA programming language~\citep{madsen1993object}
and later generalized in \texttt{gbeta}~\citep{ernst2000gbeta}.
CaesarJ~\citep{aracic2006overview}, an aspect-oriented programming language
based on Java, also supports virtual classes but does not allow cross-family
inheritance and dynamic inheritance. Newspeak~\citep{bracha2010modules}, a
descendant of Smalltalk, combines virtual classes and first-class modules (i.e.
instances of top-level classes) but is dynamically typed. The calculi
Jx~\citep{nystrom2004scalable}, J\&~\citep{nystrom2006j},
\emph{vc}~\citep{ernst2006virtual}, Tribe~\citep{clarke2007tribe}, and
.FJ~\citep{saito2008lightweight}, just to name a few, formalize virtual classes
with static inheritance but do not support dynamic inheritance.

\citet{zhang2017familia} propose the Familia programming language that unites
object-oriented polymorphism and parametric polymorphism by unifying interfaces
and type classes. In Familia, a mechanism of family polymorphism based on
\emph{nested inheritance}, similarly to Jx~\citep{nystrom2004scalable}, is also
deployed. During compilation, a \emph{linkage} is computed for every class,
which consists of a self-reference, a dispatch table, and the linkages of its
nested classes, among others. At the heart of the mechanism is \emph{further
binding}~\citep{madsen1993object}: rewiring self-references for nested classes.
Further binding is realized in Familia by linkage concatenation between
families. This process is similar to the nested trait composition in CP, but
there is a significant distinction in terms of separate compilation. CP
\emph{only} needs type information of the imported modules at compile time,
while Familia requires class linkages that contain some implementation details
of the imported modules (e.g. method definitions) and copy these details from
superclasses' linkages. In this sense, with linkages, Familia supports some
degree of separate compilation, but not to the same extent as the CP compiler
does. Moreover, since Familia does not support dynamic inheritance, their class
hierarchies are determined statically. In contrast, CP supports dynamic trait
composition, which brings extra flexibility.

More recently, \citet{kravchuk2024persimmon} propose \textsc{Persimmon}, a
functional programming language that features extensible variant types and
extensible pattern matching. CP also supports them via compositional interfaces
and method patterns. \textsc{Persimmon} additionally allows types to be members
of a family, relying on the support for \emph{relative path
types}~\citep{saito2008lightweight} in their core calculus. Internally,
\textsc{Persimmon} makes use of \emph{linkages} that are similar to those in
Familia. An important limitation of their current design is that modular type
checking and separate compilation are \emph{not} supported for multi-file
programs, while CP fully supports them.