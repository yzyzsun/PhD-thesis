\chapter{Conclusion and Future Work} \label{ch:conclusion}

\section{Future Work}

\subsection{Evaluating Document DSLs}

Recently, \citet{crichton2024core} proposed a core calculus for documents...

\subsection{Improving the CP Compiler}

\paragraph{Side effects in top-like terms.}
Our formalization of \lambdaiplus follows the distributive subtyping rules
proposed by \citet{barendregt1983filter}, which allows the top type $\top$ to be
a subtype of $\top \to \top$. Via transitivity and contravariance of functions,
we can show that $\top$ is a subtype of any function type returning $\top$:
\begin{mathpar}
\inferrule*[right=S-Trans]{\top \sub \top \to \top \\
                           \inferrule*[right=S-Fun]{A <: \top \\ \top <: \top}
                                                   {\top \to \top \sub A \to \top}}
                          {\top \sub A \to \top}
\end{mathpar}
This relation can be understood as a generalization of the distributive
subtyping rule for function types, where $n = 0$:
\begin{equation*}
(A \to B_1) \tand (A \to B_2) \tand \cdots \tand (A \to B_n)
\sub A \to (B_1 \tand B_2 \tand \cdots \tand B_n)
\end{equation*}
Since $\top$ represents the 0-ary intersection, the relation above is simplified
to $\top <: A \to \top$. The other direction $A \to \top <: \top$ also holds
because any type is a subtype of the maximal type $\top$. This means that $A \to
\top$ is equivalent to $\top$ and is called \emph{top-like} (i.e. $\rceil A \to
\top \lceil$) in \autoref{fig:disj}.

In our elaboration rules in \autoref{fig:source-typing}, all top-like terms are
treated as $\top$ and elaborated to an empty record (see
\rref{Ela-Top,Ela-TopAbs,Ela-TopRcd}). Moreover, the coercive subtyping
\rref{S-Top} in \autoref{fig:source-subtype} coerce a target term to an empty
record if $B$ is top-like in $A <: B$. As a result, side effects in top-like
terms are erased during elaboration, which is not desired in imperative
languages.

One reason we choose this design is to avoid the semantic ambiguity introduced
by the overlapping of disjoint types on the top-like part. For example...

\paragraph{Reconciliation between eagerness and laziness.}
As observed by \citet{fan2022direct}, trait instantiation may diverge in CP if a
call-by-value evaluation strategy is used. More than a score of years ago, a
similar observation was made by \citet{bruce1999comparing} in the context of
object encodings. \citeauthor{fan2022direct} fixed the divergence issue by
switching to a call-by-name evaluation strategy completely, which is inefficient
in practice. That is why we turn to a hybrid strategy: only self-annotated trait
fields are lazily evaluated, as shown in \autoref{fig:call-by-need}, and other
constructs are eagerly evaluated. Our hybrid approach is implemented in the CP
compiler and proves to be effective in practice, but it has not been formalized
or well studied. It is worthwhile to investigate formal semantics with the
hybrid evaluation strategy and see how it can be applied to other languages.

First, let us review why the trait instantiation does not terminate with the
call-by-value strategy. The example used for \autoref{fig:call-by-need}
corresponds to the following code using the fixpoint operator, and it reduces as
follows:
\begin{lstlisting}
    fix this. { x = this.y; y = 48 }
|-> { x = (fix this. { x = this.y; y = 48 }).y; y = 48 }
|-> { x = { x = (fix this. { x = this.y; y = 48 }).y; y = 48 }.y; y = 48 }
|-> ...
\end{lstlisting}
The evaluation diverges because the variable \lstinline{this} is evaluated
repeatedly, despite the fact that only \lstinline{this.y} is needed. Therefore,
we can conclude that the key to laziness is: do not evaluate anything until it
is needed...

A natural direction is to integrate the call-by-push-value
model~\citep{levy2012call}, which reconciles the call-by-value and call-by-name
strategies...

\paragraph{Representation of type indices.}
As discussed in \autoref{sec:index}, we use a shorter representation for
function types in our implementation, which only includes the return type.
However, there are some rare corner cases where this design causes trouble. Here
is an example:
\begin{lstlisting}
type A = (Int -> Int&Bool) -> Int;
type B = (Int -> Int&String) -> Int;
f (g: Int -> Int) = g 0;
f' = f : A&B;
f'' = f' : A;
f'' (\(x:Int) -> x,true)
\end{lstlisting}
Although the types of components of a merge are guaranteed to be disjoint, there
is no rule guaranteeing components of an intersection type to be disjoint.
Therefore, we can duplicate the function by casting \lstinline{f} to type
\lstinline{A&B}. Since \lstinline{A} and \lstinline{B} are not disjoint and have
the same return type, they would share the same type index
\lstinline{"func_int"}. We could not distinguish them by type indices, and the
field for \lstinline{f:B} would override that for \lstinline{f:A} in our
implementation. At first sight, it somehow makes sense because they are
essentially the same function so we do not need to keep two copies. However, the
trouble here is that their parameters have different type indices:
\lstinline{"func_(bool&int)"} for \lstinline{A}'s parameter and
\lstinline{"func_(int&string)"} for \lstinline{B}'s. This subtle difference
leads to slightly different compilation results for \lstinline{f:A} and
\lstinline{f:B}. The code above would go wrong as we do not distinguish them and
misuse \lstinline{f:B} as \lstinline{f:A}.

The fix can be easy. Our formalization in \autoref{ch:calculi} uses a more
precise representation for function types: both the parameter and return types
are included in the type indices. Nevertheless, the issue is really rare, and we
have never encountered such a case in benchmarks. For the considerations of
performance and code size, we still use the shorter type indices that only
include return types in our implementation. It is worthwhile to investigate the
trade-off between soundness and performance in the representation of type
indices. For example, we can use bit sets to represent intersection types.

\subsection{Mixing Named and Positional Arguments}

As shown in \autoref{tab:survey}, CP advocate distinctness between named and
positional arguments. An interesting direction worth exploring is to drop
distinctness. By this means, we can support omitting keywords if the order of
arguments is not changed from how they are defined. However, mixing positional
and named arguments can easily lead to ambiguity, especially when some arguments
are optional. An important first step is to find an appropriate restriction that
we should impose to mitigate the ambiguity problem.

One possibility is to follow the restriction in Python: ...
