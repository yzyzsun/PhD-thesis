\section{Introduction}

The $\lambda$-calculus, introduced by \citet{church1941calculi}, shows how to
model computation solely with function abstraction and application. For example,
natural numbers, boolean values, pairs, and lists, as well as various operations
on them, can be represented by higher-order functions via Church encoding. In
the $\lambda$-calculus, a function only has one parameter and can only be
applied to one argument. Many programming languages in the ML family inherit
this feature. If more than one argument is desired in those languages, we need
to create a sequence of functions, each with a single argument, and perform an
iteration of applications. This idea is called \emph{currying}. Currying brings
brevity to functional programming and naturally allows partial application, but
it usually limits the flexibility of function application. For example, we
cannot pass arguments in a different order nor omit some of them by providing
default values. Both demands are not rare in practical programming and can be
met in a language that supports \emph{named} and \emph{optional} arguments.
Named arguments also largely improve the readability of function calls. For
example, it is unclear which is the source and which is the destination in
\lstinline{copy(x, y)}, while \lstinline{copy(to: x, from: y)} is
self-explanatory.

\begin{figure}
\begin{subfigure}{0.5\textwidth}
\begin{lstlisting}[language={[3]Python}]
def exp(x, base=math.e):
  return base ** x

exp(10, 2)  #= exp(x=10, base=2) = 1024
exp(base=2, x=10)               #= 1024
exp(x=10)                       #= e^10

args = { "base": 2, "x": 10 }
exp(**args) #= exp(base=2, x=10) = 1024
\end{lstlisting}
\caption{The Python way.} \label{fig:python}
\end{subfigure}
\hfill
\begin{subfigure}{0.35\textwidth}
\begin{lstlisting}[language=Ruby]
def exp(x:, base: Math::E)
  base ** x
end
exp(10, 2) # ArgumentError!
exp(base: 2, x: 10) #= 1024
exp(x: 10)          #= e^10

args = { base: 2, x: 10 }
exp(**args)         #= 1024
\end{lstlisting}
\caption{The Ruby way.} \label{fig:ruby}
\end{subfigure}
\caption{Named arguments in Python and Ruby.} \label{fig:python-ruby}
\end{figure}

Named arguments are widely supported in mainstream programming languages, such
as Python, Ruby, OCaml, C\#, and Scala, just to name a few. The earliest
instance, to the best of our knowledge, is Smalltalk, where every method
argument \emph{must} be associated with a \emph{keyword} (i.e. an external
name). In other words, there are no positional arguments (i.e. arguments with no
keywords) in Smalltalk. The syntax of modern languages is usually less rigid, so
programmers can choose whether to attach keywords to arguments or not. There are
two ways to reconcile positional and named arguments. One way, employed by
Python and shown in \autoref{fig:python}, is to make parameter names in a
function definition as non-mandatory keywords. Thus, every argument can be
passed with or without keywords. As shown in the Python code,
\lstinline{exp(10, 2)} is equivalent to \lstinline{exp(x=10, base=2)}. To
reconcile the two forms in the same call, a restriction is imposed that all
positional arguments must appear to the left of named ones. The other way, shown
in \autoref{fig:ruby} and used in Ruby, is to strictly distinguish named
arguments from positional ones. When defining a Ruby function, a named parameter
should always end with a colon even if it does not have a default value. By this
means, they are syntactically distinct from positional parameters, and their
keywords cannot be omitted in a function call. The two kinds of arguments are
usually used in different scenarios: positional arguments are used when the
number of arguments is small and the order is clear, while named arguments are
used in more complex cases especially when settings or configurations are
involved.

More interestingly, named arguments are \emph{first-class} values in Python and
Ruby: they can be assigned to a variable. As shown at the bottom of
\autoref{fig:python-ruby}, the variable \lstinline{args} stores the two
arguments named \lstinline{base} and \lstinline{x}, and we can later pass it to
\lstinline{exp} by unpacking it with \lstinline{**} (sometimes called the splat
operator). In fact, \lstinline{args} is a dictionary in Python and similarly a
hash in Ruby. Thus, first-class named arguments can be manipulated and passed
around like standard data structures. This feature is widely used in Python and
Ruby.

\begin{table}
\caption{Named arguments with different design choices in different languages.} \label{tab:survey}
\centering
\small
\setlength{\tabcolsep}{1ex}
\begin{tabular}{*{11}{c}}
\toprule
                    &Smalltalk&  Python &  Ruby   &  Racket     &  OCaml  &  C\#    &  Scala  &  Dart   &  Swift  &  CP     \\
\midrule \midrule
Commutativity       & \Circle & \CIRCLE & \CIRCLE & \CIRCLE     & \CIRCLE & \CIRCLE & \CIRCLE & \CIRCLE & \Circle & \CIRCLE \\
Optionality         & \Circle & \CIRCLE & \CIRCLE & \CIRCLE     & \CIRCLE & \CIRCLE & \CIRCLE & \CIRCLE & \CIRCLE & \CIRCLE \\
Currying            & \Circle & \Circle & \Circle & \Circle     & \CIRCLE & \Circle & \Circle & \Circle & \Circle & \Circle \\
Distinctness        & \em n/a & \Circle & \CIRCLE & \CIRCLE     & \CIRCLE & \Circle & \Circle & \CIRCLE & \CIRCLE & \CIRCLE \\
First-class value   & \Circle & \CIRCLE & \CIRCLE & \LEFTcircle & \Circle & \Circle & \Circle & \Circle & \Circle & \CIRCLE \\
\midrule
Static typing       & \Circle & \Circle & \Circle & \Circle     & \CIRCLE & \CIRCLE & \CIRCLE & \CIRCLE & \CIRCLE & \CIRCLE \\
Soundness proof     & \Circle & \Circle & \Circle & \Circle     & \CIRCLE & \Circle & \Circle & \Circle & \Circle & \CIRCLE \\
\bottomrule
\end{tabular}
\vskip 1ex
{\footnotesize
  \emph{n/a}:  Smalltalk does not support positional arguments at all. \\
  \LEFTcircle: Racket's support for first-class named arguments is limited and forbids commutativity.
}
\end{table}

Including the distinctness and first-class values illustrated above, we have
identified five important design choices found in existing languages that
support named arguments:
\begin{enumerate}
\item \emph{Commutativity}: whether the order of (actual) arguments can be
      different from that of (formal) parameters originally declared.
\item \emph{Optionality}: whether some arguments can be omitted in a function
      call if their default values are predefined.
\item \emph{Currying}: whether a function that takes more than one argument is
      always converted into a chain of functions that each take a single
      argument.
\item \emph{Distinctness}: whether named arguments are distinct from positional
      ones in how they are defined and passed.
\item \emph{First-class value}: whether named arguments are first-class values.
\end{enumerate}
As shown in \autoref{tab:survey}, the first two properties hold for most
mainstream programming languages, with Smalltalk and Swift being two exceptions.
Commutativity and optionality are so useful that we believe they should not be
compromised. Concerning the third point, OCaml is the only language that manages
to reconcile currying with commutativity, though at the cost of introducing a
very complicated core calculus. We agree that currying is very useful when we
use normal positional arguments, but we argue here that currying can be
temporarily dropped when we use named arguments because the most common use case
for named arguments is to represent a whole chunk of parameters like settings or
configurations. The fourth design, distinctness, is endorsed by Ruby, Racket,
OCaml, Dart, and Swift. It improves the readability of call sites to enforce
keywords whenever arguments are defined to be named. We advocate distinctness in
this work also because it simplifies the language design and allows us to focus
on more important topics, especially type safety with first-class named
arguments.

Although named arguments are ubiquitous, they have not attracted enough
attention in the research of programming languages. Among the few related
papers, the work by \citet{garrigue1994label} formalizes a label-selective
$\lambda$-calculus and eventually apply it to OCaml~\citep{garrigue2001labeled}.
Another work by \citet{rytz2010named} discusses the design of named and optional
arguments in Scala, but it mainly focuses on practical aspects. The core
features of Scala are formalized in a family of DOT calculi, but named arguments
are never included. The support for named arguments is implemented as macros in
Racket~\citep{flatt2009keyword}. So their extension is more like userland
syntactic sugar and requires no changes to the core compiler. Haskell does not
support named arguments natively, but the paradigm of \emph{named arguments as
records} is folklore. We will discuss OCaml, Scala, Racket, and Haskell in
detail in \autoref{sec:related}. In short, named arguments are implemented in an
ad-hoc manner and are not well founded from a type-theoretic perspective in most
languages, especially object-oriented ones. Only OCaml and CP provide
\emph{soundness proofs} for the feature of named arguments.

An important issue that has not been explored in the literature is the
interaction between subtyping and first-class named arguments. A naive design
can easily lead to a type-safety issue. We will show in \autoref{sec:mypy} that
the most widely used optional type checker for Python, mypy~\citep{mypy}, fails
to detect a type-unsafe use of first-class named arguments. The same issue also
exists in Ruby with Steep~\citep{steep} or Sorbet~\citep{sorbet}. It arises from
subtyping hiding some arguments from their static type and bypassing the type
checking for optional arguments. As a result, an optional argument may have an
unexpected type at run time, which leads to a runtime error.

In this chapter, we present a type-safe foundation for named and optional
arguments. At the heart of our approach is the translation into a core calculus
called \lambdaiu, which features \emph{intersection and union
types}~\citep{barbanera1995intersection,frisch2008semantic,dunfield2014elaborating}.
Our approach supports first-class named arguments like Python and Ruby, but the
type-safety issue is addressed by us. The \lambdaiu calculus has been shown to
be type-sound~\citep{rehman2023blend}, and we show that our translation from our
source language into \lambdaiu is type-safe. Thus, we establish the type safety
of our approach.
