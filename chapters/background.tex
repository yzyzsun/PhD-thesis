\chapter{Disjoint Intersection Types and First-Class Traits} \label{ch:background}

This part provides a brief introduction to the background of this thesis. We
first introduce intersection and union types, merging, disjointness, and traits
in this chapter, laying the type-theoretic foundation for CP. Next in
\autoref{ch:cp}, we will give a crash course in the CP language in a friendly manner.

\newcommand{\tand}{\,\&\,}
\newcommand{\tor}{\,|\,}
\newcommand{\sub}{\;<:\;}

\section{Intersection Types}

Generics or parametric polymorphism provides a mechanism for code reuse by
assigning infinite possibilities of types to a single definition. This kind of
polymorphism is \emph{uniform} in that all possibilities behave identically
despite different types instantiated.

In contrast, intersection types allows explicitly enumerating all possible types
that a single definition can have. In recently developed
theories~\citep{dunfield2014elaborating,castagna2023programming}, intersection
types are closely related to overloading or \emph{ad-hoc} polymorphism. For
example, the type of the addition operator \lstinline{(+)} for both integers and
floating-point numbers can be written as:
\begin{equation*}
(\mathbf{Int} \times \mathbf{Int} \to \mathbf{Int}) \quad\&\quad
(\mathbf{Double} \times \mathbf{Double} \to \mathbf{Double})
\end{equation*}
Note that the two versions of addition do not necessarily have the same
behavior. Semantic subtyping~\citep{frisch2008semantic} provides a
set-theoretic foundation for overloaded functions typed as intersections and is
employed in the Elixir type system~\citep{castagna2023design}.

However, this connection is not the original motivation for intersection types
(the connection did not even hold!) if we look back at the history from
1970s~\citep{bono2020tale}. Instead, functions with intersection types had
uniform behavior and could be regarded as a finite portion of parametrically
polymorphic functions. This notion is called \emph{coherent} overloading or
\emph{finitary} polymorphism~\citep{pierce1991programming}, which is a limited
form of ad-hoc polymorphism. Nevertheless, intersection types are useful for
some functions that cannot be typed in the simply typed $\lambda$-calculus (\`a
la Curry), such as the $\Omega$-combinator ($\lambda x.\ x\ x$). The
$\Omega$-combinator can be typed using the two elimination rules for
intersection types in \autoref{fig:typ-and} (\textsc{T-Abs} and \textsc{T-App}
are standard rules for functions and thus omitted):\footnote{In this thesis, we
assume that $\&$ has precedence over $\to$. For example, $(A \to B) \tand A \to B$
is parsed as $((A \to B) \tand A) \to B$.}
\begin{mathpar}
\inferrule*[right=T-Abs]{\inferrule*[right=T-App]{\inferrule[T-AndElimL]{\dots}
                                                                        {x : (A \to B) \tand A \;\vdash\; x : A \to B} \\
                                                  \inferrule[T-AndElimR]{\dots}
                                                                        {x : (A \to B) \tand A \;\vdash\; x : A}}
                                                 {x : (A \to B) \tand A \;\vdash\; x\ x : B}}
                        {\cdot \vdash \lambda x.\ x\ x : (A \to B) \tand A \to B}
\end{mathpar}
Intersection types in this setting are typically used to characterize the
termination properties of $\lambda$-terms~\citep{dezani2005compositional}.
Because of the correspondence with termination, type inference with intersection
types is \emph{not} decidable; neither is type
inhabitation~\citep{urzyczyn1999emptiness}.

\begin{figure}
\begin{mathpar}
\inferrule[T-AndIntro]{e : A \\ e : B}
                      {e : A \tand B}

\inferrule[T-AndElimL]{e : A \tand B}
                      {e : A}

\inferrule[T-AndElimR]{e : A \tand B}
                      {e : B}
\end{mathpar}
\caption{Typing rules for intersection types.} \label{fig:typ-and}
\end{figure}

\begin{figure}
\begin{mathpar}
\inferrule[S-AndL]{}{A \tand B \sub A}

\inferrule[S-AndR]{}{A \tand B \sub B}

\inferrule[S-And]{A <: B \\ A <: C}
                 {A \sub B \tand C}
\end{mathpar}
\caption{Subtyping rules for intersection types.} \label{fig:sub-and}
\end{figure}

In light of the analogy with set intersection, it is tempting to add subtyping
to an intersection type system. \autoref{fig:sub-and} shows three subtyping
rules for intersection types, which are naturally induced by set inclusion.
Besides the subtyping rules, we also need to add the standard subsumption rule
to the typing rules in \autoref{fig:typ-and}:
\begin{mathpar}
\inferrule[T-Sub]{e : A \\ A <: B}
                 {e : B}
\end{mathpar}
With \textsc{T-Sub}, the previous elimination rules \textsc{T-AndElimL} and
\textsc{T-AndElimR} can now be derived from the subtyping rules \textsc{S-AndL}
and \textsc{S-AndR}.

\paragraph{Distributive subtyping.}
It is well known that $\lambda$-calculi have a close relationship with logical
systems~\citep{wadler2015propositions}: the Curry-Howard
isomorphism states that propositions are types and proofs are terms. An
interesting property in logic is that implication is left-distributive over
conjunction, as shown in the following equivalence:
\begin{equation*}
A \to B \land C \quad\Longleftrightarrow\quad (A \to B) \land (A \to C)
\end{equation*}
Since implication corresponds to function types and conjunction corresponds to
intersection types in our setting, the distributive property can be interpreted
as two subtyping rules:
\begin{align}
          A \to B \tand C \quad <:& \quad (A \to B) \tand (A \to C) \label{eq:dist-1} \\
(A \to B) \tand (A \to C) \quad <:& \quad A \to B \tand C \label{eq:dist-2}
\end{align}
\autoref{eq:dist-1} can be derived from the aforementioned subtyping rules for
intersection types:
\begin{mathpar}
\small
\inferrule*[right=S-And]
  {\inferrule*[left=S-Fun]{A <: A \\
                           \inferrule*[right=S-AndL]{ }
                                                    {B \tand C \sub B}}
                          {A \to B \tand C \sub A \to B} \\
   \inferrule*[right=S-Fun]{A <: A \\
                            \inferrule*[right=S-AndR]{ }
                                                     {B \tand C \sub C}}
                           {A \to B \tand C \sub A \to C}}
  {A \to B \tand C \sub (A \to B) \tand (A \to C)}
\end{mathpar}
However, \autoref{eq:dist-2} is not derivable so far. Instead, it is a key rule
of distributive subtyping, originating from the BCD type assignment
system~\citep{barendregt1983filter}. An interesting consequence of adding
\autoref{eq:dist-2} is that $(A \to B) \tand (C \to D)$ is a subtype of $A \tand
C \to B \tand D$. This can be derived as follows via transitivity:
\begin{mathpar}
\text{(I)}
\footnotesize
\inferrule*[right=S-And]{\inferrule*[left=S-AndL]{\inferrule*[left=S-Fun]{A \tand C <: A \\ B <: B}
                                                                         {A \to B \sub A \tand C \to B}}
                                                 {(A \to B) \tand (C \to D) <: A \tand C \to B} \\
                         \inferrule*[right=S-AndR]{\inferrule*[right=S-Fun]{A \tand C <: C \\ D <: D}
                                                                           {C \to D \sub A \tand C \to D}}
                                                  {(A \to B) \tand (C \to D) <: A \tand C \to D}}
                        {(A \to B) \tand (C \to D) \sub (A \tand C \to B) \tand (A \tand C \to D)}

\normalsize
\inferrule*[right=S-Trans]
  {\text{(I)}  \quad (A \to B) \tand (C \to D) \sub (A \tand C \to B) \tand (A \tand C \to D) \\
   \text{(II)} \quad (A \tand C \to B) \tand (A \tand C \to D) \sub A \tand C \to B \tand D}
  {(A \to B) \tand (C \to D) \sub A \tand C \to B \tand D}
\end{mathpar}
Judgment (I) is derivable as shown above, and Judgment (II) directly follows
\autoref{eq:dist-2}. In short, distributive subtyping allows using an
intersection of two function types as if it is a function type with the two
parameter types intersected and the two return types intersected. This form of
distributivity can be extended to record types and reveals the essence of nested
composition~\citep{bi2018essence}. As a result, family
polymorphism~\citep{ernst2001family} can be achieved, providing an elegant
solution to the expression
problem~\citep{wadler1998expression,ernst2004expression}. A detailed discussion
about family polymorphism in CP can be found in \autoref{sec:ep}.

\paragraph{Interaction with mutable references.}
Although distributive intersection subtyping is powerful, it poses a significant
challenge for type safety when mutable references are involved. In fact,
intersection subtyping alone without distributivity is already problematic.
\citet{davies2000intersection} illustrated the problem with the following
example, assuming that \lstinline[morekeywords=Pos]{Pos} (positive numbers) is a
subtype of \lstinline[morekeywords=Nat]{Nat} (natural numbers):
\begin{lstlisting}[morekeywords={Nat,Pos}]
let x = ref 48 : Ref Nat & Ref Pos in
let y = (x := 0) in    -- x is used as Ref Nat
let z = !x in z : Pos  -- x is used as Ref Pos
\end{lstlisting}
The code is well-typed but could cause a runtime error, because \lstinline{z} is
expected to be positive while it is actually a non-positive natural number
\lstinline{0}. The solution proposed by \citeauthor{davies2000intersection} is a
variant of \emph{value restriction}~\citep{wright1995simple}, which requires
that the introduction rule of intersection types (i.e. \textsc{T-AndIntro} in
\autoref{fig:typ-and}) only applies to values. By this means, \lstinline{ref 1}
cannot be typed as \lstinline[morekeywords={Nat,Pos}]{Ref Nat & Ref Pos} since
it is not a value. However, this solution does not work well with distributive
subtyping. \citeauthor{davies2000intersection} showed a similar example to the
previous one:
\begin{lstlisting}[morekeywords={Nat,Pos}]
let x = (\() -> ref 48) () : Ref Nat & Ref Pos in
let y = (x := 0) in    -- x is used as Ref Nat
let z = !x in z : Pos  -- x is used as Ref Pos
\end{lstlisting}
Since the anonymous function \lstinline{(\() -> ref 48)} is a value,
it can be typed as an intersection type
\lstinline[morekeywords={Nat,Pos}]{(() -> Ref Nat) & (() -> Ref Pos)}.
Besides, this function can be used as
\lstinline[morekeywords={Nat,Pos}]{() -> Ref Nat & Ref Pos}
via distributive subtyping. Applying it to \lstinline{()} yields almost the same
code as the previous example. To address this type-safety issue,
\citeauthor{davies2000intersection} have to drop the distributivity rule.

Recently, \citet{ye2024imperative} proposed a simpler solution to this problem
based on \emph{bidirectional typing}~\citep{pierce2000local,dunfield2021bidirectional}.
Bidirectional type system divides traditional type assignment ($e : A$) into two
modes: type checking ($e \Leftarrow A$) and type inference ($e \Rightarrow a$).
The key idea of \citeauthor{ye2024imperative}'s solution is that the type of the
value stored in a reference can only be inferred but not checked:
\begin{mathpar}
\inferrule[T-Ref-Before]{e : A}
                 {\mathbf{ref}\ e : \mathbf{Ref}\ A}

\inferrule[T-Ref-After]{e \Rightarrow A}
                 {\mathbf{ref}\ e \Rightarrow \mathbf{Ref}\ A}
\end{mathpar}
With the rule \textsc{T-Ref-After}, \lstinline{ref 48} can only have type
\lstinline[morekeywords=Pos]{Ref Pos} because \lstinline{48} is inferred to have
type \lstinline[morekeywords=Pos]{Pos}. Moreover, \lstinline{ref 48} cannot be
checked against \lstinline[morekeywords=Nat]{Ref Nat} since reference types are
invariant. As a result, the previous two examples cannot type-check in this
bidirectional type system. By this means, the type-safety issue is resolved
without enforcing the value restriction on intersection introduction or
sacrificing distributive intersection subtyping. A final note is that, to have a
reference of type \lstinline[morekeywords=Nat]{Ref Nat} with the initial value
\lstinline{48}, one can write \lstinline[morekeywords=Nat]{ref (48 : Nat)}
instead.

\section{Merging and Disjointness}

In last section, we have mentioned that intersection types correspond to logical
conjunction when introducing distributive subtyping. However, this
correspondence does not apply to the original intersection type systems,
including the BCD system~\citep{barendregt1983filter}. \citet{hindley1983coppo}
gave a counterexample showing that some uninhabited intersection types are
provable in most logics. The root cause is that the introduction rule
(\textsc{T-AndIntro} in \autoref{fig:typ-and}) requires the two premises to have
the same term $e$, which is not the case in logical conjunction.
\citet{dunfield2014elaborating} proposed to add a merge construct ($e_1 \bbcomma
e_2$), where the comma\footnote{We use a single comma in this thesis instead of
double commas used by \citet{dunfield2014elaborating}. Although it is styled as
$\bbcomma$ in formulas, it is written as a plain comma \lstinline{,} in CP
code.} $(\bbcomma)$ is called the merge operator, to close the gap between
intersection types and logical conjunction. Two typing rules (\textsc{T-MergeL}
and \textsc{T-MergeR}) are added by \citeauthor{dunfield2014elaborating}, and
together with \textsc{T-AndIntro}, we can show that $e_1 \bbcomma e_2$ has type
$A \tand B$ if $e_1$ has type $A$ and $e_2$ has type $B$:
\begin{mathpar}
\inferrule*[lab=T-MergeL]{e_1 : A}
                         {e_1 \bbcomma e_2 : A}

\inferrule*[lab=T-MergeR]{e_2 : B}
                         {e_1 \bbcomma e_2 : B}

\inferrule*[right=T-AndIntro]{\inferrule*[left=T-MergeL]{e_1 : A}
                                                        {e_1 \bbcomma e_2 : A} \\
                              \inferrule*[right=T-MergeR]{e_2 : B}
                                                         {e_1 \bbcomma e_2 : B}}
                             {e_1 \bbcomma e_2 : A \tand B}
\end{mathpar}
The merge operator can be traced back to Forsythe, an ALGOL-like language
designed by \citet{reynolds1997design}. Merging in Forsythe is biased to avoid
semantic ambiguity. For example, if $f$ is a function, $e \bbcomma f$ will
override all function components in $e$. Consequently, merging in Forsythe
cannot be used to encode function overloading, though intersections of function
types are supported and a limited form of \emph{coherent} overloading can be
achieved. A significant outcome of merging is the ability to encode multi-field
records, where width and permutation subtyping are naturally supported via
intersection subtyping.

Contrary to Forsythe, merging in \citeauthor{dunfield2014elaborating}'s system
has the nice property of \emph{commutativity}: the order of merging does not
matter. Since there is no bias in favor of either side, semantic ambiguity
becomes an important problem. For example, applying
$(\lambda x.\ x + 1)\bbcomma(\lambda x.\ x + 2)$ to $0$ can yield either $1$ or
$2$. To address this issue, \citet{oliveira2016disjoint} require that the two
terms to be merged must have \emph{disjoint} types. A typical typing rule for
disjoint merges is as follows:
\begin{mathpar}
\inferrule[T-Merge-Disjoint]{e_1 \Rightarrow A \\ e_2 \Rightarrow B \\ A * B}
                            {e_1 \bbcomma e_2 \Rightarrow A \tand B}
\end{mathpar}
The premise $A * B$ denotes that $A$ and $B$ are disjoint. The aforementioned
example is now \emph{not} well-typed because $\mathbf{Int} \to \mathbf{Int}$ is
\emph{not} disjoint with $\mathbf{Int} \to \mathbf{Int}$ itself. More generally,
$A \to \mathbf{Int}$ is not disjoint with $B \to \mathbf{Int}$ no matter what
$A$ and $B$ are, because they have an overlapping part $A \tand B \to
\mathbf{Int}$ (or technically speaking, a common subtype). For example, consider
this overloaded function:
\begin{equation*}
(\lambda x.\ x + 1)\bbcomma(\lambda x.\ \mathbf{if}\ x\ \mathbf{then}\ 1\ \mathbf{else}\ 0)
: (\mathbf{Int} \to \mathbf{Int}) \tand (\mathbf{Bool} \to \mathbf{Int})
\end{equation*}
Applying it to $1 \bbcomma \mathbf{false}$ can yield $2$ if we choose the
left-hand side or $0$ if we choose the right-hand side. In contrast,
\emph{overloading by return type} is allowed in
\citeauthor{oliveira2016disjoint}'s system with disjoint intersection types.
For example, consider this exotically overloaded function:
\begin{equation*}
f =\; (\lambda x.\ x + 1)\bbcomma(\lambda x.\ x > 0)
: (\mathbf{Int} \to \mathbf{Int}) \tand (\mathbf{Int} \to \mathbf{Bool})
\end{equation*}
Applying it to an integer never causes ambiguity when the expected type is
known. For example, $f\ 1 + 2$ yields $4$, while $\mathbf{not}\ (f\ 1)$ yields
$\mathbf{false}$.

Another application of disjoint merges is to model record concatenation: if
specializing the operands to records, the merge operator is essentially
concatenating two records. As shown previously in \textsc{T-Merge-Disjoint}, the
two records must have disjoint types to avoid ambiguity. For example,
\lstinline|{ x: Int }| and \lstinline|{ x: Int }| itself are not disjoint, so
the merge \lstinline{r,s} is rejected in the following code:
\begin{lstlisting}
let merge (r: { x: Int }) (s: { x: Int }) = r,s in  -- Type Error!
merge { x = 1 } { x = 3 }  --> { x = ? }
\end{lstlisting}

\paragraph{Interaction between merging and subtyping.}
If we further consider subtyping, the merge operator is still problematic, and
disjointness alone is not sufficient to prevent ambiguity. For example, consider
the following code:
\begin{lstlisting}
let merge (r: { x: Int }) (s: { y: Int }) = r,s in
merge { x = 1; y = 2 } { x = 3; y = 4 }  --> { x = ?; y = ? }
\end{lstlisting}
Note that we change the type of \lstinline{s} from \lstinline|{ x: Int }| to
\lstinline|{ y: Int }|. Although the type of \lstinline{s} is now disjoint with
that of \lstinline{r}, we can pass terms of their subtypes to \lstinline{merge}.
In this case, \lstinline{r} has an extra field \lstinline{y} and \lstinline{s}
has an extra \lstinline{x}. Now the issue of ambiguity occurs again.

If we look at the function \lstinline{merge} statically, we would expect that
the field \lstinline{x} is from \lstinline{r} and \lstinline{y} from
\lstinline{s}. Therefore, the most reasonable result for the code above is
\lstinline|{ x = 1; y = 4 }|. However, there is no naive way to implement the
merge operator to achieve this result. \emph{Neither} left-biased \emph{nor}
right-biased overriding is able to handle this case. Furthermore, selecting
other fields at run time can actually lead to type unsoundness. For example,
consider a variant of the previous merge:
\begin{lstlisting}
merge { x = 1; y = "Hi" } { x = "Bye"; y = 4 }  --> { x = ?; y = ? }
\end{lstlisting}
Statically, the function is expected to compute a value of type
\lstinline|{ x: Int; y: Int }|, but fields of type \lstinline{String} could be
selected. The interaction between record concatenation and subtyping is
inherently difficult and was the reason preventing
\citet{cardelli1991operations} from choosing concatenation as the primitive
operator in their calculus.

The solution found in the line of work by \citet{oliveira2016disjoint} is to
employ a \emph{coercive} semantics of subtyping, where a subtyping relationship
$A <: B$ implies a coercion function of type $A \to B$. This solution picks the
field \lstinline{x} from \lstinline{s} and \lstinline{y} from \lstinline{r}, by
being aware of the static types when selecting components. In the previous
example, during the function application, \lstinline{r} is coerced to a
single-field record \lstinline|{ x = 1 }|, corresponding to the parameter type
\lstinline|{ x: Int }|. A similar coercion is inserted for \lstinline{s} as
well, coercing it to \lstinline|{ y = 4 }|. Then the merge operator simply
concatenates \lstinline|{ x = 1 }| and \lstinline|{ y = 4 }|, which has no
ambiguity. Thus, a combination of disjointness and a coercive approach to
subtyping is able to eliminate the ambiguity introduced by an unrestricted merge
operator.

\paragraph{Disjoint polymorphism and disjointness constraints.}
In the previous example, some type information about the records being merged is
lost. But we may wish to preserve other fields in the records that do not create
ambiguity. This can be achieved by merging polymorphic terms, whose static types
are not fully known. For example, consider a variant of the previous example:
\begin{lstlisting}
let mergeSub (A <: { x: Int }) (B <: { y: Int }) (r: A) (s: B) = r,s in
mergeSub @{ x: Int; y: Int } @{ x: Int; y: Int }  -- explicit type application
          { x = 1;  y = 2  }  { x = 3;  y = 4  }
\end{lstlisting}
The code is written in pseudo-CP, where \lstinline{<:} denotes the upper bound
of a type parameter. In this example, \lstinline{A} and \lstinline{B} are
declared to be subtypes of \lstinline|{ x: Int }| and \lstinline|{ y: Int }|
respectively. Since CP does not yet support implicit polymorphism, both type
parameters are instantiated explicitly on the second line. Like in Haskell,
\lstinline|@| is the prefix of type arguments in CP. With bounded
quantification~\citep{cardelli1985understanding}, we cannot guarantee the
disjointness of \lstinline{A} and \lstinline{B}, so the issue of ambiguity comes
back again. This issue can be solved by \emph{disjoint
quantification}~\citep{alpuim2017disjoint}:
\begin{lstlisting}
let mergeDis (A * { y: Int }) (B * A & { x: Int })  -- B * A and B * { x: Int }
             (r: A & { x: Int }) (s: B & { y: Int }) = r,s in
mergeDis @{y: Int} @{x: Int} { x = 1; y = 2 } { x = 3; y = 4 }  -- Type Error!
mergeDis @Top @Top { x = 1 } { y = 4 }  --> { x = 1; y = 4 }
\end{lstlisting}
Note that the type of \lstinline{r} is now \lstinline|A & { x: Int }| instead of
\lstinline{A}. This is how we usually translate subtype-bounded quantification
to disjoint quantification~\citep{xie2020row}. The type parameter \lstinline{A}
is declared to be disjoint with \lstinline|{ y: Int }| to avoid the overlap, and
\lstinline{B} is disjoint with \lstinline|{ x: Int }| similarly. Another
important constraint here is the disjointness of \lstinline{A} and
\lstinline{B}, ensuring that other fields will never conflict as well. For
example, consider a third field of type \lstinline|{ z: Int }|:
\begin{lstlisting}
mergeDis @{z: Int} @{z: Int} { x = 1; z = 5 } { y = 4; z = 6 }  -- Type Error!
mergeDis @Top @{z: Int} { x = 1 } { y = 4; z = 6 }  --> { x = 1; y = 4; z = 6 }
\end{lstlisting}
The first line of code fails to type-check because \lstinline{A} and
\lstinline{B} are not disjoint and both contain a field of type
\lstinline|{ z: Int }|. The second line resolves the conflict, and we can access
all three fields after merging. As we shall see in \autoref{sec:dynamic}, the
ability to express absence of certain fields is important for CP to safely
handle dynamic inheritance.

\section{Union Types}

Union types are the dual concept of intersection types. While a value of the
intersection type $A \tand B$ can be assigned both $A$ and $B$, a value of the
union type $A \tor B$ can be assigned either $A$ or $B$. Although tagged unions
are more common in functional languages, as seen in algebraic data types, we
focus on \emph{un}tagged unions in this thesis.

In classic $\lambda$-calculi with union types~\citep{barbanera1995intersection},
typical typing rules for union types are as follows:
\begin{mathpar}
\inferrule[T-OrIntroL]{\Gamma \vdash e : A}
                      {\Gamma \vdash e : A \tor B}

\inferrule[T-OrIntroR]{\Gamma \vdash e : B}
                      {\Gamma \vdash e : A \tor B}

\inferrule[T-OrElim]{\Gamma \vdash e' : A \tor B \\
                     \Gamma, x : A \,\vdash\, e : C \\
                     \Gamma, x : B \,\vdash\, e : C}
                    {\Gamma \vdash [ e' / x ] e : C}
\end{mathpar}
The union introduction rules (\textsc{T-OrIntroL} and \textsc{T-OrIntroR}) are
straightforward and dual to the intersection elimination rules
(\textsc{T-AndElimL} and \textsc{T-AndElimR}). In contrast, the union
elimination rule (\textsc{T-OrElim}) is more intriguing. It basically states
that, if $[ e' / x ] e$ can be typed as $C$ no matter whether we assume $e'$
evaluates to a value of type $A$ or $B$, then it is safe to type $[ e' / x ] e$
as $C$ when $e'$ has type $A \tor B$. An intermediate variable $x$ is used
because a typing context can only associate a type with a variable rather than
an expression.

A direct application of union types is to model nullable
types~\citep{nieto2020scala}, which can be represented as $A \tor
\mathbf{Null}$. To make union or nullable types more useful, \emph{occurrence
typing}~\citep{castagna2022revisiting}, also known as \emph{flow-sensitive
typing}, is usually employed. Without occurrence typing, the following code
cannot type-check:
\begin{lstlisting}
double (x: Int|Null) = if x == null then 0 else 2*x;
\end{lstlisting}
The reason is that \lstinline{2*x} is well-typed only if \lstinline{x} has type
\lstinline{Int}, but \lstinline{x} actually has type \lstinline{Int|Null} as
declared. With occurrence typing, the type of \lstinline{x} is refined to
\lstinline{Int} in the \lstinline{else}-clause (and is refined to
\lstinline{Null} in the \lstinline{then}-clause), so the code is well-typed.
\citeauthor{castagna2022revisiting} generalize the \lstinline{null}-test to
\emph{type-cases} and develop a theoretic framework to refine the type of
expressions occurring in type-cases. CP follows a simpler treatment proposed by
\citet{rehman2023blend} and represents a type-case as
\lstinline{switch e0 as x case A => e1 case B => e2}. For example, the previous
function can be rewritten as:
\begin{lstlisting}
double (x: Int|Null) = switch x as x case Null => 0 case Int => 2*x;
\end{lstlisting}
Now union types have an explicit elimination form, \textsc{T-OrElim} can be
replaced by \textsc{T-Switch}:
\begin{mathpar}
\inferrule[T-Switch]{\Gamma \vdash e_0 : A \tor B \\
                     \Gamma, x : A \,\vdash\, e_1 : C \\
                     \Gamma, x : B \,\vdash\, e_2 : C}
                    {\Gamma \vdash \mathbf{switch}\ e_0\ \mathbf{as}\ x\ \mathbf{case}\ A \Rightarrow e_1\
                                                                         \mathbf{case}\ B \Rightarrow e_2 : C}
\end{mathpar}

\paragraph{Distributive subtyping.}
Again, if we consider the logical counterparts of intersection and union types,
a noticeable property is that implication is right-\emph{anti}distributive over
disjunction:
\begin{equation*}
A \lor B \to C \quad\Longleftrightarrow\quad (A \to C) \land (B \to C)
\end{equation*}
Note that the disjunction on the left-hand side turns into a conjunction on the
right-hand side. This property comes from the duality between conjunction and
disjunction in De Morgan's laws. In fact, this duality has been exploited in the
design of Forsythe~\citep{reynolds1997design}, where
$\lambda x:\mathbf{Int}\tor\mathbf{Double}.\ x > 0$ has type
$(\mathbf{Int}\to\mathbf{Bool})\tand(\mathbf{Double}\to\mathbf{Bool})$. Although
union types are not supported by Forsythe, the notation of the alternative
operator ($\tor$) implies that the parameter type is essentially a union. In
other words, the following subtyping relation holds (assuming $A = \mathbf{Int}$,
$B = \mathbf{Double}$, and $C = \mathbf{Bool}$):
\begin{equation}
A \tor B \to C  \quad <: \quad  (A \to C) \tand (B \to C) \label{eq:dist-3}
\end{equation}
Similar to \autoref{eq:dist-1}, \autoref{eq:dist-3} can also be derived from
standard subtyping rules for intersection and union types:
\begin{mathpar}
\inferrule*[right=S-And]
  {\inferrule*[left=S-Fun]{\inferrule*[left=S-OrL]{ }
                                                  {A \sub A \tor B} \\
                           C <: C}
                          {A \tor B \to C \sub A \to C} \\
   \inferrule*[right=S-Fun]{\inferrule*[left=S-OrR]{ }
                                                   {B \sub A \tor B} \\
                            C <: C}
                           {A \tor B \to C \sub B \to C}}
  {A \tor B \to C \sub (A \to C) \tand (B \to C)}
\end{mathpar}
However, its converse (\autoref{eq:dist-4}) is not derivable and is added as an
axiom, for example, by \citet{barbanera1995intersection} and \citet{rehman2023blend}:
\begin{equation}
  (A \to C) \tand (B \to C)  \quad <: \quad  A \tor B \to C \label{eq:dist-4}
\end{equation}
A final note is that intersections and unions can usually distribute over each other:
\begin{align}
(A \tor B) \tand C  \quad          <:& \quad  (A \tand C) \tor (B \tand C) \label{eq:dist-and} \\
(A \tor C) \tand (B \tor C)  \quad <:& \quad  (A \tand B) \tor C \label{eq:dist-or}
\end{align}
The converses of \autoref{eq:dist-and} and \autoref{eq:dist-or} are derivable so
they need not be axioms. The algorithm for subtyping becomes non-trivial with
the four distributivity axioms. The one employed in CP is proposed by
\citet{huang2021distributing}, having the advantage of not requiring types to be
normalized and being relatively simple to implement.

\section{Traits and Mechanisms for Code Reuse}

Traits are an overloaded term in the literature, referring to various mechanisms
for code reuse. The earliest use of the term ``traits'' in the context of
object-oriented programming can be traced back to Mesa~\citep{curry1982traits},
an ALGOL-like language developed by Xerox PARC. That traits model of subclassing
is used to handle multiple inheritance in Xerox Star workstation software.
Later, \citet{ungar1991organizing} reused the term in the context of \self, a
dynamically typed prototype-based language, which is also developed by Xerox
PARC. There is no class in a prototype-based language, and objects inherits
directly from other objects. Traits in \self are parent objects that are shared
among multiple objects, playing a similar role to traditional classes.

\citet{ducasse2006traits} reconceptualized traits as a reuse mechanism for
class-based languages. In their model, traits are purely units of reuse, while
classes are generators of objects. Not only can a trait \emph{provide} methods,
but it can also \emph{require} methods that are used but not implemented. A
class can be composed of multiple traits while resolving conflicts and
implementing the required methods in the meantime. An important property of
traits is that the composition order is irrelevant, in contrast to
order-sensitive mixins~\citep{bracha1990mixin}. Besides the sum operator ($+$)
for trait composition, exclusion ($-$), aliasing ($\to$), and overriding
($\rhd$) are also supported in \citeauthor{ducasse2006traits}'s traits model to
resolve conflicts.

The term ``traits'' has recently been popularized by Scala and Rust, though it
refers to different concepts from all the mechanisms mentioned above. Scala's
traits correspond to Java 8's interfaces, which may contain abstract members and
default implementations thereof. The semantics of trait composition in Scala is
more like mixins despite the name. Rust's traits are similar to type classes in
Haskell, providing a less ad-hoc mechanism for ad-hoc polymorphism. In short,
traits in Scala and Rust are significantly different from traits in CP, which
follows the principle of the traits mechanism proposed by
\citeauthor{ducasse2006traits}

\paragraph{Dependency injection.}

\paragraph{First-class traits.}
CP-flavored traits are first-class~\citep{bi2018typed} in the sense that they
can be passed around like other values. For example, we can write a function
that takes a trait (not an instance of a trait) and returns a new trait, or
assigns a trait to a variable. First-class traits usually imply dynamic trait
inheritance, where a trait can inherit from a trait expression, whose result is
determined dynamically. The dynamic nature of first-class traits poses a
challenge for static typing. The aforementioned notions of merging and
disjointness play a crucial role in ensuring that the dynamic trait inheritance
is type-safe. A type-safety issue of dynamic inheritance can be found in
\autoref{sec:override}, and CP's solution is later presented in
\autoref{sec:dynamic}.
