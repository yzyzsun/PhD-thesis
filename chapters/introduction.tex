\chapter{Introduction} \label{ch:introduction}

This thesis focuses on the practical aspects of \emph{compositional
programming}, which is a statically typed programming paradigm that emphasizes
modularity and extensibility, proposed by Weixin Zhang, Yaozhu Sun (the author),
and Bruno C. d. S. Oliveira \citeyearpar{zhang2021compositional}...

\section{Motivation}

Compositionality is an important principle in software engineering. It means
that a complex system can be built by composing simpler parts. However, it is
non-trivial to achieve compositionality in practice. During the evolution of a
software system, we often need to add new features, including extending the
current system with both new data types and new operations. The \emph{expression
problem}~\citep{wadler1998expression} illustrates the difficulty of such
two-dimensional extensibility in a modular way.

This is not only a theoretic problem, but also a practical issue, for example,
in the development of domain-specific languages (DSLs). A common approach to
defining DSLs is via a direct embedding into a host language. There are several
well-known techniques to do such embeddings, including \emph{shallow} and
\emph{deep} embeddings. However, it is hard to modularly extend language
constructs (data types) in deep embeddings or to modularly extend
interpretations (operations) in shallow embeddings. This two-dimensional dillema
corresponds to the aforementioned expression problem. Polymorphic
embeddings~\citep{hofer2008polymorphic}, tagless-final
embeddings~\citep{carette2009finally}, and object
algebras~\citep{oliveira2012extensibility} are some of the techniques that have
been proposed to address this issue, yet they all have drawbacks, such as
requiring boilerplate code and lacking a mechanism to modularly handle
dependencies between interpretations. In contrast, the CP language provides
native support for compositional embeddings, which is a modular paradigm to
support two-dimensional extensibility as well as dependent interpretations.

In general-purpose programming languages, compositionality often requires
mechanisms of code reuse. Object-oriented programming languages usually employs
inheritance, which allows code reuse by defining new classes that inherit from
existing ones. However, traditional inheritance mechanisms have limitations. For
example, multiple inheritance can lead to the diamond problem, where a class
inherits from two classes that have a common ancestor. This can cause ambiguity
when resolving method calls. Traits are a mechanism that addresses the diamond
problem by detecting ambiguous compositions and requiring the programmer to
resolve the conflicts explicitly. However, the detection of conflicts becomes
more challenging when dealing with dynamic inheritance, where classes (or
traits) can be composed at run time. Most statically typed languages only
provide static inheritance to achieve type safety at the cost of flexibility.

\paragraph{A three-fold challenge: achieving flexibility, efficiency, and type safety.}
Ideally, programming languages with dynamic inheritance and first-class classes
should have three properties:

\begin{enumerate}

\item \textbf{Flexibility:} The language should be flexible so that highly
      dynamic patterns of inheritance are allowed. Thus, it should be possible
      to support dynamic forms of mixins or traits, as well as nested classes or
      even virtual classes and family polymorphism.

\item \textbf{Reasonable efficiency and separate compilation:} For practical
      implementations, it is desirable to have a compilation model that is
      reasonably efficient and supports good software engineering properties,
      such as separate compilation.

\item \textbf{Type safety:} The language should be type-safe, so that type
      errors can be prevented statically.

\end{enumerate}

\noindent
Both JavaScript and TypeScript support points (1) and (2) well. As we have seen,
with first-class classes, we can model dynamic inheritance, mixins, nested
classes, and even virtual classes and family polymorphism. Therefore, the
inheritance model provided by JavaScript and TypeScript is expressive and
flexible. Furthermore, there has been a lot of work on optimizing JavaScript
implementations, so JavaScript and TypeScript's inheritance and class model are
reasonably efficient.

Unfortunately, for point (3), TypeScript's support for type-checking first-class
classes has a few type-soundness holes. Some of these holes, such as the use of
bivariant subtyping, are known and documented. First-class classes bring new
issues, such as the inexact superclass problem. The inexact superclass problem
can be avoided by moving into a model based on static inheritance, which is the
option widely adopted by most mainstream languages. However, this trades
flexibility for type safety. Ideally, we want to avoid this trade-off. Retaining
flexibility and type safety while addressing the inexact superclass problem is
non-trivial. In particular, it seems to be hard with the overriding semantics of
JavaScript, which simply overrides properties that have the same name. Thus, to
achieve the three goals together, a new compilation scheme seems desirable.

Previous work on compositional programming has addressed points (1) and (3).
However, that work has not studied practical implementability questions, such as
how to have a reasonably efficient compilation model with separate compilation.
Although there is an implementation of the CP language, this implementation is
based on an interpreter. Moreover, the semantics for compositional programming
languages is based on \emph{coercive subtyping}~\citep{luo2013coercive}, which
raises immediate questions in terms of efficiency, since upcasts have a
computational cost. A naive implementation that inserts coercions every time
upcasting is needed has a prohibitive cost, which can be \emph{orders of
magnitude} slower than JavaScript programs.

\paragraph{Solving the three-fold challenge: efficient compilation for CP.}
The problem that this paper solves is how to compile compositional programming
much more efficiently while also supporting separate compilation. Therefore, we
obtain property (2), which was missing on previous work on CP. Thus, we can
solve the three-fold challenge. We should emphasize that our work lacks various
features supported by JavaScript and TypeScript, and the semantics we employ for
inheritance has some important differences from JavaScript. Thus, our work does
not offer an immediate solution that TypeScript can adopt as a type-safe
replacement for their current class model. Nevertheless, our compilation model
can be useful for new languages that aim to have highly expressive models of
inheritance while ensuring type safety. Moreover, it can inform existing
language designers, who may be able to borrow some ideas to improve their
language designs.

\section{Organization}

\begin{description}
\item[\autoref{pt:prologue}] is the prologue. \autoref{ch:introduction}
      motivates this thesis and outlines its organization.
\item[\autoref{pt:background}] provides background information.
      \autoref{ch:background} introduces intersection and union types, merges,
      disjointness, and traits. \autoref{ch:cp} gives a crash course in the CP
      language, which implements the compositional programming paradigm.
\item[\autoref{pt:why}] explains why compositional programming matters. We
      illustrate the reasons with two applications of compositional programming
      in this part: \autoref{ch:embedding} proposes a new embedding of
      domain-specific languages; \autoref{ch:inheritance} presents a type-safe
      approach to dynamic inheritance via merging in CP.
\item[\autoref{pt:compile}] focuses on the compilation of compositional
      programming. \autoref{ch:key} describes the key ideas in our compilation
      scheme and its implementation in the CP compiler. \autoref{ch:calculi}
      formalizes a simplified version of the compilation scheme along some of
      the key ideas. \autoref{ch:compilation} explains implementation details,
      including the JavaScript code that is generated and some core
      optimizations in the CP compiler. \autoref{ch:empirical} provides an
      empirical evaluation.
\item[\autoref{pt:union}] further extends the CP language with union types.
      \autoref{ch:arguments} shows that this extension enables a type-safe
      encoding of named and optional arguments.
\item[\autoref{pt:epilogue}] is the epilogue. \autoref{ch:related} discusses
      related work, while \autoref{ch:conclusion} concludes this thesis and
      outlines future work.
\end{description}

\noindent\hrulefill

\paragraph{Prior publications.}
The main content of this thesis is based on three of my published papers.
\autoref{ch:embedding} is based on a conference paper:
\begin{itemize}
\item Yaozhu Sun, Utkarsh Dhandhania, and Bruno C.~d.~S.~Oliveira. 2022.
\textbf{Compositional Embeddings of Domain-Specific Languages}. In OOPSLA
\textit{(ACM SIGPLAN International Conference on Object-Oriented Programming
Systems, Languages, and Applications)}.
\end{itemize}
\autoref{ch:inheritance} and the whole \autoref{pt:compile} are based on a
journal paper:
\begin{itemize}
\item Yaozhu Sun, Xuejing Huang, and Bruno C.~d.~S.~Oliveira. 2025.
\textbf{Type-Safe Compilation of Dynamic Inheritance via Merging}.
\textit{ACM Transactions on Programming Languages and Systems}.
\end{itemize}
\autoref{ch:arguments} is based on another conference paper:
\begin{itemize}
\item Yaozhu Sun and Bruno C.~d.~S.~Oliveira. 2025.
\textbf{Named Arguments as Intersections, Optional Arguments as Unions}.
In ESOP \textit{(European Symposium on Programming)}.
\end{itemize}
In addition, \autoref{ch:cp} in the background part also adapts some material
from my co-authored journal paper:
\begin{itemize}
\item Weixin Zhang, Yaozhu Sun, and Bruno C.~d.~S.~Oliveira. 2021.
\textbf{Compositional Programming}. \textit{ACM Transactions on Programming
Languages and Systems}.
\end{itemize}
