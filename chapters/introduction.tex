\chapter{Introduction} \label{ch:introduction}

\section{Motivation}

\section{Organization}

\begin{description}
\item[\autoref{pt:prologue}] is the prologue. \autoref{ch:introduction}
      introduces the motivation and the organization of this thesis.
      \autoref{ch:background} provides background information on intersection
      and union types, traits, and the CP language, which implements the
      compositional programming paradigm.
\item[\autoref{pt:why}] explains why compositional programming matters. We
      illustrate the reasons with two applications of compositional programming
      in this part: \autoref{ch:embedding} proposes a new embedding of
      domain-specific languages; \autoref{ch:inheritance} presents a type-safe
      approach to dynamic inheritance via merging in CP.
\item[\autoref{pt:compile}] focuses on the compilation of compositional
      programming. \autoref{ch:key} describes the key ideas in our compilation
      scheme and its implementation in the CP compiler. \autoref{ch:calculi}
      formalizes a simplified version of the compilation scheme along some of
      the key ideas. \autoref{ch:compilation} explains implementation details,
      including the JavaScript code that is generated and some core
      optimizations in the CP compiler. \autoref{ch:empirical} provides an
      empirical evaluation.
\item[\autoref{pt:union}] further extends the CP language with union types.
      \autoref{ch:arguments} shows that this extension enables a type-safe
      encoding of named and optional arguments.
\item[\autoref{pt:epilogue}] is the epilogue. \autoref{ch:related} discusses
      related work, while \autoref{ch:conclusion} concludes this thesis and
      outlines future work.
\end{description}

The main content of this thesis is based on three of my published papers.
Specifically, \autoref{ch:embedding} is based on:
\begin{itemize}
\item Yaozhu Sun, Utkarsh Dhandhania, and Bruno C.~d.~S.~Oliveira. 2022.
\textbf{Compositional Embeddings of Domain-Specific Languages}. In OOPSLA
\textit{(ACM SIGPLAN International Conference on Object-Oriented Programming
Systems, Languages, and Applications)}.
\end{itemize}
\autoref{ch:inheritance} and the whole \autoref{pt:compile} are based on:
\begin{itemize}
\item Yaozhu Sun, Xuejing Huang, and Bruno C.~d.~S.~Oliveira. 2025.
\textbf{Type-Safe Compilation of Dynamic Inheritance via Merging}.
\textit{ACM Transactions on Programming Languages and Systems}.
\end{itemize}
\autoref{ch:arguments} is based on:
\begin{itemize}
\item Yaozhu Sun and Bruno C.~d.~S.~Oliveira. 2025.
\textbf{Named Arguments as Intersections, Optional Arguments as Unions}.
In ESOP \textit{(European Symposium on Programming)}.
\end{itemize}
In addition, the background part (i.e. \autoref{ch:background}) also adapts some
material from my co-authored paper:
\begin{itemize}
\item Weixin Zhang, Yaozhu Sun, and Bruno C.~d.~S.~Oliveira. 2021.
\textbf{Compositional Programming}. \textit{ACM Transactions on Programming
Languages and Systems}.
\end{itemize}
