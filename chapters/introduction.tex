\chapter{Introduction} \label{ch:introduction}

This thesis focuses on the practical aspects of \emph{compositional
programming}, which is a statically typed programming paradigm that emphasizes
modularity and extensibility, proposed by Weixin Zhang, Yaozhu Sun (the author),
and Bruno C. d. S. Oliveira \citeyearpar{zhang2021compositional}. In this
chapter, we first give an overview of the motivation and contributions of this
thesis. Then, we outline the organization of this thesis.

\section{Motivation}

Modern software systems are becoming increasingly complex, with codebases
spanning millions to billions of lines of code. Such complexity escalates
development costs and maintenance efforts, making it harder to evolve software
systems. \citet{brooks1987no} claimed in his famous essay \emph{No Silver Bullet}:
\begin{quoting}
``The complexity of software is an essential property, not an accidental one.''
\end{quoting}
However, the situation is far less pessimistic nowadays. We have seen many
advances in programming languages, tools, and methodologies that help manage
complexity. Even back in 1990s, \citet{cox95no} criticized
\citeauthor{brooks1987no}'s claim and emphasized the importance of
\emph{reusable software components}. Modularity, the ability to decompose a
system into independent reusable components, is a cornerstone of managing
complexity in software engineering. Like LEGO bricks, these components are
standardized, self-contained units that snap together via precise interfaces,
requiring no modification. The principle of modularity has been widely adopted
in modern software systems. For example, the microservices
architecture~\citep{richardson2018microservices}, employed by Amazon, Netflix,
Alibaba, and many other companies, organizes a system as a collection of
fine-grained services.

The benefits of modularity are threefold. First, it reduces coupling between
components, minimizing the impact of changes in one component on others. Second,
it enhances code reuse, allowing components to be repurposed across different
projects. Third, team collaboration is facilitated by modular design, as
different team members can work on different components independently.

Modularity promises ``plug-and-play'' composition, where components can interact
with each other through well-defined interfaces. However, achieving such
compositionality in practice remains challenging. The fundamental challenges
include:

\begin{enumerate}
\item \textbf{Flexibility versus safety.} Dynamic languages like JavaScript
      allow flexible composition but lack static guarantees, while static
      languages like Java pursue type safety at the cost of flexibility.
\item \textbf{Compositional conflicts.} The composition of two components may
      lead to ambiguity, inconsistency, or unintended behavior. These conflicts
      are typically due to name clashes, interface mismatches, or semantic
      contradictions.
\item \textbf{Modular dependencies.} A component may depend on another
      component, and the dependencies between components form a complex graph,
      which may include cycles. Thus, it is challenging to manage dependencies
      modularly.
\item \textbf{The expression problem~\textnormal{\citep{wadler1998expression}}.}
      Traditional paradigms, such as object-oriented programming (OOP) and
      functional programming (FP), struggle to support two-dimensional extensibility
      -- adding new data types and new operations -- in a modular way.
\end{enumerate}
The need to address these challenges simultaneously calls for a new programming
paradigm that emphasizes compositionality. The compositional programming
paradigm and the CP language~\citep{zhang2021compositional} are an attempt to
address these challenges.

\paragraph{Type-safe dynamic inheritance.}
To better illustrate the challenges of compositionality, let us first consider
their manifestation in the realm of OOP. Modularity in OOP is often achieved via
classes, and classes can be extended or composed via inheritance. In other
words, inheritance provides a mechanism for code reuse by defining new classes
that inherit from existing ones. However, traditional inheritance mechanisms
have limitations. For example, multiple inheritance can lead to the
\emph{diamond problem}, as shown in \autoref{fig:diamond-1}, where a class
inherits from two classes that have a common ancestor. The method call
\lstinline{new D().m()} is ambiguous, as it is unclear whether it should return
\lstinline{"B"} or \lstinline{"C"}.

\begin{figure}
\begin{subfigure}{.54\textwidth}
\begin{lstlisting}[language=TypeScript]
class A { m() { return "A"; } }
class B extends A { m() { return "B"; } }
class C extends A { m() { return "C"; } }
class D extends B, C {}
new D().m() // Should it return "B" or "C"?
\end{lstlisting}
\caption{Multiple inheritance.} \label{fig:diamond-1}
\end{subfigure}
\hfill
\begin{subfigure}{.39\textwidth}
\begin{lstlisting}[language=TypeScript]
function compose(X, Y) {
  return class extends X, Y {};
}
D = compose(B, C);
new D().m()
\end{lstlisting}
\caption{Dynamic multiple inheritance.} \label{fig:diamond-2}
\end{subfigure}
\caption{The diamond problem in OOP.}
\end{figure}

Traits~\citep{ducasse2006traits} are a mechanism that addresses the diamond
problem by detecting ambiguous compositions and requiring the programmer to
resolve the conflicts explicitly. However, the detection of conflicts becomes
more challenging when dealing with dynamic inheritance, where classes (or
traits) can be composed at run time. This feature can typically be found in
languages with first-class classes~\citep{takikawa2012gradual,lee2015theory}.
\autoref{fig:diamond-2} shows an example of the diamond problem with dynamic
multiple inheritance. Since we know nothing about the classes \lstinline{X} and
\lstinline{Y} until the function \lstinline{compose} is called, it is difficult
to detect the ambiguity in advance. Similar code can cause more severe issues
that break type safety. As a result, most statically typed languages only
provide static inheritance to achieve type safety at the cost of flexibility.

The diamond problem manifests the challenges of compositionality in OOP,
especially the aforementioned 1\st (\emph{flexibility versus safety}) and 2\nd
(\emph{compositional conflicts}) points. Ideally, an OOP language should be
flexible so that highly dynamic patterns of inheritance are allowed. At the same
time, it should be type-safe and can identify conflicts statically, so that type
errors and semantic ambiguities are prevented at compile time. It is interesting
to see whether and how the CP language can address these challenges.

\paragraph{Embedded domain-specific languages.}
The difficulty of compositionality also arises in the context of domain-specific
languages (DSLs). A DSL is a programming language tailored to a specific domain,
such as database queries, web development, and document authoring. DSLs can be
external or internal~\citep{ghosh2010dsls}. External DSLs are standalone
languages with their own syntax and semantics, such as SQL, HTML, and CSS.
Internal DSLs are typically embedded in a host language, such as parser
combinators~\citep{leijen2001parsec} and fluent
interfaces~\citep{fowler2005fluent}. There are several well-known techniques to
do such embeddings, including \emph{deep} and \emph{shallow}
embeddings~\citep{boulton1992experience}.

If we take a minimal DSL with numbers and addition as an example, deep and
shallow embeddings are implemented in \autoref{fig:embeddings}. In a deep
embedding, the abstract syntax tree of the DSL is represented as a data type
(e.g. \lstinline{Exp}), and the semantics of the DSL are defined by an
interpreter (e.g. \lstinline{eval}). In a shallow embedding, there is no
explicit data type for the DSL; the DSL is defined directly in terms of its
semantics (e.g. \lstinline{lit} and \lstinline{add} for evaluation).

\begin{figure}
\hspace{.06\textwidth}
\begin{subfigure}{.41\textwidth}
\begin{lstlisting}[language=Haskell]
data Exp where
  Lit :: Int -> Exp
  Add :: Exp -> Exp -> Exp

eval :: Exp -> Int
eval (Lit n) = n
eval (Add l r) = eval l + eval r
\end{lstlisting}
\caption{Deep embedding.}
\end{subfigure}
\hspace{.06\textwidth}
\begin{subfigure}{.31\textwidth}
\begin{lstlisting}[language=Haskell]
type Exp = Int

lit :: Int -> Exp
lit n = n

add :: Exp -> Exp -> Exp
add l r = l + r
\end{lstlisting}
\caption{Shallow embedding.}
\end{subfigure}
\caption{Deep and shallow embeddings of a minimal DSL.} \label{fig:embeddings}
\end{figure}

The two approaches have different trade-offs. Deep embeddings support
definitions by pattern matching, including optimizations and transformations
over the abstract syntax, and allow for multiple interpretations. Shallow
embeddings have the ability to reuse host-language optimizations in the DSL and
easily add new DSL constructs. Owing to these trade-offs, many embedded DSLs end
up using a mix of approaches in practice, requiring a substantial amount of
code, as well as some advanced coding techniques. What is worse, if one
interpretation of the DSL depends on another, there is no easy way in existing
embedding techniques to decouple them and maintain modularity. These challenges
manifest the 3\rd (\emph{modular dependencies}) and 4\th (\emph{the expression
problem}) points mentioned earlier and motivate us to explore a new embedding
technique that not only combines the advantages of deep and shallow embeddings
but also supports modular dependencies.

\paragraph{The 5\th challenge: efficient compilation for CP.}
While previous work on compositional programming enables a solution to the four
challenges of compositionality, a critical gap remains in terms of practical
implementability -- specifically, how to achieve efficient compilation with
separate compilation. To date, the only implementation of the CP language is
based on an interpreter, which, while functional, does not provide the
performance necessary for practical applications.

It is non-trivial to compile CP efficiently and separately because CP supports
several novel features that are not present in mainstream programming languages.
As mentioned earlier, \emph{dynamic inheritance} and \emph{compositional
embeddings} are two key features of compositional programming. They already pose
significant challenges to type-safe compilation, let alone efficiency and
separate compilation. Another challenging feature is \emph{nested
composition}~\citep{bi2018essence}. This feature enables a form of family
polymorphism~\citep{ernst2001family}, providing an elegant solution to the
expression problem~\citep{ernst2004expression}, so it is crucial to modularity
and extensibility in CP. Finally, though \emph{parametric polymorphism} is a
common feature in mainstream languages, it raises additional challenges in the
context of compilation for CP. The reason why this feature is challenging is a
bit more technical: the compilation scheme for CP relies on the concrete type of
each term, while parametric polymorphism allows abstracting over types and
delays type instantiation until run time. All these features are worth further
investigation to understand how to compile CP efficiently.

Moreover, the semantics underlying compositional programming languages rely on
\emph{coercive subtyping}~\citep{luo2013coercive}, which introduces inherent
efficiency concerns, since upcasts lead to computational overhead. A naive
implementation that inserts coercions every time upcasting is needed has a
prohibitive cost, which can be \emph{orders of magnitude} slower than JavaScript
programs. Therefore, how to reduce the overhead of coercive subtyping is also an
inevitable part of the 5\th challenge.

\paragraph{Compositional programming with union types.}
The modularity and extensibility of compositional programming are supported by
intersection types from a type-theoretic perspective. It is tempting to explore
how compositional programming can be extended with union types, the dual of
intersection types. Our preliminary investigation shows that union types enable
optional arguments in CP, while named arguments are already supported by
intersection types. Named and optional arguments are prevalent features in many
existing programming
languages~\citep{garrigue2001labeled,flatt2009keyword,rytz2010named}, enhancing
code readability and flexibility. Despite widespread use, their formalization
has not been extensively studied in the literature. Especially in languages with
subtyping, such as Python and Ruby, first-class named arguments can lead to
type-safety issues. It is worthwhile to conduct a formal study of type-safe
foundations for named and optional arguments.

\section{Contributions}

\textbf{\autoref{pt:why}} mainly serves as an appetizer for the whole thesis,
presenting two applications of compositional programming that illustrate the
importance of the paradigm:
\begin{itemize}
\item \textbf{Dynamic inheritance via merging.} We identify a type-safety issue
      in TypeScript which we call the inexact superclass problem. We model
      family-polymorphic dynamic multiple inheritance as nested trait
      composition via merging in CP, which is free from the inexact superclass
      problem and semantic ambiguity by avoiding implicit overriding.
      Disjointness plays a crucial role in ensuring type safety in the presence
      of dynamic inheritance.
\item \textbf{Compositional embeddings.} Compositional programming, together
      with the CP language, enables a new form of embedding, which we call a
      compositional embedding. We reveal that compositional embeddings have most
      of the advantages of shallow and deep embeddings. We also make a detailed
      comparison between compositional embeddings and various other techniques
      used in embedded language implementations, including hybrid, polymorphic,
      and tagless-final embeddings.
\end{itemize}
The main body of this thesis significantly improves the implementation of the CP
language in several aspects:\footnote{The latest version of CP is available at
\url{https://github.com/yzyzsun/CP-next}.}
\begin{itemize}
\item \textbf{In-browser interpreter and the \ExT DSL.} We reimplement CP as an
      in-browser interpreter. On top of this, we implement a DSL for document
      authoring called \ExT as a realistic instance of compositional embeddings.
      \ExT is extremely flexible and customizable by users, with many features
      implemented in a modular way. We have built several applications with
      \ExT, three of which are discussed in more detail in this thesis. The
      largest application is Minipedia, and the other two applications
      illustrate computational graphics like fractals and a document extension
      for charts.\footnote{The online editor for \ExT and its applications are
      available at \url{https://plground.org}.}
\item \textbf{Compiler from CP to JavaScript.} We implement a compiler for CP
      that targets JavaScript, supporting modular type checking and separate
      compilation. We propose an efficient compilation scheme that translates
      merges into extensible records, where types are used as record labels to
      perform lookup on merges. In our concrete implementation, records are
      modeled as JavaScript objects, and record extension is modeled by
      JavaScript's support for object extension.
\item \textbf{Several optimizations and an empirical evaluation.} We discuss
      several optimizations that we employ in the CP compiler and conduct an
      empirical evaluation to measure their impact. Besides, we benchmark the
      JavaScript code generated by our compiler together with handwritten
      JavaScript code.\footnote{The benchmark suite is available at
      \url{https://github.com/yzyzsun/CP-next/tree/toplas}.}
\item \textbf{Extension with named and optional arguments.} We further extend CP
      with union types and add support for named and optional arguments. Named
      arguments in CP are first-class values and avoid the type-safety issue in
      Python and Ruby. We show a simple example of fractals that benefits from
      named and optional arguments.
\end{itemize}
To ensure the correctness of the CP compiler and its extension, we provide the
following formalization mechanized using the Coq proof assistant:
\begin{itemize}
\item \textbf{Type-safety proofs for the compilation scheme.} We formalize the
      compilation scheme as an elaboration from the \lambdaiplus
      calculus~\citep{bi2018essence,huang2021taming} to a calculus with
      extensible records called \lambdar and prove the type safety
      thereof.\footnote{The Coq formalization is available at
      \url{https://github.com/yzyzsun/CP-next/tree/main/theories}.}
\item \textbf{Type-safety proofs for named and optional arguments.} We formalize
      the encoding of named and optional arguments as an elaboration from a
      minimal functional language called \uaena to a core calculus with
      intersection and union types called \lambdaiu~\citep{rehman2023blend} and
      prove the type safety thereof.\footnote{The Coq formalization is available
      at \url{https://github.com/yzyzsun/lambda-iu}.}
\end{itemize}

\section{Organization}

\begin{description}
\item[\autoref{pt:prologue}] is the prologue. \autoref{ch:introduction}
      motivates this thesis and outlines its organization.
\item[\autoref{pt:background}] provides background information.
      \autoref{ch:background} introduces intersection and union types, merges,
      disjointness, and traits. \autoref{ch:cp} gives a crash course in the CP
      language, which implements the compositional programming paradigm.
\item[\autoref{pt:why}] explains why compositional programming matters. We
      illustrate the reasons with two applications of compositional programming
      in this part: \autoref{ch:inheritance} presents a type-safe approach to
      dynamic inheritance via merging in CP; \autoref{ch:embedding} proposes a
      new embedding technique for domain-specific languages.
\item[\autoref{pt:compile}] focuses on the compilation of compositional
      programming. \autoref{ch:key} describes the key ideas in our compilation
      scheme and its implementation in the CP compiler. \autoref{ch:calculi}
      formalizes a simplified version of the compilation scheme along some of
      the key ideas. \autoref{ch:compilation} explains implementation details,
      including the JavaScript code that is generated and some core
      optimizations in the CP compiler. \autoref{ch:empirical} provides an
      empirical evaluation.
\item[\autoref{pt:union}] further extends the CP language with union types.
      \autoref{ch:arguments} shows that this extension enables a type-safe
      encoding of named and optional arguments.
\item[\autoref{pt:epilogue}] is the epilogue. \autoref{ch:related} discusses
      related work, while \autoref{ch:conclusion} concludes this thesis and
      outlines future work.
\end{description}

\pagebreak

\paragraph{Prior publications.}
The main content of this thesis is based on three of my papers.
\autoref{ch:embedding} is based on a conference paper:
\begin{itemize}
\item Yaozhu Sun, Utkarsh Dhandhania, and Bruno C.~d.~S.~Oliveira. 2022.
\textbf{Compositional Embeddings of Domain-Specific Languages}. In OOPSLA
\textit{(ACM SIGPLAN International Conference on Object-Oriented Programming
Systems, Languages, and Applications)}.
\end{itemize}
\autoref{ch:arguments} is based on another conference paper:
\begin{itemize}
\item Yaozhu Sun and Bruno C.~d.~S.~Oliveira. 2025.
\textbf{Named Arguments as Intersections, Optional Arguments as Unions}.
In ESOP \textit{(European Symposium on Programming)}.
\end{itemize}
\autoref{ch:inheritance} and the whole \autoref{pt:compile} are based on an
unpublished paper:
\begin{itemize}
\item Yaozhu Sun, Xuejing Huang, and Bruno C.~d.~S.~Oliveira. 2025.
\textbf{Type-Safe Compilation of Dynamic Inheritance via Merging}.
In submission to \textit{ACM Transactions on Programming Languages and Systems}.
\end{itemize}
In addition, \autoref{ch:cp} in the background part also adapts some material
about modular dependencies from my co-authored journal paper:
\begin{itemize}
\item Weixin Zhang, Yaozhu Sun, and Bruno C.~d.~S.~Oliveira. 2021.
\textbf{Compositional Programming}. \textit{ACM Transactions on Programming
Languages and Systems}.
\end{itemize}
