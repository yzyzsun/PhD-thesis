First of all, I would like to thank my PhD supervisor, Prof.\ Bruno C.\ d.\ S.\
Oliveira, for his continuous guidance and support throughout my study. Bruno is
a very patient and responsive supervisor, who is always willing to listen to
every detail of my research and provide valuable feedback in our weekly
meetings. I knew very little about type systems before I started my study at
HKU. It was Bruno who introduced me to the world of intersection types and
gradually guided me to the research area of my thesis. Without his help, I can
hardly imagine how I could have published several papers and completed my PhD
study. Bruno is also easy-going in daily life, so I never hesitate to share
happy moments or difficulties with him. I could not expect a better supervisor
than Bruno.

My past supervisor during my undergraduate exchange to Tokyo Institute of
Technology, Prof.\ Hidehiko Masuhara, was the first professor who showed me the
fun of programming languages research. I am grateful to him and Dr.\ Matthias
Springer for their very first guidance in my research career. I really miss the
enjoyable one year in Tokyo. Fortunately, Prof.\ Taro Sekiyama recently afforded
me a postdoctoral position at National Institute of Informatics, allowing me to
return to Tokyo and continue my research on type systems. I cannot wait to
reunite with my friends in Tokyo and enjoy the life there again. Furthermore, I
would also like to thank Prof.\ Jonathan Aldrich from Carnegie Mellon University,
who approved to be my external examiner.
% who served as my external examiner, for his insightful comments on my thesis.

The first two or three years of my PhD study were tough because of the COVID-19
pandemic. I was stuck in Hong Kong and could not travel to any conferences
abroad. I only went home once during the pandemic due to mainland China's
strict quarantine policies. Nevertheless, I was lucky to have a group of friends
who accompanied me and made my life in Hong Kong more colorful. These friends
include my past roommates: Xu Xue, Chen Cui, Zhengjie Shu, Guangqin Song, and
Yunsong Lu. Sincere thanks also go to my friends-and-collaborators: Weixin
Zhang, Xuejing Huang, Andong Fan, Han Xu, Utkarsh Dhandhania, and Wenjia Ye.
Other members or alumni of HKU programming languages group deserve my thanks as
well: Xuan Bi, Yanlin Wang, Ningning Xie, Jinxu Zhao, Yaoda Zhou, Baber Rehman,
Mingqi Xue, Shengyi Jiang, Jinhao Tan, Litao Zhou, Qianyong Wan, Bowen Su,
Yicong Luo, and Ziyu Li.

I would like to extend my gratitude to Chuchu Gui from AKB48 Team SH, among
other idols, who has been a constant source of positive energy for me during my
hard times. I really appreciate the joy of growing up along with her. The naming
of \lambdaiu and \uaena in \autoref{ch:arguments} is also inspired by her: IU is
her favorite singer and UAENA refers to IU's fans.

Last but not least, my greatest thanks are due to my parents. My parents have
always been supportive of my pursuit of a PhD degree. When I was hesitating
between continuing my study in academia and finding a job in industry, they
encouraged me to follow my heart and worry less about money. My life and my
current life are both given by my parents.
