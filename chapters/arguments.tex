\chapter{Named Arguments as Intersections, Optional Arguments as Unions} \label{ch:arguments}

Named and optional arguments are prevalent features in many mainstream
programming languages, enhancing code readability and flexibility. Despite
widespread use, their formalization has not been extensively studied.

This part extends compositional programming with union types, enabling a
type-safe foundation for named and optional arguments. We first conduct a survey
of existing languages' support for named arguments in \autoref{sec:existing}.
Then we identify in \autoref{sec:bad} a critical type-safety issue in popular
static type checkers for Python and Ruby, particularly in handling first-class
named arguments in the presence of subtyping. Our solution is informally
presented in \autoref{sec:type-safe} and formalized in \autoref{sec:iu-uaena}
through an elaboration from a functional language with named and optional
arguments (\uaena) to a minimal core calculus with intersection and union types
(\lambdaiu). We conclude this chapter with a discussion of related work in
\autoref{sec:related}.

\section{Named Arguments in Existing Languages} \label{sec:existing}

The $\lambda$-calculus, introduced by \citet{church1941calculi}, shows how to
model computation solely with function abstraction and application. For example,
natural numbers, boolean values, pairs, and lists, as well as various operations
on them, can be represented by higher-order functions via Church encoding. In
the $\lambda$-calculus, a function only has one parameter and can only be
applied to one argument. Many programming languages in the ML family inherit
this feature. If more than one argument is desired in those languages, we need
to create a sequence of functions, each with a single argument, and perform an
iteration of applications. This idea is called \emph{currying}. Currying brings
brevity to functional programming and naturally allows partial application, but
it usually limits the flexibility of function application. For example, we
cannot pass arguments in a different order nor omit some of them by providing
default values. Both demands are not rare in practical programming and can be
met in a language that supports \emph{named} and \emph{optional} arguments.
Named arguments also largely improve the readability of function calls. For
example, it is unclear which is the source and which is the destination in
\lstinline{copy(x, y)}, while \lstinline{copy(to: x, from: y)} is
self-explanatory.

\begin{figure}
\begin{subfigure}{0.5\textwidth}
\begin{lstlisting}[language={[3]Python}]
def exp(x, base=math.e):
  return base ** x

exp(10, 2)  #= exp(x=10, base=2) = 1024
exp(base=2, x=10)               #= 1024
exp(x=10)                       #= e^10

args = { "base": 2, "x": 10 }
exp(**args) #= exp(base=2, x=10) = 1024
\end{lstlisting}
\caption{The Python way.} \label{fig:python}
\end{subfigure}
\hfill
\begin{subfigure}{0.35\textwidth}
\begin{lstlisting}[language=Ruby]
def exp(x:, base: Math::E)
  base ** x
end
exp(10, 2) # ArgumentError!
exp(base: 2, x: 10) #= 1024
exp(x: 10)          #= e^10

args = { base: 2, x: 10 }
exp(**args)         #= 1024
\end{lstlisting}
\caption{The Ruby way.} \label{fig:ruby}
\end{subfigure}
\caption{Named arguments in Python and Ruby.} \label{fig:python-ruby}
\end{figure}

Named arguments are widely supported in mainstream programming languages, such
as Python, Ruby, OCaml, C\#, and Scala, just to name a few. The earliest
instance, to the best of our knowledge, is Smalltalk, where every method
argument \emph{must} be associated with a \emph{keyword} (i.e. an external
name). In other words, there are no positional arguments (i.e. arguments with no
keywords) in Smalltalk. The syntax of modern languages is usually less rigid, so
programmers can choose whether to attach keywords to arguments or not. There are
two ways to reconcile positional and named arguments. One way, employed by
Python and shown in \autoref{fig:python}, is to make parameter names in a
function definition as non-mandatory keywords. Thus, every argument can be
passed with or without keywords. As shown in the Python code,
\lstinline{exp(10, 2)} is equivalent to \lstinline{exp(x=10, base=2)}. To
reconcile the two forms in the same call, a restriction is imposed that all
positional arguments must appear to the left of named ones. The other way, shown
in \autoref{fig:ruby} and used in Ruby, is to strictly distinguish named
arguments from positional ones. When defining a Ruby function, a named parameter
should always end with a colon even if it does not have a default value. By this
means, they are syntactically distinct from positional parameters, and their
keywords cannot be omitted in a function call. The two kinds of arguments are
usually used in different scenarios: positional arguments are used when the
number of arguments is small and the order is clear, while named arguments are
used in more complex cases especially when settings or configurations are
involved.

More interestingly, named arguments are \emph{first-class} values in Python and
Ruby: they can be assigned to a variable. As shown at the bottom of
\autoref{fig:python-ruby}, the variable \lstinline{args} stores the two
arguments named \lstinline{base} and \lstinline{x}, and we can later pass it to
\lstinline{exp} by unpacking it with \lstinline{**} (sometimes called the splat
operator). In fact, \lstinline{args} is a dictionary in Python and similarly a
hash in Ruby. Thus, first-class named arguments can be manipulated and passed
around like standard data structures. This feature is widely used in Python and
Ruby.

\begin{table}
\caption{Named arguments with different design choices in different languages.} \label{tab:survey}
\centering
\small
\setlength{\tabcolsep}{1ex}
\begin{tabular}{*{11}{c}}
\toprule
                    &Smalltalk&  Python &  Ruby   &  Racket     &  OCaml  &  C\#    &  Scala  &  Dart   &  Swift  &  CP     \\
\midrule \midrule
Commutativity       & \Circle & \CIRCLE & \CIRCLE & \CIRCLE     & \CIRCLE & \CIRCLE & \CIRCLE & \CIRCLE & \Circle & \CIRCLE \\
Optionality         & \Circle & \CIRCLE & \CIRCLE & \CIRCLE     & \CIRCLE & \CIRCLE & \CIRCLE & \CIRCLE & \CIRCLE & \CIRCLE \\
Currying            & \Circle & \Circle & \Circle & \Circle     & \CIRCLE & \Circle & \Circle & \Circle & \Circle & \Circle \\
Distinctness        & \em n/a & \Circle & \CIRCLE & \CIRCLE     & \CIRCLE & \Circle & \Circle & \CIRCLE & \CIRCLE & \CIRCLE \\
First-class value   & \Circle & \CIRCLE & \CIRCLE & \LEFTcircle & \Circle & \Circle & \Circle & \Circle & \Circle & \CIRCLE \\
\midrule
Static typing       & \Circle & \Circle & \Circle & \Circle     & \CIRCLE & \CIRCLE & \CIRCLE & \CIRCLE & \CIRCLE & \CIRCLE \\
Soundness proof     & \Circle & \Circle & \Circle & \Circle     & \CIRCLE & \Circle & \Circle & \Circle & \Circle & \CIRCLE \\
\bottomrule
\end{tabular}
\vskip 1ex
{\footnotesize
  \emph{n/a}:  Smalltalk does not support positional arguments at all. \\
  \LEFTcircle: Racket's support for first-class named arguments is limited and forbids commutativity.
}
\end{table}

Including the distinctness and first-class values illustrated above, we have
identified five important design choices found in existing languages that
support named arguments:
\begin{enumerate}
\item \emph{Commutativity}: whether the order of (actual) arguments can be
      different from that of (formal) parameters originally declared.
\item \emph{Optionality}: whether some arguments can be omitted in a function
      call if their default values are predefined.
\item \emph{Currying}: whether a function that takes more than one argument is
      always converted into a chain of functions that each take a single
      argument.
\item \emph{Distinctness}: whether named arguments are distinct from positional
      ones in how they are defined and passed.
\item \emph{First-class value}: whether named arguments are first-class values.
\end{enumerate}
As shown in \autoref{tab:survey}, the first two properties hold for most
mainstream programming languages, with Smalltalk and Swift being two exceptions.
Commutativity and optionality are so useful that we believe they should not be
compromised. Concerning the third point, OCaml is the only language that manages
to reconcile currying with commutativity, though at the cost of introducing a
very complicated core calculus. We agree that currying is very useful when we
use normal positional arguments, but we argue here that currying can be
temporarily dropped when we use named arguments because the most common use case
for named arguments is to represent a whole chunk of parameters like settings or
configurations. The fourth design, distinctness, is endorsed by Ruby, Racket,
OCaml, Dart, and Swift. It improves the readability of call sites to enforce
keywords whenever arguments are defined to be named. We advocate distinctness in
this work also because it simplifies the language design and allows us to focus
on more important topics, especially type safety with first-class named
arguments.

Although named arguments are ubiquitous, they have not attracted enough
attention in the research of programming languages. Among the few related
papers, the work by \citet{garrigue1994label} formalizes a label-selective
$\lambda$-calculus and eventually apply it to OCaml~\citep{garrigue2001labeled}.
Another work by \citet{rytz2010named} discusses the design of named and optional
arguments in Scala, but it mainly focuses on practical aspects. The core
features of Scala are formalized in a family of DOT calculi, but named arguments
are never included. The support for named arguments is implemented as macros in
Racket~\citep{flatt2009keyword}. So their extension is more like userland
syntactic sugar and requires no changes to the core compiler. Haskell does not
support named arguments natively, but the paradigm of \emph{named arguments as
records} is folklore. We will discuss OCaml, Scala, Racket, and Haskell in
detail in \autoref{sec:related}. In short, named arguments are implemented in an
ad-hoc manner and are not well founded from a type-theoretic perspective in most
languages, especially object-oriented ones. Only OCaml and CP provide
\emph{soundness proofs} for the feature of named arguments.

An important issue that has not been explored in the literature is the
interaction between subtyping and first-class named arguments. A naive design
can easily lead to a type-safety issue. We will show in \autoref{sec:mypy} that
the most widely used optional type checker for Python, mypy~\citep{mypy}, fails
to detect a type-unsafe use of first-class named arguments. The same issue also
exists in Ruby with Steep~\citep{steep} or Sorbet~\citep{sorbet}. It arises from
subtyping hiding some arguments from their static type and bypassing the type
checking for optional arguments. As a result, an optional argument may have an
unexpected type at run time, which leads to a runtime error.

In this chapter, we present a type-safe foundation for named and optional
arguments. At the heart of our approach is the translation into a core calculus
called \lambdaiu, which features \emph{intersection and union
types}~\citep{barbanera1995intersection,frisch2008semantic,dunfield2014elaborating}.
Our approach supports first-class named arguments like Python and Ruby, but the
type-safety issue is addressed by us. The \lambdaiu calculus has been shown to
be type-sound~\citep{rehman2023blend}, and we show that our translation from our
source language into \lambdaiu is type-safe. Thus, we establish the type safety
of our approach.

\section{Named and Optional Arguments: The Bad Parts}

Since named and optional arguments are not well studied in most languages, the
ad-hoc mechanisms employed in those languages may sometimes surprise programmers
or even cause safety issues.

\subsection{Gotcha! Mutable Default Arguments in Python} \label{sec:mut}

Let us consider a simple Python function that appends an element to a list. We
provide a default value for the list, which is an empty list:

\begin{multicols}{2}
\begin{lstlisting}[language={[3]Python}]
def append(x, xs=[]):
  xs.append(x)
  return xs
append(1) #= [1]
append(2) #= [1, 2]
\end{lstlisting}
\end{multicols}

\noindent
After calling \lstinline{append(1)} above, we get the expected result
\lstinline{[1]}. However, continuing to call \lstinline{append(2)} gives us
\lstinline{[1, 2]} instead of \lstinline{[2]}. This is because Python only
evaluates the default value once when the function is defined, so the same list
initialized for \lstinline{xs} is shared across different calls to
\lstinline{append}. When calling \lstinline{append(2)}, the default value for
\lstinline{xs} is no longer an empty list but the list that has been modified by
the previous call \lstinline{append(1)}.

This issue, while seemingly minor, highlights the importance of understanding
the semantics of default arguments. Our design strives to avoid such surprises,
following the principle of least astonishment, yet this is not our main focus.
We will discuss the more critical issue about type safety next.

\subsection{Caution! Type Safety with First-Class Named Arguments} \label{sec:mypy}

As we have shown in \autoref{fig:python-ruby}, quite a few languages, especially
dynamically typed ones like Python and Ruby, treat named arguments as
first-class values. This feature is particularly helpful for passing settings
because they are usually stored in a separate configuration file. We can read
the settings from the file and pass them as named arguments using the
\lstinline{**} operator. For example, we can find such code in Python to run a
web server:
\begin{lstlisting}[language={[3]Python}]
class App: # from a web server library
  def run(self, host: str, port: int, debug: bool = False):
    assert isinstance(debug, bool) # actual code omitted...

args = { "host": "0.0.0.0", "port": 80, "debug": True }
app.run(**args) #= app.run(host="0.0.0.0", port=80, debug=True)
\end{lstlisting}
Although Python is dynamically typed, there is continuous effort in the Python
community to improve the detection of type errors earlier in the development
process, primarily through static analysis. There is an optional static type
checker for Python called mypy~\citep{mypy}.\footnote{Our code is tested against
mypy 1.14.0 (released on 20 December 2024).} In the example above, we make use
of \emph{type hints}, introduced in Python 3.5, to specify the types of the
parameters and the return value of the \lstinline{run} method. The type hints
have no effect at run time but can be used by external tools like mypy to
statically check if the code is well-typed. Perhaps surprisingly, the code above
cannot pass mypy's type checking, because the type inferred for \lstinline{args}
(i.e. \lstinline{dict[str,object]}) is not precise enough. The type checker
needs to know what keys \lstinline{args} exactly has and what types the values
associated with those keys have, in order to make sure that \lstinline{**args}
is compatible with the parameters of \lstinline{app.run}.

Fortunately, \lstinline{TypedDict} is added in Python 3.8 to represent a
specific set of keys and their associated types. By default, every specified key
is required, except when it is marked as \lstinline{NotRequired}, which is a
type qualifier added later in Python 3.11. With \lstinline{TypedDict} and
\lstinline{NotRequired}, we can now define a precise dictionary type for
\lstinline{args} that passes mypy's type checking:
\begin{lstlisting}[language={[3]Python}]
class Args(TypedDict):
  host: str
  port: int
  debug: NotRequired[bool]

args0: Args = { "host": "0.0.0.0", "port": 80, "debug": True }
app.run(**args0) # type-checks in mypy
args1: Args = { "host": "0.0.0.0", "port": 80 }
app.run(**args1) # type-checks in mypy, too
\end{lstlisting}
The mypy type checker will raise an error if we provide an argument with an
incompatible type, such as a string for the \lstinline{debug} key:
\begin{lstlisting}[language={[3]Python}]
class In(TypedDict):
  host: str
  port: int
  debug: str

args2: In = { "host": "0.0.0.0", "port": 80, "debug": "Oops!" }
app.run(**args2) # TypeError: Argument "debug"
# has incompatible type "str"; expected "bool"  [arg-type]
\end{lstlisting}
However, mypy's type system is not completely type-safe. We can create a
function \lstinline{f} that takes a dictionary with three keys specified in type
\lstinline{In} and returns a dictionary with only two keys specified in type
\lstinline{Out}. The function type-checks in mypy because type \lstinline{In} is
compatible whenever type \lstinline{Out} is expected. Roughly speaking, it means
that \lstinline{In} is a subtype of \lstinline{Out}. Then we can use
\lstinline{f} to forget the \lstinline{debug} key in the static type:
\begin{lstlisting}[language={[3]Python}]
class Out(TypedDict):
  host: str
  port: int

def f(args: In) -> Out: return args

args3 = f(args2) # still contains { "debug": "Oops!" }
app.run(**args3) # type-checks in mypy, but has a runtime error!
\end{lstlisting}
Here \lstinline{args3} has type \lstinline{Out} without the \lstinline{debug}
key specified. From a static viewpoint, \lstinline{args3} only has two keys
\lstinline{host} and \lstinline{port}, which are compatible with the parameters
of \lstinline{app.run} since \lstinline{debug} is optional and has a default
value. That is why \lstinline{app.run(**args3)} type-checks in mypy. However, at
run time, the \lstinline{debug} key is still present in \lstinline{args3}, so
the string \lstinline{"Oops!"} is passed as a named argument to
\lstinline{app.run}, which originally expects a boolean value. This results in a
runtime error since there is an assertion in \lstinline{app.run} to ensure that
\lstinline{debug} is boolean.

This issue is not unique to Python and mypy. We have reproduced nearly the same
issue in Ruby with two popular type checkers, namely Steep~\citep{steep} and
Sorbet~\citep{sorbet}, which is illustrated in \autoref{sec:steep} and
\autoref{sec:sorbet}.\footnote{Our code is tested against Steep 1.9.2 (16
December 2024) and Sorbet 0.5.11708 (20 December 2024).}

In conclusion, subtyping can lead to a fundamental type-safety issue when
dealing with first-class named and optional arguments. In essence, the following
subsumption chain is questionable:
\begin{lstlisting}[language={[3]Python}]
   { host: str, port: int, debug: str }
<: { host: str, port: int }
<: { host: str, port: int, debug?: bool }
\end{lstlisting}
Following this chain bypasses mypy's type compatibility checking for the
\lstinline{debug} key. Next we will show how to break the chain and address the
type-safety issue.

\section{Our Type-Safe Approach} \label{sec:type-safe}

In this section, we informally present how we translate named and optional
arguments into a core language with intersection and union types, while
retaining type safety. We start by introducing the core language constructs that
we need. Then we illustrate our translation scheme by example and demonstrate
how it recovers type safety. After that, we showcase a practical example in the
CP language, which has incorporated our approach to support named and optional
arguments. Finally, we discuss how our translation scheme can be applied to
other languages.

\subsection{Core Language} \label{sec:core}

The core language features intersection and union types, which establish an
elegant duality in the type system. A value of the intersection type $\ottnt{A}  \land  \ottnt{B}$
can be assigned both $\ottnt{A}$ and $\ottnt{B}$, whereas a value of the union type
$\ottnt{A}  \lor  \ottnt{B}$ can be assigned either $\ottnt{A}$ or $\ottnt{B}$. Intersection and union
types correspond to the logical conjunction and disjunction
respectively.\footnote{In this chapter, we use the notation $\ottnt{A}  \land  \ottnt{B}$ for
intersections and $\ottnt{A}  \lor  \ottnt{B}$ for unions, to better align with the literature,
rather than the notation $A \tand B$ and $A \tor B$ in CP.} Similar calculi are widely
studied~\citep{barbanera1995intersection,frisch2008semantic,dunfield2014elaborating}
and provide a well-understood foundation for named and optional arguments.

\paragraph{Named arguments as intersections.}
Named arguments are translated to multi-field records. However, the core
language does not support multi-field records directly. There are only
single-field records in the core language, and multiple fields are represented
as intersections of single-field record types. For example,
$\ottsym{\{}  \ottmv{x}  \ottsym{:}   \mathbb{Z}   \ottsym{\}}  \land  \ottsym{\{}  \ottmv{y}  \ottsym{:}   \mathbb{Z}   \ottsym{\}}$ represents a record type with two integer fields $x$ and
$y$. With intersection types, width subtyping for record types comes for free,
and permutations of record fields are naturally
allowed~\citep{reynolds1997design}.

At the term level, a merge
operator~\citep{dunfield2014elaborating,rehman2023blend} is used to concatenate
multiple single-field records to form multi-field records, reminiscent of
Forsythe~\citep{reynolds1997design}. For example, $\ottsym{\{}  \ottmv{x}  \ottsym{=}  \ottsym{1}  \ottsym{\}}  \bbcomma  \ottsym{\{}  \ottmv{y}  \ottsym{=}  \ottsym{2}  \ottsym{\}}$ forms a
two-field record from two single-field records.

\paragraph{Optional arguments as unions.}
Optional arguments are translated to nullable types. A nullable type is not
implicit in the core language but is represented as a union with the null
type~\citep{nieto2020scala}. For example, an optional integer argument named $z$
is translated to $\ottsym{\{}  \ottmv{z}  \ottsym{:}   \mathbb{Z}   \lor  \ottkw{Null}  \ottsym{\}}$.

At the term level, a type-based switch
expression~\citep{frisch2008semantic,rehman2023blend} is used to scrutinize a
term of a union type, reminiscent of ALGOL~68~\citep{van1975revised}. For
example, $\ottkw{switch} \, \ottmv{z} \, \ottkw{case} \,  \mathbb{Z}   \Rightarrow  \ottnt{e_{{\mathrm{1}}}} \, \ottkw{case} \, \ottkw{Null}  \Rightarrow  \ottnt{e_{{\mathrm{2}}}}$ returns $\ottnt{e_{{\mathrm{1}}}}$ if $z$
is an integer or $\ottnt{e_{{\mathrm{2}}}}$ if $\ottkw{null}$.

\subsection{Translation by Example} \label{sec:example}

Let us review the previous Python function in \autoref{sec:mut} that appends an
element to a list, which defaults to an empty list:
\begin{lstlisting}[language={[3]Python}]
def append(x: int, xs: list[int] = []): ...
\end{lstlisting}
The function will be translated to a core function as follows:
\begin{align*}
\mathtt{append} =~&  \lambda \ottmv{args} \!:\! \ottsym{\{}  \ottmv{x}  \ottsym{:}   \mathbb{Z}   \ottsym{\}}  \land  \ottsym{\{}  \ottmv{xs}  \ottsym{:}  \ottsym{[}   \mathbb{Z}   \ottsym{]}  \lor  \ottkw{Null}  \ottsym{\}} .\;     \\
                  & \ottkw{let} \, \ottmv{x}  \ottsym{=}  \ottmv{args}  \ottsym{.}  \ottmv{x} \, \ottkw{in} \\
                  & \ottkw{let} \, \ottmv{xs}  \ottsym{=}  \ottkw{switch} \, \ottmv{args}  \ottsym{.}  \ottmv{xs} \, \ottkw{as} \, \ottmv{xs} \, \ottkw{case} \,  \mathbb{Z}   \Rightarrow  \ottmv{xs} \, \ottkw{case} \, \ottkw{Null}  \Rightarrow   [\,]  \, \ottkw{in} \\
                  & \cdots
\end{align*}
Here we can see that the default value (i.e. the empty list) is not shared
across different calls to \lstinline{append} because the default value is
evaluated within the function body. Therefore, calling \lstinline{append(x=1)}
will consistently return \lstinline{[1]} instead of surprisingly modifying the
default value. This design leads to less astonishment and more predictable
behavior.

Since we translate named parameter types to record types, we correspondingly
translate named arguments to records. For example, the function call
\lstinline{append(x=1, xs=[0])} will be translated to {\small
$\mathtt{append}\,\ottsym{(}  \ottsym{\{}  \ottmv{x}  \ottsym{=}  \ottsym{1}  \ottsym{\}}  \bbcomma  \ottsym{\{}  \ottmv{xs}  \ottsym{=}  \ottsym{[0]}  \ottsym{\}}  \ottsym{)}$}.

\paragraph{Rewriting call sites.}
More importantly, we also rewrite call sites to add null values for absent
optional arguments. For example, the function call \lstinline{append(x=1)} will
be rewritten and translated to {\small
$\mathtt{append}\,\ottsym{(}  \ottsym{\{}  \ottmv{x}  \ottsym{=}  \ottsym{1}  \ottsym{\}}  \bbcomma  \ottsym{\{}  \ottmv{y}  \ottsym{=}  \ottkw{null}  \ottsym{\}}  \ottsym{)}$}.

\paragraph{Dependent default values.}
Another advantage of our translation scheme is that it naturally allows default
values to depend on earlier arguments. Python and Ruby do not support dependent
default values, but this feature can be useful in some practical scenarios. For
example, when setting up I/O, we may want to output error messages to the same
stream as \lstinline{out} by default:
\begin{lstlisting}[language={[3]Python}]
def setIO(in_, out, err = out): ...
\end{lstlisting}
The variable \lstinline{out} can be used in the default value of \lstinline{err}
because it has been brought into scope by the previous let-in binding:
\begin{align*}
& \ottkw{let} \, \ottmv{out}  \ottsym{=}  \ottmv{args}  \ottsym{.}  \ottmv{out} \, \ottkw{in} \\
& \ottkw{let} \, \ottmv{err}  \ottsym{=}  \ottkw{switch} \, \ottmv{args}  \ottsym{.}  \ottmv{err} \, \ottkw{as} \, \ottmv{err} \, \ottkw{case} \, \ottkw{IO}  \Rightarrow  \ottmv{err} \, \ottkw{case} \, \ottkw{Null}  \Rightarrow  \ottmv{out} \, \ottkw{in}
\end{align*}

\subsection{Recovering Type Safety}

The type safety of our translation scheme is essentially guaranteed by call site
rewriting. Besides adding null values for absent optional arguments, we also
sanitize arguments to ensure that they are expected from the parameter list.
Since named arguments are first-class and can be passed as a variable, we may
not have literals like \lstinline{append(x=1, xs=[0])} but splats like
\lstinline{append(**args)}. So the matching between (formal) parameters and
(actual) arguments is performed based on their static types:
\begin{itemize}
\item If \lstinline{args} has type $\ottsym{\{}  \ottmv{x}  \ottsym{:}   \mathbb{Z}   \ottsym{\}}  \land  \ottsym{\{}  \ottmv{xs}  \ottsym{:}  \ottsym{[}   \mathbb{Z}   \ottsym{]}  \ottsym{\}}$, the call site will be
      rewritten to something equivalent to \lstinline{append(x=args.x, xs=args.xs)}.
\item If \lstinline{args} only has type $\ottsym{\{}  \ottmv{x}  \ottsym{:}   \mathbb{Z}   \ottsym{\}}$, the call site will be
      rewritten to something equivalent to \lstinline{append(x=args.x, xs=null)}.
\end{itemize}
For the \lstinline{append} function, no other cases can pass the sanitization process.

Let us review the previous type-unsafe Python example in \autoref{sec:mypy}:
\begin{lstlisting}[language={[3]Python}]
def f(args: In) -> Out: return args
args = f({ "host": "0.0.0.0", "port": 80, "debug": "Oops!" })
app.run(**args) #= app.run(host="0.0.0.0", port=80, debug="Oops!")
\end{lstlisting}
Recall that \lstinline{args} has type \lstinline{Out}, which is similar to
\lstinline[language={[3]Python}]|{ host: str, port: int }|. The \lstinline{debug} key is forgotten in
the static type but is still present at run time. It passes mypy's type checking
but raises a runtime error. In our translation scheme, the call site will be
rewritten to the following form based on the type of \lstinline{args} (i.e.
\lstinline{Out}):
\begin{lstlisting}
app.run(host=args.host, port=args.port, debug=null)
\end{lstlisting}
Therefore, type safety is recovered in our translation scheme.

\paragraph{Takeaways.}
There are two important observations from our translation scheme:
\begin{enumerate}
\item \lstinline|{ required: A, optional?: B }| is not equivalent to
      \lstinline|{ required: A }| because the former contains more information
      that prevents \lstinline{optional} from being associated with other types
      than \lstinline{B}. In other words, the \lstinline{optional} argument can
      be absent, but if it is present, it must have type \lstinline{B}.
\item Corresponding to the above observation at the type level, we explicitly
      pass a null value as an optional argument if it is statically missing. The
      null value fills the position of a potentially forgotten argument that may
      have a wrong type. In other words, we implement the splat operator as per
      the static type of named arguments.
\end{enumerate}

\subsection{Implementation in the CP Language}

Our approach to named and optional arguments has been implemented in the CP
language. CP supports not only intersection and union types but also the merge
operator and type-based switch expression. The implementation of named and
optional arguments in CP is a direct application of our translation scheme.

More interestingly, the sanitization process during call site rewriting comes
for free because CP employs a \emph{coercive} semantics for
subtyping~\citep{luo2013coercive}. For example, a subtyping relation between
\lstinline[language={[3]Python}]|{ host: str, port: int, debug: str }| and
\lstinline[language={[3]Python}]|{ host: str, port: int }| implies a coercion
function from subtype to supertype. In CP, such coercions are implicitly
inserted to remove the forgotten fields (e.g. \lstinline{debug} in this case).
Therefore, the only remaining work is to add null values for absent optional
arguments.

To demonstrate the use of named and optional arguments in CP, we show a fractal
example in \autoref{fig:fractal}, which is adapted from code in
\autoref{sec:fractal}. The code makes use of named and optional arguments a lot,
including both the \lstinline{SVG}/\lstinline{Rect} constructors from the
library and the \lstinline{fractal} function defined by the client. For example,
\lstinline{fractal} has five named arguments (\lstinline{level}, \lstinline{x},
\lstinline{y}, \lstinline{width}, and \lstinline{height}), among which
\lstinline{level} is optional with a default value of \lstinline{4}.

\begin{figure}[t]
\begin{lstlisting}[language=CP,xleftmargin=.05\textwidth]
-- from a SVG library in CP
SVG: { width: Int; height: Int } -> [Element] -> Graphic;    -- <svg>
Rect: { x: Int; y: Int; width: Int; height: Int
      ; rx?: Int; ry?: Int; color?: String } -> Element;     -- <rect>
......
-- client code
fractal { level = 4; x: Int; y: Int; width: Int; height: Int } =
  let args = { level = level-1; width = width/3; height = height/3 } in
  let center = Rect (args,{ x = x + width/3; y = y + height/3
                          ; color = "white" }) in
  if level == 0 then [center]
  else fractal (args,{ x = x;             y = y })
    ++ fractal (args,{ x = x + width/3;   y = y })
    ++ fractal (args,{ x = x + width*2/3; y = y })
    ++ fractal (args,{ x = x;             y = y + height/3 })
    ++ [center]
    ++ fractal (args,{ x = x + width*2/3; y = y + height/3 })
    ++ fractal (args,{ x = x;             y = y + height*2/3 })
    ++ fractal (args,{ x = x + width/3;   y = y + height*2/3 })
    ++ fractal (args,{ x = x + width*2/3; y = y + height*2/3 });
init = { x = 0; y = 0; width = 600; height = 600; color = "black" };
main = SVG init ([Rect init] ++ fractal init);
\end{lstlisting}
\caption{Sierpiński carpets implemented in the CP language.} \label{fig:fractal}
\end{figure}

It is worth noting that named arguments are used as first-class values in the CP
code. On the first line of the \lstinline{fractal} body, we store three fields
\lstinline{level}, \lstinline{width}, and \lstinline{height} in a variable
\lstinline{args}, which are shared arguments for later calls. When constructing
the center rectangle, we merge \lstinline{args} with three more fields
\lstinline{x}, \lstinline{y}, and \lstinline{color} to form a full set of named
arguments we need for the \lstinline{Rect} constructor. When recursively calling
\lstinline{fractal}, we pass \lstinline{args} merged with different
\lstinline{x} and \lstinline{y} values to draw the eight sub-copies. In the
\lstinline{main} function, we also use a variable \lstinline{init} to avoid
repeating the same set of arguments for \lstinline{SVG}, \lstinline{Rect}, and
\lstinline{fractal} calls. The \lstinline{**} operator is not needed in CP when
passing first-class named arguments. Note that the parameter lists of these
three constructors/functions are not completely the same, but we can still use a
larger set of named arguments to cover all the cases. This is possible because
CP allows subtyping for named arguments while retaining type safety.

\subsection{Applications to Other Languages}

Although we base our translation scheme on a core language with intersection and
union types for type-theoretic solidness and elegance, it can work for a wider
range of languages. We discuss the alternatives to intersections and unions
below.

\paragraph{Alternative to intersections.}
Record types have existed long before intersection types were invented. In
practice, multi-field records are rarely represented as intersections of
single-field records. For example, \emph{Software Foundations}~\citep{plf}
demonstrates how to directly model multi-field records and define depth, width,
and permutation subtyping without intersections, though their formalization is
more complex than ours.

There is a merge operator in our translation scheme, but we only use it to
construct multi-field records statically. Although the merge operator can be
powerful if we want to construct first-class named arguments at run time like in
CP, its absence does not disable our translation scheme. In other words, we only
assume a simplified version that does not merge terms dynamically.

\paragraph{Alternative to unions.}
Nullable types are rarely represented as unions with the null type too. For
example, C\#, Kotlin, and Dart support nullable types as a primitive data
structure. Putting a question mark behind any type makes it nullable in these
languages (e.g. \lstinline{int?}).

No matter how a nullable type is represented, there is usually some expression
that can check whether a nullable value is null or not. For example, C\#
provides the \lstinline[language={[Sharp]C}]{is} operator to examine the runtime
type, which is generally known as type introspection and is similar to the
type-based switch. C\# also provides the null coalescing operator \lstinline{??}
and simplifies the common pattern $\ottkw{switch} \, \ottnt{e} \, \ottkw{as} \, \ottmv{x} \, \ottkw{case} \, \ottnt{A}  \Rightarrow  \ottmv{x} \, \ottkw{case} \, \ottkw{Null}  \Rightarrow  \ottnt{d}$
as \lstinline{e??d} for nullable values.

\paragraph{Dynamically typed languages.}
It may be surprising at first sight that dynamically typed languages can benefit
from our work with static typing, but recall that the type-safety issue in
\autoref{sec:mypy} was found in Python. Nowadays, popular dynamically typed
languages have been retrofitted with gradual typing. For example, Python has
type hints and mypy~\citep{mypy}, Ruby has RBS and Steep~\citep{steep},
JavaScript gets typed by TypeScript~\citep{typescript}, and Lua gets typed by
Luau~\citep{luau}. All of these typed versions support record-like and union
types, and all except Python also support intersection types. Our translation
scheme can almost directly apply to these languages. For a concrete example, we
show how the aforementioned \lstinline{exp} function can be encoded in
TypeScript:
\begin{lstlisting}[language=TypeScript]
function exp(args: { x: number } & { base: number|null }) {
  let x = args.x;
  let base = (typeof args.base === "number") ? args.base : Math.E;
  return Math.pow(base, x);
}
exp({ base: 2, x: 10 })     //= 1024
exp({ x: 10, base: null })  //= e^10
\end{lstlisting}
The code is almost the same as in \autoref{sec:example}. Note that the
\lstinline[language=TypeScript]{typeof} operator is the standard way to perform
type introspection in TypeScript, and the type of \lstinline{args.base} is
refined from the \lstinline[language=TypeScript]{number|null} to
\lstinline[language=TypeScript]{number} in the true-branch. We assume the call
sites have been rewritten in the code above. In this manner, named and optional
arguments can be added to TypeScript as syntactic sugar.

Although we have discussed several alternatives to intersection and union types,
we believe that if a language is designed from scratch, our approach is a good
choice. Intersection and union types not only subsume multi-field record and
nullable types but also provide a solid and elegant foundation for other
advanced features, such as function overloading and heterogeneous data
structures. The essay by \citet{castagna2023programming} is an excellent further
reading on the beauty of programming with intersection and union types.

\section{Formalization} \label{sec:iu-uaena}

In this section, we formalize the translation of named and optional arguments as
an elaboration semantics. The target of elaboration is called \lambdaiu, and the
source is called \uaena. We prove that the source language with named and
optional arguments is type-safe via (1) the type soundness of the target
calculus and (2) the type soundness of elaboration. All the theorems are
mechanically proven using the Coq proof assistant.

\subsection{The Target Calculus: \lambdaiu} \label{sec:lambdaiu}

\lambdaiu is an extension to the calculus in Chapter 5 of the dissertation by
\citet{rehman2023blend} with $\ottkw{null}$, single-field records, and let-in
bindings. The addition of let-in bindings is not essential because they can be
desugared into lambda abstractions and applications:
\begin{equation*}
  \ottkw{let} \, \ottmv{x}  \ottsym{=}  \ottnt{e_{{\mathrm{1}}}} \, \ottkw{in} \, \ottnt{e_{{\mathrm{2}}}} \quad \equiv \quad (\lambda x.\ e_2)\ e_1
\end{equation*}
However, we still have let-in bindings for the sake of readability, and this
form of let-in bindings simplifies the rules of parameter elaboration
(introduced later in \autoref{fig:elab}). Another difference is that the
original calculus uses the locally nameless
representation~\citep{chargueraud2012locally} while ours directly uses names for
bound variables.

Our changes to Rehman's calculus are relatively trivial, and we do not touch the
rules for intersection and union types. We will not discuss his design choices
in this paper, because our focus is on the type soundness with the addition of
$\ottkw{Null}$ and record types. We have proven in Coq that these extensions
preserve type soundness.

\subsubsection{Syntax of \lambdaiu}
\begin{align*}
  &\text{Types}          &\ottnt{A},\ottnt{B} ::=&~  \top  ~|~  \bot  ~|~ \ottkw{Null} ~|~  \mathbb{Z}  ~|~ \ottnt{A}  \rightarrow  \ottnt{B} ~|~ \ottsym{\{}  \ell  \ottsym{:}  \ottnt{A}  \ottsym{\}} ~|~ \ottnt{A}  \land  \ottnt{B} ~|~ \ottnt{A}  \lor  \ottnt{B} \\
  &\text{Expressions}    &      \ottnt{e} ::=&~ \ottsym{\{\}} ~|~ \ottkw{null} ~|~  n  ~|~ \ottmv{x} ~|~  \lambda \ottmv{x} \!:\! \ottnt{A} .\; \ottnt{e} \!:\! \ottnt{B}  ~|~ \ottnt{e_{{\mathrm{1}}}} \, \ottnt{e_{{\mathrm{2}}}} ~|~ \ottsym{\{}  \ell  \ottsym{:}  \ottnt{A}  \ottsym{=}  \ottnt{e}  \ottsym{\}} ~|~ \ottnt{e}  \ottsym{.}  \ell \\
  &                      &              |&~ \ottnt{e_{{\mathrm{1}}}}  \bbcomma  \ottnt{e_{{\mathrm{2}}}} ~|~ \ottkw{switch} \, \ottnt{e_{{\mathrm{0}}}} \, \ottkw{as} \, \ottmv{x} \, \ottkw{case} \, \ottnt{A}  \Rightarrow  \ottnt{e_{{\mathrm{1}}}} \, \ottkw{case} \, \ottnt{B}  \Rightarrow  \ottnt{e_{{\mathrm{2}}}} ~|~ \ottnt{letin} \, \ottnt{e}
\end{align*}
The types include the top type $ \top $, the bottom type $ \bot $, the null
type $\ottkw{Null}$, the integer type $ \mathbb{Z} $, function types $\ottnt{A}  \rightarrow  \ottnt{B}$, record
types $\ottsym{\{}  \ell  \ottsym{:}  \ottnt{A}  \ottsym{\}}$, intersection types $\ottnt{A}  \land  \ottnt{B}$, and union types $\ottnt{A}  \lor  \ottnt{B}$.
$\ottkw{Null}$ is a unit type that has only one value $\ottkw{null}$.

The expressions include the empty record $\ottsym{\{\}}$, the null value $\ottkw{null}$,
integer literals $ n $, variables $\ottmv{x}$, lambda abstractions $ \lambda \ottmv{x} \!:\! \ottnt{A} .\; \ottnt{e} \!:\! \ottnt{B} $,
function applications $\ottnt{e_{{\mathrm{1}}}} \, \ottnt{e_{{\mathrm{2}}}}$, record literals $\ottsym{\{}  \ell  \ottsym{:}  \ottnt{A}  \ottsym{=}  \ottnt{e}  \ottsym{\}}$, record
projections $\ottnt{e}  \ottsym{.}  \ell$, merges $\ottnt{e_{{\mathrm{1}}}}  \bbcomma  \ottnt{e_{{\mathrm{2}}}}$, type-based switch expressions
$\ottkw{switch} \, \ottnt{e_{{\mathrm{0}}}} \, \ottkw{as} \, \ottmv{x} \, \ottkw{case} \, \ottnt{A}  \Rightarrow  \ottnt{e_{{\mathrm{1}}}} \, \ottkw{case} \, \ottnt{B}  \Rightarrow  \ottnt{e_{{\mathrm{2}}}}$, and let-in bindings $\ottnt{letin} \, \ottnt{e}$.
The syntax of $\ottnt{letin}$ is as follows:
\begin{equation*}
  \ottnt{letin} ::= \ottkw{let} \, \ottmv{x}  \ottsym{=}  \ottnt{e} \, \ottkw{in} ~|~  \ottnt{letin_{{\mathrm{1}}}} \circ \ottnt{letin_{{\mathrm{2}}}}  ~|~ \ottkw{id}
\end{equation*}
The composition of two let-in bindings is denoted by $ \ottnt{letin_{{\mathrm{1}}}} \circ \ottnt{letin_{{\mathrm{2}}}} $, and an
empty binding is denoted by $\ottkw{id}$.

\paragraph{Subtyping.}
\autoref{fig:sub} shows the subtyping rules of \lambdaiu. The rules are standard
for a type system with intersection and union types. \Rref{Sub-Top} shows that
the top type $ \top $ is a supertype of any type, and \rref{Sub-Bot} shows that
the bottom type $ \bot $ is a subtype of any type.
\Rref{Sub-And,Sub-AndL,Sub-AndR} handle the subtyping for intersection types,
while \rref{Sub-Or,Sub-OrL,Sub-OrR} are for union types. \Rref{Sub-Null,Sub-Rcd}
added by us are straightforward. We prove that the subtyping relation is
reflexive and transitive.

\begin{theorem}[Subtyping Reflexivity]
  $\forall A, \ottnt{A}  \ottsym{<:}  \ottnt{A}$.
\end{theorem}
\begin{theorem}[Subtyping Transitivity]
  If $\ottnt{A}  \ottsym{<:}  \ottnt{B}$ and $\ottnt{B}  \ottsym{<:}  \ottnt{C}$, then $\ottnt{A}  \ottsym{<:}  \ottnt{C}$.
\end{theorem}

\begin{figure}[b!]
\IUdefnsub{}
\caption{Subtyping of \lambdaiu.} \label{fig:sub}
\end{figure}

\begin{figure}
\begin{align*}
  &\text{Typing contexts}&\Gamma ::=&~  \cdot  ~|~ \Gamma  ,\,  \ottmv{x}  \ottsym{:}  \ottnt{A}
\end{align*}
\IUdefntyping{}
\IUdefnletbind{}
\caption{Typing of \lambdaiu.} \label{fig:typ}
\end{figure}

\paragraph{Typing.}
\autoref{fig:typ} shows the typing rules of \lambdaiu. The empty record $\ottsym{\{\}}$
has the top type $ \top $, as shown in \rref{Typ-Top}. \Rref{Typ-Merge} is the
introduction rule for intersection types. Merging two functions is used for
function overloading, and merging two records is used for record concatenation.
\Rref{Typ-Switch} is the elimination rule for union types. The type-based switch
expression scrutinizes an expression having a union of the two scrutinizing
types (i.e. $\ottnt{e_{{\mathrm{0}}}}:\ottnt{A}  \lor  \ottnt{B}$). This premise ensures the exhaustiveness of the
cases in the switch. The $\ottkw{as}$-variable $\ottmv{x}$ is refined to type
$\ottnt{A}$ in $\ottnt{e_{{\mathrm{1}}}}$ and to type $\ottnt{B}$ in $\ottnt{e_{{\mathrm{2}}}}$.
\Rref{Typ-Null,Typ-Rcd,Typ-Prj} added by us are straightforward. \Rref{Typ-Let}
uses an auxiliary judgment $\Gamma  \,\vdash\,  \ottnt{letin}  \,\dashv\,  \Gamma'$ to obtain the typing context
for the body of the let-in binding. For example, if $\ottnt{e_{{\mathrm{1}}}}$ has type $\ottnt{A}$,
then $\ottkw{let} \, \ottmv{x}  \ottsym{=}  \ottnt{e_{{\mathrm{1}}}} \, \ottkw{in} \, \ottnt{e_{{\mathrm{2}}}}$ adds $\ottmv{x}:\ottnt{A}$ to the typing context before
type-checking $\ottnt{e_{{\mathrm{2}}}}$.

\paragraph{Dynamic semantics.}
We have a small-step operational semantics for \lambdaiu. The judgment
$\ottnt{e}\longrightarrow\ottnt{e'}$ means that $\ottnt{e}$ reduces to $\ottnt{e'}$ in one step,
and $\ottnt{e}\longrightarrow^*\ottnt{e'}$ is for multi-step reduction. We extend the
original dynamic semantics by adding rules for records and projections.
Similarly to the applicative dispatch for function applications in the original
calculus, we add a relation called projective dispatch for record projections.
For example, $\ottsym{(}  \ottsym{\{}  \ottmv{x}  \ottsym{=}  \ottsym{1}  \ottsym{\}}  \bbcomma  \ottsym{\{}  \ottmv{y}  \ottsym{=}  \ottsym{2}  \ottsym{\}}  \ottsym{)}  \ottsym{.}  \ottmv{x}$ reduces to $\ottsym{\{}  \ottmv{x}  \ottsym{=}  \ottsym{1}  \ottsym{\}}  \ottsym{.}  \ottmv{x}$ via
projective dispatch to select the needed field.

Since the dynamic semantics of \lambdaiu is independent of the elaboration from
\uaena to \lambdaiu, we omit the rules here but leave them in
\autoref{sec:iu-dyn}. Note that the operational semantics is not commonplace in
that it is type-directed and non-deterministic. Please refer to the dissertation
by \citet{rehman2023blend} for detailed explanations.

\begin{theorem}[Progress]
  If $ \cdot   \,\vdash\,  \ottnt{e}  \ottsym{:}  \ottnt{A}$, then either $\ottnt{e}$ is a value or $\exists \ottnt{e'}, \ottnt{e} \longrightarrow \ottnt{e'}$.
\end{theorem}
\begin{theorem}[Preservation]
  If $\Gamma  \,\vdash\,  \ottnt{e}  \ottsym{:}  \ottnt{A}$ and $\ottnt{e} \longrightarrow \ottnt{e'}$, then $\Gamma  \,\vdash\,  \ottnt{e'}  \ottsym{:}  \ottnt{A}$.
\end{theorem}
Putting progress and preservation together, we conclude that \lambdaiu is type-sound:
a well-typed term can never reach a stuck state.
\begin{corollary}[Type Soundness]
  If $ \cdot   \,\vdash\,  \ottnt{e}  \ottsym{:}  \ottnt{A}$ and $\ottnt{e} \longrightarrow^* \ottnt{e'}$, then either $\ottnt{e'}$ is a value or $\exists \ottnt{e''}, \ottnt{e'} \longrightarrow \ottnt{e''}$.
\end{corollary}

\subsection{The Source Calculus: \uaena} \label{sec:uaena}

\uaena (\emph{Unnamed Arguments Extended with Named Arguments})
is a minimal calculus with named and optional arguments. Although the
calculus is small, named arguments are supported as first-class values and can
be passed to or returned by a function. Besides functions with named
arguments, \uaena also supports normal functions with positional
arguments. The two kinds of functions are distinguished in the syntax, as seen in
Ruby, Racket, OCaml, etc.

\subsubsection{Syntax of \uaena}
\begin{align*}
  &\text{Types}                 &\mathcal{A},\mathcal{B} ::=&~  \mathbb{Z}  ~|~  ( \mathcal{A} ) \rightarrow \mathcal{B}  ~|~ \ottsym{\{}  \mathcal{P}  \ottsym{\}}  \rightarrow  \mathcal{B} ~|~ \ottsym{\{}  \mathcal{K}  \ottsym{\}} \\
  &\text{Named parameter types} &        \mathcal{P} ::=&~  \cdot  ~|~ \mathcal{P}  ;\,  \ell  \ottsym{:}  \mathcal{A} ~|~ \mathcal{P}  ;\,  \ell  \ottsym{\mbox{?}}  \ottsym{:}  \mathcal{A} \\
  &\text{Named argument types}  &        \mathcal{K} ::=&~  \cdot  ~|~ \mathcal{K}  ;\,  \ell  \ottsym{:}  \mathcal{A} \\
  &\text{Expressions}           &       \epsilon ::=&~  n  ~|~ \ottmv{x} ~|~  \lambda ( \ottmv{x} \!:\! \mathcal{A} ).\; \epsilon  ~|~  \lambda \{ \rho \}.\; \epsilon  ~|~ \epsilon_{{\mathrm{1}}} \, \epsilon_{{\mathrm{2}}} ~|~ \ottsym{\{}  \kappa  \ottsym{\}} \\
  &\text{Named parameters}      &        \rho ::=&~  \cdot  ~|~ \rho  ;\,  \ell  \ottsym{:}  \mathcal{A} ~|~ \rho  ;\,  \ell  \ottsym{=}  \epsilon \\
  &\text{Named arguments}       &        \kappa ::=&~  \cdot  ~|~ \kappa  ;\,  \ell  \ottsym{=}  \epsilon
\end{align*}
The types include the integer type $ \mathbb{Z} $, normal function types $ ( \mathcal{A} ) \rightarrow \mathcal{B} $,
function types with named parameters $\ottsym{\{}  \mathcal{P}  \ottsym{\}}  \rightarrow  \mathcal{B}$, and (first-class)
named argument types $\ottsym{\{}  \mathcal{K}  \ottsym{\}}$. The expressions include integer literals
$ n $, variables $\ottmv{x}$, normal lambda abstractions $ \lambda ( \ottmv{x} \!:\! \mathcal{A} ).\; \epsilon $,
lambda abstractions with named parameters $ \lambda \{ \rho \}.\; \epsilon $, function applications
$\epsilon_{{\mathrm{1}}} \, \epsilon_{{\mathrm{2}}}$, and (first-class) named arguments $\ottsym{\{}  \kappa  \ottsym{\}}$.

A named parameter type $\mathcal{P}$ can be required ($\ell:\mathcal{A}$) or optional
($\ell?:\mathcal{A}$). If a named parameter is optional, its default value must be
provided in the function definition. For example,
$\lambda\{\ottmv{x}: \mathbb{Z} ;\;\ottmv{y}=0\}.\;\ottmv{x}  \ottsym{+}  \ottmv{y}$ has type
$\{\ottmv{x}: \mathbb{Z} ;\;\ottmv{y}?: \mathbb{Z} \} \rightarrow  \mathbb{Z} $. A function with named
parameters can only be applied to named arguments, which are basically a list of
key-value pairs. For example, the previous function can be applied to
$\{\ottmv{x}=1;\;\ottmv{y}=2\}$ or $\{\ottmv{x}=1\}$ or a variable having a compatible type.
The variable case demonstrates the first-class nature of named arguments in \uaena.

% manually break line for the conclusion
\renewcommand{\IUdrulePElaXXOptional}[1]{\ottdrule[#1]{%
\ottpremise{ \Delta \,\vdash\, _{\!\!\!\! \ottmv{x} }\:\: \rho \ottsym{:} \mathcal{P} \,\rightsquigarrow\, \ottnt{letin} \,\dashv\, \Delta' }%
\ottpremise{\Delta'  \,\vdash\,  \epsilon  \ottsym{:}  \mathcal{A}  \,\rightsquigarrow\,  \ottnt{e}}%
}{
\Delta\,\vdash_{\!\ottmv{x}}\,\ottsym{(}  \rho  ;\,  \ell  \ottsym{=}  \epsilon  \ottsym{)}:\ottsym{(}  \mathcal{P}  ;\,  \ell  \ottsym{\mbox{?}}  \ottsym{:}  \mathcal{A}  \ottsym{)}\\
 \,\rightsquigarrow\,  \ottnt{letin} \circ \ottkw{let} \, \ell  \ottsym{=}  \ottkw{switch} \, \ottmv{x}  \ottsym{.}  \ell \, \ottkw{as} \, \ottmv{y} \, \ottkw{case} \, |  \mathcal{A}  |  \Rightarrow  \ottmv{y} \, \ottkw{case} \, \ottkw{Null}  \Rightarrow  \ottnt{e} \, \ottkw{in}  \,\dashv\, \Delta'  ,\,  \ell  \ottsym{:}  \mathcal{A}}{%
{\ottdrulename{PEla\_Optional}}{}%
}}

\begin{figure}
\begin{align*}
  &\text{Typing contexts}&\Delta ::=&~  \cdot  ~|~ \Delta  ,\,  \ottmv{x}  \ottsym{:}  \mathcal{A}
\end{align*}
\IUdefnelab{}
\IUdefnpelab{}
\caption{Type-directed elaboration from \uaena to \lambdaiu.} \label{fig:elab}
\end{figure}

Careful readers may notice that a named argument type can also serve as the
parameter of a normal function. This also demonstrates the first-class nature of
named arguments. But note that a normal function that takes named
arguments is different from a function with named parameters. Consider the
following two functions, the former of which is a function with named parameters
and the latter is a normal function:
\begin{align*}
                    (\lambda\{\ottmv{x}: \mathbb{Z} ;\;\ottmv{y}=0\}.\;\ottmv{x}  \ottsym{+}  \ottmv{y}) \quad&:\quad \{\ottmv{x}: \mathbb{Z} ;\;\ottmv{y}?: \mathbb{Z} \} \rightarrow  \mathbb{Z}  \\
  (\lambda(\ottmv{args}:\{\ottmv{x}: \mathbb{Z} ;\;\ottmv{y}: \mathbb{Z} \}).\;\ottmv{args}) \quad&:\quad (\{\ottmv{x}: \mathbb{Z} ;\;\ottmv{y}: \mathbb{Z} \}) \rightarrow \{\ottmv{x}: \mathbb{Z} ;\;\ottmv{y}: \mathbb{Z} \}
\end{align*}
Although both functions can be applied to $\{\ottmv{x}=1;\;\ottmv{y}=2\}$, there are two
main differences between them. First, optional parameters cannot be defined in a
normal function. So we cannot provide $\ottmv{y}=0$ as a default value in the second
function. Second, $\ottmv{x}$ and $\ottmv{y}$ are not brought into the scope of the
function body in a normal function. So the only accessible variable is $\ottmv{args}$
in the second function.

\paragraph{Elaboration.}
The type-directed elaboration from \uaena to \lambdaiu is defined at the top of
\autoref{fig:elab}. $\Delta  \,\vdash\,  \epsilon  \ottsym{:}  \mathcal{A}  \,\rightsquigarrow\,  \ottnt{e}$ means that the source
expression $\epsilon$ has type $\mathcal{A}$ and elaborates to the target expression
$\ottnt{e}$ under the typing context $\Delta$. \Rref{Ela-Abs,Ela-App} for normal
functions are straightforward. In \rref{Ela-NAbs} for functions with named
parameters, besides inferring the type of the function body $\epsilon$ and
elaborating it to $\ottnt{e}$, we generate let-bindings for the named parameters,
which is delegated to the auxiliary judgment $ \Delta \,\vdash\, _{\!\!\!\! \ottmv{x} }\:\: \rho \ottsym{:} \mathcal{P} \,\rightsquigarrow\, \ottnt{letin} \,\dashv\, \Delta' $.
In \rref{Ela-NApp}, there is also an auxiliary judgment $ \Delta \,\vdash\, _{\!\!\!\! \ottnt{e} }\:\: \mathcal{P} \,\diamond\, \mathcal{K} \,\rightsquigarrow\, \ottnt{e'} $
that rewrites call sites according to the parameter and argument types.
\Rref{Ela-NEmpty,Ela-NField} are used to elaborate named arguments.

\paragraph{Named parameter elaboration.}
As shown at the bottom of \autoref{fig:elab},
$ \Delta \,\vdash\, _{\!\!\!\! \ottmv{x} }\:\: \rho \ottsym{:} \mathcal{P} \,\rightsquigarrow\, \ottnt{letin} \,\dashv\, \Delta' $ means that the named parameter
$\rho$ is inferred to have type $\mathcal{P}$ and elaborates to a series of let-in
bindings $\ottnt{letin}$, given that the named parameters correspond to the target
bound variable $\ottmv{x}$. In the meanwhile, the typing context $\Delta$ is extended
with the types of the named parameters to form $\Delta'$. $\Delta'$ is used for
typing the body of the function with named parameters. \Rref{PEla-Required}
simply generates $\ottkw{let} \, \ell  \ottsym{=}  \ottmv{x}  \ottsym{.}  \ell \, \ottkw{in}$, while \rref{PEla-Optional} generates
$\ottkw{let} \, \ell  \ottsym{=}  \ottkw{switch} \, \ottmv{x}  \ottsym{.}  \ell \, \ottkw{as} \, \ottmv{y} \, \ottkw{case} \, |  \mathcal{A}  |  \Rightarrow  \ottmv{y} \, \ottkw{case} \, \ottkw{Null}  \Rightarrow  \ottnt{e} \, \ottkw{in}$ to provide a
default value $\ottnt{e}$ for the $\ottkw{Null}$ case.

\begin{figure}
\IUdefnpmatch{}
\IUdefnlookup{}
\IUdefnlookdown{}
\caption{Type-directed call site rewriting in \uaena.} \label{fig:match}
\end{figure}

\paragraph{Call site rewriting.}
As shown in \autoref{fig:match},
$ \Delta \,\vdash\, _{\!\!\!\! \ottnt{e} }\:\: \mathcal{P} \,\diamond\, \mathcal{K} \,\rightsquigarrow\, \ottnt{e'} $ means that if the parameter type
$\mathcal{P}$ is compatible with the argument type $\mathcal{K}$, the target expression
$\ottnt{e}$, which corresponds to the named arguments, will be rewritten to
$\ottnt{e'}$. The compatibility check is based on the parameter type $\mathcal{P}$.
\Rref{PMat-Required} handles the case where the argument is required, while
\rref{PMat-Present,PMat-Absent} handle the cases where the optional argument
with a specific type is present and where the optional argument is absent,
respectively. The remaining case, where the optional argument is present but
associated with a wrong type, is prohibited and cannot elaborate to any term.
We have two more auxiliary judgments $ \mathcal{K} :: \ell \Rightarrow \mathcal{A} $ and
$ \mathcal{K} :: \ell \nRightarrow $ to indicate that the argument type $\mathcal{K}$ contains a field $\ell$
with type $\mathcal{A}$ or $\mathcal{K}$ does not contain $\ell$.

\paragraph{Type translation.}
As we have informally mentioned in \autoref{sec:core}, we translate named
parameters to intersection types and optional parameters to union types. The
rules for $|\cdot|$ can be found in \autoref{fig:trans-iu}. Having
defined the translation, we can prove the soundness of call site rewriting and
elaboration.

\begin{figure}
\framebox{$|  \mathcal{A}  |$} \quad Type translation
\begin{mathpar}
|   \mathbb{Z}   | \equiv  \mathbb{Z} 

|   ( \mathcal{A} ) \rightarrow \mathcal{B}   | \equiv |  \mathcal{A}  |  \rightarrow  |  \mathcal{B}  |

|  \ottsym{\{}  \mathcal{P}  \ottsym{\}}  \rightarrow  \mathcal{B}  | \equiv |  \mathcal{P}  |  \rightarrow  |  \mathcal{B}  |

|  \ottsym{\{}  \mathcal{K}  \ottsym{\}}  | \equiv  | \mathcal{K} | 
\end{mathpar}

\noindent
\framebox{$|  \mathcal{P}  |$} \quad Parameter type translation
\begin{mathpar}
|   \cdot   | \equiv  \top 

|  \mathcal{P}  ;\,  \ell  \ottsym{:}  \mathcal{A}  | \equiv |  \mathcal{P}  |  \land  \ottsym{\{}  \ell  \ottsym{:}  |  \mathcal{A}  |  \ottsym{\}}

|  \mathcal{P}  ;\,  \ell  \ottsym{\mbox{?}}  \ottsym{:}  \mathcal{A}  | \equiv |  \mathcal{P}  |  \land  \ottsym{\{}  \ell  \ottsym{:}  |  \mathcal{A}  |  \lor  \ottkw{Null}  \ottsym{\}}
\end{mathpar}

\noindent
\framebox{$ | \mathcal{K} | $} \quad Argument type translation
\begin{mathpar}
 |  \cdot  |  \equiv  \top 

 | \mathcal{K}  ;\,  \ell  \ottsym{:}  \mathcal{A} |  \equiv  | \mathcal{K} |   \land  \ottsym{\{}  \ell  \ottsym{:}  |  \mathcal{A}  |  \ottsym{\}}
\end{mathpar}

\noindent
\framebox{$|  \Delta  |$} \quad Typing context translation
\begin{mathpar}
|   \cdot   | \equiv  \cdot 

|  \Delta  ,\,  \ottmv{x}  \ottsym{:}  \mathcal{A}  | \equiv |  \Delta  |  ,\,  \ottmv{x}  \ottsym{:}  |  \mathcal{A}  |
\end{mathpar}
\caption{Type translation from \uaena to \lambdaiu.} \label{fig:trans-iu}
\end{figure}


\begin{theorem}[Soundness of Call Site Rewriting]
  If $ \Delta \,\vdash\, _{\!\!\!\! \ottnt{e} }\:\: \mathcal{P} \,\diamond\, \mathcal{K} \,\rightsquigarrow\, \ottnt{e'} $ and $|  \Delta  |  \,\vdash\,  \ottnt{e}  \ottsym{:}   | \mathcal{K} | $,
  then $|  \Delta  |  \,\vdash\,  \ottnt{e'}  \ottsym{:}  |  \mathcal{P}  |$.
\end{theorem}

\begin{theorem}[Soundness of Elaboration]
  If $\Delta  \,\vdash\,  \epsilon  \ottsym{:}  \mathcal{A}  \,\rightsquigarrow\,  \ottnt{e}$, then $|  \Delta  |  \,\vdash\,  \ottnt{e}  \ottsym{:}  |  \mathcal{A}  |$.
\end{theorem}

\noindent
With the two theorems above and the type soundness of \lambdaiu, we can
conclude that \uaena is type-safe.

\section{Discussion} \label{sec:related}

In this section, we first discuss OCaml, the only language we know of that has
well-studied support for named and optional arguments, though its mechanism does
not go well with higher-order functions. Then we briefly show how named
arguments are handled very differently in Scala and Racket. After that, we
illustrate how named arguments can be encoded as records in Haskell while not
natively supported. Finally, we discuss two more approaches we find in record
calculi~\citep{ohori1995polymorphic,osinski2006polymorphic}. We will also
explain why all these approaches have drawbacks.

\subsection{OCaml}

OCaml did not support named arguments originally. Nevertheless,
\citet{garrigue1994label} conducted research on the label-selective
$\lambda$-calculus and implemented it in OLabl~\citep{olabl}, which extends
OCaml with labeled and optional arguments, among others. All features of OLabl
were merged into OCaml 3, despite subtle
differences~\citep{garrigue2001labeled}.

Here is an example of the exponential function defined in a labeled style:

\begin{lstlisting}[language={[Objective]Caml}]
let exp ?(base = Float.exp 1.0) x = base ** x
(* val exp : ?base:float -> float -> float *)
exp 10.0              (*= e^10. *)
exp 10.0 ~base:2.0    (*= 1024. *)
(exp 10.0) ~base:2.0  (* TypeError! *)
\end{lstlisting}

\noindent In the definition of \lstinline{exp}, \lstinline{base} is an optional
labeled parameter while \lstinline{x} is a positional parameter. Changing
\lstinline{x} into a second labeled parameter will trigger an
unerasable-optional-argument warning because OCaml expects that there should be
a positional parameter after all optional parameters. This expectation is at the
heart of how OCaml resolves the ambiguity introduced by currying.

For example, consider the function application \lstinline{exp 10.0}. Is it a
partially applied function or a fully applied one using the default value of
\lstinline{base}? Both interpretations are possible, but OCaml considers it to
be a full application because the trailing positional argument \lstinline{x} is
given. The presence of the positional argument is used to indicate that the
optional arguments before it can be replaced by their default values. However,
this design may confuse users since \lstinline{(exp 10.0) ~base:2.0} will raise
a type error but \lstinline{exp 10.0 ~base:2.0} will not. Partial application
does not lead to an equivalent program in such situations.

\paragraph{Option Types.}
In OCaml, an optional argument is internally represented as an \lstinline{option}
type, which comprises two constructors: \lstinline{None} and \lstinline{Some}.
Here is an equivalent definition for \lstinline{exp}:

\begin{lstlisting}[language={[Objective]Caml}]
let exp ?(base : float option) x =
  let base = match base with
             | None -> Float.exp 1.0
             | Some b -> b in
  base ** x
(* val exp : ?base:float -> float -> float *)
exp 10.0            (*> exp 10.0 ~base:None *)
exp 10.0 ~base:2.0  (*> exp 10.0 ~base:(Some 2.0) *)
\end{lstlisting}

\noindent This encoding is similar to union types, but it depends on the
\lstinline{option} type in the standard library. Unfortunately, this specific
kind of \lstinline{option} is not a built-in type in many mainstream languages,
especially in those languages that do not support algebraic data types.

\paragraph{Higher-Order Functions.}
A surprising gotcha in OCaml is that the commutativity breaks down when we pass
a function with labeled arguments to another function. \emph{Real World
OCaml}~\citep{madhavapeddy2022real} gives the following example:
\begin{lstlisting}[language={[Objective]Caml}]
let apply1 f (fst,snd) = f ~fst ~snd
(* val apply1 : (fst:'a -> snd:'b -> 'c) -> 'a * 'b -> 'c *)
let apply2 f (fst,snd) = f ~snd ~fst
(* val apply2 : (snd:'a -> fst:'b -> 'c) -> 'b * 'a -> 'c *)
let divide ~fst ~snd = fst / snd
(* val divide : fst:int -> snd:int -> int *)
apply1 divide (48,3)  (*= 16 *)
apply2 divide (48,3)
(* TypeError: "divide" has type fst:int -> snd:int -> int
       but was expected of type snd:'a -> fst:'b -> 'c *)
\end{lstlisting}
Normally, the order of named arguments does not matter in OCaml, so it
type-checks whether we call \lstinline{divide ~fst ~snd} or
\lstinline{divide ~snd ~fst}. However, order matters when we pass
\lstinline{divide} to a higher-order function. That is why
\lstinline{apply1 divide} type-checks while \lstinline{apply2 divide} does not.
It turns out that the OCaml way of handling labeled arguments does not go well
with other features like higher-order functions. Our approach scales better in
this regard and the commutativity still holds in higher-order contexts via
intersection subtyping.

In short, OCaml has a very powerful label-selective core calculus that
reconciles commutativity and currying, but it is quite complicated and may
hinder its integration with other language features. Another thing worth
mentioning is that labeled arguments in OCaml are not first-class values, so
they cannot be assigned to a variable or passed around by functions. In contrast
to OCaml, our approach supports first-class named arguments and targets a
minimal core calculus with intersection and union types, which is compatible
with many popular languages like Python, Ruby, JavaScript, etc.

\subsection{Scala}

\citet{rytz2010named} described the design of named and default arguments in
Scala. Like in Python, parameter names in a method definition are non-mandatory
keywords in Scala, and thus every argument can be passed with or without
keywords. Furthermore, the parameter names are not part of the public interface
of a method. This design is partly due to the backward compatibility with
earlier versions of Scala, so the addition of named arguments will not break any
existing code. As a result of the conservative treatment, named arguments are
not first-class values in Scala and cannot be defined in an anonymous function.
In short, named arguments are more like syntactic sugar in Scala and do not
interact with the type system.

Below we show an example in Scala. In order to let the default value of
\lstinline{c} depend on \lstinline{a} and \lstinline{b}, we make the function
partly curried:
\begin{lstlisting}[language=Scala]
def f(a: Int, b: Int)(c: Int = a+b) = c
f(b = 1+1, a = 1)()
\end{lstlisting}
The code will be translated to equivalent code without keywords or defaults:
\begin{lstlisting}[language=Scala]
def f(a: Int, b: Int)(c: Int) = c
def f$default$3(a: Int, b: Int): Int = a+b
{
  val x$1 = 1+1
  val x$2 = 1
  val x$3 = f$default$3(x$2, x$1)
  f(x$2, x$1)(x$3)
}
\end{lstlisting}
There are two things to note here. First, a new function \lstinline{f$default$3}
is generated for the default value of \lstinline{c}, taking two parameters
\lstinline{a} and \lstinline{b}. Second, the call site is translated to a series
of variable assignments for each argument and a keyword-free call to
\lstinline{f} with arguments reordered. The whole call site is wrapped in a
block to avoid polluting the namespace.

In conclusion, named arguments in Scala are handled in a very different way from
OCaml and our approach. The Scala way is more syntactic than type-theoretic, so
it is hard to do an apples-to-apples comparison with our approach.

\subsection{Racket}

\lstdefinelanguage{Racket}[]{Lisp}{
  morekeywords={define,lambda,keyword-apply,Number},
  literate={\#:a}{{\textbf{\color{darkgray}\#:a}}}1 {\#:b}{{\textbf{\color{darkgray}\#:b}}}1 {\#:c}{{\textbf{\color{darkgray}\#:c}}}1
           {->*}{{$\rightarrow^*$}}1
}

\citet{flatt2009keyword} introduced keyword and optional arguments into Racket,
which was known as PLT Scheme at that time. A keyword is prefixed with
{\color{gray}\lstinline{#:}} in syntax and is implemented as a new built-in type
in Racket. Keyword arguments are supported by replacing
\lstinline[language=Racket]{define}, \lstinline[language=Racket]{lambda}, and
the core application form with newly defined macros that recognize
keyword-argument forms. Here is an example of a function \lstinline{f} with
three keyword arguments \lstinline{a}, \lstinline{b}, and \lstinline{c}, among
which \lstinline{c} is optional and defaults to \lstinline{a+b}:
\begin{lstlisting}[language=Racket]
(define (f #:a a #:b b #:c [c (+ a b)]) c)
(f #:b (+ 1 1) #:a 1)
\end{lstlisting}
The function call with keywords seems hard to implement because it just lists
the function and arguments in juxtaposition. In fact, an application form in
Racket implicitly calls \lstinline{#%app} in its lexical scope, so the support
for keyword arguments is done by supplying an \lstinline{#%app} macro. A new
\lstinline[language=Racket]{keyword-apply} function is also defined to accept
keyword arguments as first-class values. For example, we can rewrite the
function call above as follows\footnote{\lstinline[language=Racket]{'} is
\lstinline{quote}, \lstinline[language=Racket]{`} is \lstinline{quasiquote}, and
\lstinline[language=Racket]{,} is \lstinline{unquote} in Racket.}:
\begin{lstlisting}[language=Racket]
(keyword-apply f '(#:a #:b) `(1 ,(+ 1 1)) '()) ; OK!
(keyword-apply f '(#:b #:a) `(,(+ 1 1) 1) '()) ; Contract violation!
\end{lstlisting}
Note that we need to separate keywords and corresponding arguments into two
lists. The third list is for positional arguments, so it is empty in this case.
We cannot list keywords in arbitrary order: a contract violation will be
signaled unless the keywords are sorted in alphabetical order. In other words,
commutativity is lost for first-class keyword arguments in Racket.

In Typed Racket~\citep{typedracket}, Racket's gradually typed sister language,
\lstinline{f} can be typed as
\lstinline[language=Racket]{(->* (#:a Number #:b Number) (#:c Number) Number)}.
The first list contains the required arguments
(\lstinline[language=Racket]{#:a} and \lstinline[language=Racket]{#:b}), and the
second list contains the optional ones (\lstinline[language=Racket]{#:c}). However,
Typed Racket does not provide a typed version of
\lstinline[language=Racket]{keyword-apply}, and it is unclear how to properly type it.

In conclusion, Racket supports keyword and optional arguments in a unique way
via its powerful macro system. However, the support for first-class keyword arguments
is very limited and cannot easily transfer to a type-safe setting.

\subsection{Haskell}

\begin{figure}
\begin{subfigure}{0.45\textwidth}
\begin{lstlisting}[language=Haskell]
data Settings = Settings
  { settingsPort :: Port
  , settingsHost :: HostPreference
  , settingsTimeOut :: Int
  , ...
  }
\end{lstlisting}
\caption{Record type.} \label{fig:settings}
\end{subfigure}
\hfill
\begin{subfigure}{0.35\textwidth}
\begin{lstlisting}[language=Haskell]
defaultSettings = Settings
  { settingsPort = 3000
  , settingsHost = "*4"
  , settingsTimeout = 30
  , ...
  }
\end{lstlisting}
\caption{Default values.} \label{fig:default}
\end{subfigure}
\par\bigskip
\begin{subfigure}{\textwidth}
\begin{lstlisting}[language=Haskell]
runSettings :: Settings -> Application -> IO ()
runSettings = ...

main :: IO ()
main = runSettings settings app
  where settings = defaultSettings { settingsPort = 4000, settingsHost = "*6" }
\end{lstlisting}
\caption{Updating some settings before running a server application.} \label{fig:update}
\end{subfigure}
\caption{Named arguments as records in Haskell.}
\end{figure}

Unlike the aforementioned languages, Haskell does not support named arguments
natively. However, the paradigm of \emph{named arguments as records} has long
existed in the Haskell community. Although we have to uncurry a function to have
all parameters labeled in a record, it is clearer and more human-readable,
especially when different parameters have the same type. For example, in the web
server library \emph{warp}~\citep{warp}, various server settings are bundled in
the data type \lstinline{Settings}, as shown in \autoref{fig:settings}. It is
obvious how named arguments correspond to record fields, but it needs some
thought on how to encode default values for optional arguments. The simplest
approach, also used by \emph{warp}, is to define a record
\lstinline{defaultSettings}, as shown in \autoref{fig:default}. Users can update
whatever fields they want to change while keeping others. For example, we update
\lstinline{settingsPort} and \lstinline{settingsHost} while keeping the rest
unchanged in \autoref{fig:update}. Finally, we call the library function
\lstinline{runSettings} with the updated \lstinline{settings} to run a server.

Such an approach works fine here but still has two drawbacks. The first issue is
the dependency on \lstinline{defaultSettings}. It is awkward for users to look
for a record containing particular default values, especially when there are a
few similar records in a library. A better solution is to change the parameter
of \lstinline{runSettings} from a complete \lstinline{Settings} to a function
that updates \lstinline{Settings}:

\begin{lstlisting}[language=Haskell]
runSettings' :: (Settings -> Settings) -> Application -> IO ()
runSettings' update = runSettings (update defaultSettings)

main :: IO ()
main = runSettings' update app
  where update settings = settings { settingsPort = 4000, settingsHost = "*6" }
\end{lstlisting}

\noindent With the new interface, users do not need to look for default values
anymore. However, this design still has a second drawback: all arguments must
have default values. Usually, we do not consider every argument to be optional.
For example, we may want to require users to fill in \lstinline{settingsPort}. A
workaround employed by \lstinline{SqlBackend} in the library
\emph{persistent}~\citep{persistent} is to have another function that asks for
required arguments and supplements default values for optional arguments:

\begin{lstlisting}[language=Haskell]
{-# language DuplicateRecordFields, RecordWildCards #-}

data ReqSettings = ReqSettings { settingsPort :: Port }

mkSettings :: ReqSettings -> Settings
mkSettings ReqSettings {..} =
              Settings { settingsHost = "*4", settingsTimeout = 30, .. }
\end{lstlisting}

\paragraph{Best Practice.}
Although \lstinline{mkSettings} resolves the second issue, there is a regression
concerning the first issue: users have to look for \lstinline{mk*} functions
now. Fortunately, we can harmonize both design patterns to develop a third
approach:

\begin{lstlisting}[language=Haskell]
{-# language DuplicateRecordFields, RecordWildCards #-}

data OptSettings = OptSettings { settingsHost :: HostPreference
                               , settingsTimeOut :: Int }

runSettings'' :: (OptSettings -> Settings) -> Application -> IO ()
runSettings'' update = runSettings (update defaultSettings)
  where defaultSettings = OptSettings { settingsHost = "*4"
                                      , settingsTimeout = 30 }

main :: IO ()
main = runSettings'' update app
  where update OptSettings {..} =
                  Settings { settingsPort = 4000, settingsHost = "*6", .. }
\end{lstlisting}

\noindent This last approach is probably the best practice in Haskell, though it
is already quite complicated and requires two GHC language extensions. Of
course, there could be other approaches to encoding named and optional arguments
in Haskell. Users could get confused about the various available design
patterns. This is largely due to lack of language-level support. We believe it
is better for a language to natively support named and optional arguments.

\paragraph{Sidenote.}
The design pattern of \emph{named arguments as records} can also be found in
other functional languages like Standard ML, Elm, and PureScript, just to name a
few. It is worth mentioning that these languages have first-class support for
record types, so no separate type declarations are needed like in
\autoref{fig:settings}. However, they still suffer from lack of native support
for optional arguments.

\subsection{Record Calculi}

\citet{ohori1995polymorphic} discussed how to model optional arguments in the
future work of his seminal paper on compiling a polymorphic record calculus. He
proposed to extend a record calculus with optional-field selection
($e.\ell\;?\;d$) which behaves like $e.\ell$ if $\ell$ is present in the record
$e$ or evaluates to $d$ otherwise. However, his proposal is subject to a similar
type-safety issue as mypy. The static type of $e$ can easily lose track of the
optional field $\ell$ and fail to ensure that $e.\ell$ has the same type as $d$
at run time. Since \citeauthor{ohori1995polymorphic} did not explicitly mention
how to type-check optional-field selection, we cannot make any firm conclusion
about the type safety of his proposal.

\citet{osinski2006polymorphic} also discussed the support for optional arguments
in Section 3.5 of his dissertation on compiling record concatenation. His
approach is based on row polymorphism and makes use of a sort of predicate on
rows: $\mathit{row}_1\blacktriangleright\mathit{row}_2$, which means that
$\mathit{row}_1$ consists of all the fields in $\mathit{row}_2$. With this
predicate, a function has type
$\forall\rho.\;\mathit{row}_o\blacktriangleright\rho\Rightarrow\{\mathit{row}_r,\rho\}\to\tau$
if the required and optional arguments are denoted by $\mathit{row}_r$ and
$\mathit{row}_o$, respectively. Roughly speaking, it means the parameter has a
type between $\{\mathit{row}_r\}$ and $\{\mathit{row}_r,\mathit{row}_o\}$. At
the term level, he introduced a compatible concatenation operator
\lstinline{|&|}, which allows overlapping fields with the same types and prefers
the fields on the right-hand side when overlapping occurs. An example of their
translation is as follows:
\begin{lstlisting}[language=Caml]
fun add { x, y = 0 } = ...
(* is translated to *)
fun add r = let r' = { y = 0 } |&| r in
            let x = r'.x in
            let y = r'.y in ...
\end{lstlisting}
His approach is free from the type-safety issue, though based on a more
sophisticated row-polymorphic system. There are two sorts of predicates and
three variants of record concatenation operators in his calculus, for example,
demonstrating some sophistication of his calculus.

It is worth noting that neither \citeauthor{ohori1995polymorphic}'s nor
\citeauthor{osinski2006polymorphic}'s calculus supports subtyping. This is a
significant limitation since subtyping is a common feature in many popular
languages, especially object-oriented languages.

