\begin{figure}
\begin{multicols}{3}
\begin{lstlisting}[language=TypeScript,basicstyle=\ttfamily\scriptsize,showlines=true]
/* J-Nil */
var z = {};
J;

/* J-Opt */
var z = y || {};
J;

/* J-Int */
z[T] = n;

/* J-IntOpt */
var z = n;
if (y) y[T] = n;

/* J-IntNil */
var z = n;

/* J-Var */
copy(z, x);

/* J-VarOpt */
if (y) copy(y, x);

/* J-Fix */
var x = z;
J;

/* J-Abs */
z[T] = (x, y) => {
  J;  return y0;
};

/* J-TAbs */
z[T] = (X, y) => {
  J;  return y0;
};

/* J-Rcd */
z.__defineGetter__(T, () => {
  J;
  delete this[T];
  return this[T] = y;
});

/* J-Def */
export var x = {};
J1;  J2;

/* JA-Nil */
var z = {};
J;

/* JA-Opt */
var z = y || {};
J;

/* JA-Arrow */
var y0 = {};
J1;  J2;

/* JA-ArrowEquiv */
x[T](y, z);

/* JA-ArrowOpt */
var z = x[T](y, z0);

/* JA-ArrowNil */
var z = x[T](y);

/* JA-All */
x[T](Ts, z);

/* JA-AllOpt */
var z = x[T](Ts, y);

/* JA-All */
var z = x[T](Ts);

/* JP-RcdEq */
var z = x[T];

/* JS0-Int */
J;
y = y[T];

/* JS0-Var */
J;
if (primitive(X)) y = y[T];

/* JS-Equiv */
copy(y, x);

/* JS-Bot */
y[T] = null;

/* JS-Int */
y[T] = x;

/* JS-IntAnd */
y[T] = x[T];

/* JS-Var */
copy(y, x);

/* JS-Arrow */
y[T2] = (x1, y2) => {
  var y1 = {};  J1;
  var x2 = x[T1](y1);
  y2 = y2 || {};
  J2;  return y2;
};

/* JS-All */
y[T2] = (X, y0) => {
  var x0 = x[T1](X);
  y0 = y0 || {};
  J;  return y0;
};

/* JS-Rcd */
y.__defineGetter__(T2, () => {
  var x0 = x[T1];
  var y0 = {};  J;
  delete this[T];
  return this[T] = y0;
});

/* JS-Split */
var y1 = {}; // if y1 != z
var y2 = {}; // if y2 != z
J1;  J2;  J3;

/* JM-Arrow */
z[T] = (p, y) => {
  y = y || {};
  var y1 = {}; // if y1 != y
  var y2 = {}; // if y2 != y
  x1[T1](p, y1);
  x2[T2](p, y2);
  J;  return y;
};

/* JM-All */
z[T] = (X, y) => {
  y = y || {};
  var y1 = {}; // if y1 != y
  var y2 = {}; // if y2 != y
  x1[T1](X, y1);
  x2[T2](X, y2);
  J;  return y;
};

/* JM-Rcd */
z.__defineGetter__(T, () => {
  var y = {};
  var y1 = {}; // if y1 != y
  var y2 = {}; // if y2 != y
  copy(y1, x1[T1]);
  copy(y2, x2[T2]);
  J;
  delete this[T];
  return this[T] = y;
});


\end{lstlisting}
\end{multicols}
\caption{Predefined JavaScript code.} \label{fig:code}
\end{figure}

\noindent
The syntax of \fiplus is defined below, where the main differences from
\lambdaiplus are the addition of parametric polymorphism and fixpoint
expressions:
\begin{align*}
  &\text{Types}          &A,B ::=&~  \top  ~|~  \bot  ~|~  \mathbb{Z}  ~|~ \ottmv{X} ~|~ \ottnt{A}  \rightarrow  \ottnt{B} ~|~  \forall  \ottmv{X} \!*\! \ottnt{A} .\; \ottnt{B}  ~|~ \ottsym{\{}  \ell  \ottsym{:}  \ottnt{A}  \ottsym{\}} ~|~ \ottnt{A}  \, \& \,  \ottnt{B} \\
  &\text{Expressions}    &  e ::=&~ \ottsym{\{\}} ~|~  n  ~|~ \ottmv{x} ~|~  \ottkw{fix} \, \ottmv{x} \!:\! \ottnt{A} .\; \ottnt{e}  ~|~  \lambda \ottmv{x} \!:\! \ottnt{A} .\; \ottnt{e} \!:\! \ottnt{B}  ~|~ \ottnt{e_{{\mathrm{1}}}} \, \ottnt{e_{{\mathrm{2}}}} ~|~  \Lambda \ottmv{X} \!*\! \ottnt{A} .\; \ottnt{e} \!:\! \ottnt{B}  ~|~ \ottnt{e} \, \ottnt{A} \\
  &                      &      |&~ \ottsym{\{}  \ell  \ottsym{=}  \ottnt{e}  \ottsym{\}} ~|~ \ottnt{e}  \ottsym{.}  \ell ~|~  \ottnt{e_{{\mathrm{1}}}} \bbcomma \ottnt{e_{{\mathrm{2}}}}  ~|~ \ottnt{e}  \ottsym{:}  \ottnt{A}
\end{align*}

\noindent
The compilation scheme we describe here directly generates JavaScript code
instead of \lambdar terms, which is closer to the actual implementation. We
denote the generated JavaScript code by $\ottnt{J}$, which can be empty
($\varnothing$), concatenation of two pieces of code
($\ottnt{J}_1;\ottnt{J}_2$), or some predefined \texttt{code} that is listed in
\autoref{fig:code}. There are some notations for type indices, which are
actually implemented as strings in JavaScript (as discussed in
\autoref{sec:index}). Destinations~\citep{shaikhha2017destination} also play an
important role in the compilation scheme, being part of the rules for
type-directed compilation and distributive application. The key idea of
destinations has been elaborated in \autoref{sec:dst}.
\begin{align*}
  &\text{JavaScript code}&\ottnt{J} ::=&~  \varnothing  ~|~ \ottnt{J_{{\mathrm{1}}}}  \ottsym{;}  \ottnt{J_{{\mathrm{2}}}} ~|~ \mathtt{code} \\
  &\text{Type indices}&           T ::=&~  \mathbb{Z}  ~|~  \overrightarrow{ \ottnt{T} }  ~|~  \ottnt{T} ^\forall  ~|~ \ottsym{\{}  \ell  \ottsym{:}  \ottnt{T}  \ottsym{\}} ~|~ \ottnt{T_{{\mathrm{1}}}}  \, \& \,  \ottnt{T_{{\mathrm{2}}}} \\
  &\text{Destinations}&\textit{dst} ::=&~ \ottkw{nil} ~|~ \ottmv{y}  \ottsym{\mbox{?}} ~|~ \ottmv{z}
\end{align*}

\subsubsection{Type-Directed Compilation}

Similarly to the elaboration in \autoref{ch:calculi}, the compilation process
is type-directed. Besides the typing context $[[G]]$, there is also a
destination variable \textit{dst} that guides the code generation. A rule
basically reads as: given a typing context $[[G]]$ and a destination
$\ottnt{dst}$, the \fiplus term $[[e]]$ is checked/inferred to have type $[[A]]$
and is compiled to variable $\ottmv{z}$ in JavaScript code $\ottnt{J}$. That is,
after running $\ottnt{J}$, the result is stored in the JavaScript variable
$\ottmv{z}$.

\Rref{J-Int,J-Var} have three variants for different destinations, while
\rref{J-App,J-TApp} only have one version each but delegate to three variants of
application for different destinations, which helps to generate more optimized
JavaScript code. Examples illustrating variants of \rref{J-Var} and
\rref{JA-ArrowEquiv} (via \rref{J-App}) has been explained in \autoref{sec:dst}.
\Rref{J-IntOpt,J-IntNil} are designed for the optimization of boxing/unboxing
(see \autoref{sec:boxing}). Other rules assume that the destination is present
and generate code accordingly. \Rref{J-Nil} serves as the bridge from empty
destinations to non-empty ones, while \rref{J-Opt} is for optional ones.

{\small\ottdefncompile{}}

\subsubsection{Distributive Application and Projection}

We have mentioned that function applications (and record projections) have to be
specially handled because of distributive subtyping in \fiplus. To put it
simply, we need to additionally consider the cases where functions (and records)
have intersection types or top-like types. Similarly to previous rules,
destinations also guide the code generation. A rule for function applications
basically reads as: given a typing context $[[G]]$ and a destination
$\ottnt{dst}$, applying the compiled function in $\ottmv{x}$ of type $[[A]]$ to
the compiled argument $p$ yields variable $\ottmv{z}$ of type $[[B]]$ in
JavaScript code $\ottnt{J}$. Depending on whether the function is $\lambda$- or
$\Lambda$-bound, the argument $p$ can be either a value ($\ottmv{y} : [[C]]$) or
a type ($[[C]]$).

{\small\ottdefndapp{}}

\noindent
As explained in \autoref{sec:proj}, the rules for record projections are
separated to reduce the number of coercions and improve the performance of
generated JavaScript code, although they were combined with the rules for
function applications in the latest formalization of \fiplus by
\citet{fan2022direct}. A rule for record projections basically reads as:
projecting the compiled records in $\ottmv{x}$ of type $[[A]]$ by label $[[l]]$
yields variable $\ottmv{z}$ of type $[[B]]$ in JavaScript code $\ottnt{J}$.

{\small\ottdefndproj{}}

\subsubsection{Coercive Subtyping}

In \rref{J-Sub}, we check an expression of type $A$ against its supertype $B$.
Since the two types may correspond to compiled objects of different shapes, a
coercion has to be inserted for each subtyping check. Such a form of subtyping
is called \emph{coercive} subtyping~\citep{luo2013coercive}, in contrast to
\emph{inclusive} subtyping. A rule for coercive subtyping basically reads as: to
upcast a compiled object $\ottmv{x}$ of type $[[A]]$ to a compiled object
$\ottmv{y}$ of type $[[B]]$, we need to insert a coercion in JavaScript code
$\ottnt{J}$. The umbrella rule has three variants because of the optimization of
boxing/unboxing (see \autoref{sec:boxing}).

{\small\ottdefncsubZero{}}

\noindent
As explained in \autoref{sec:elim}, we add an extra flag to help optimize
coercions for subtyping between equivalent types: $<:^+$ indicates that the
optimization \rref{JS-Equiv} can apply, while $<:^-$ not.

{\small\ottdefncsub{}}

\noindent
There are some auxiliary rules called \emph{coercive merging} for
\rref{JS-Split}. These rules mean that if the splitting relation $[[split C A
B]]$ holds, we can merge the compiled objects $\ottmv{x}$ of type $[[A]]$ and
$\ottmv{y}$ of type $[[B]]$ back into a single object $\ottmv{z}$ of type
$[[C]]$ in JavaScript code $\ottnt{J}$. Such merging is necessary after
splitting the supertype distributively. For example, consider the following
derivation of subtyping:
\begin{mathpar}
\inferrule
  {[[split Top->Int&Bool Top->Int Top->Bool]] \\
   [[Top -> Int & String & Bool <: Top -> Int]] \\
   [[Top -> Int & String & Bool <: Top -> Bool]]}
  {[[Top -> Int & String & Bool <: Top -> Int & Bool]]}
\end{mathpar}
After splitting, the compiled object would have two fields with labels
\lstinline{"fun_int"} and \lstinline{"fun_bool"}, but we expect only one field
with label \lstinline{"fun_(int&bool)"}. \Rref{JM-Arrow} handles this case and
merge the two fields back into one.

The notation may be misleading, but note that here only the variable name
$\ottmv{z}$ is given (i.e. input) while variable names $\ottmv{x}$ and
$\ottmv{y}$ are generated by the rules (i.e. output). This is because
\rref{JM-And} reuses the variable name $\ottmv{z}$ to also serve as $\ottmv{x}$
and $\ottmv{y}$, which makes the caller perform more efficient in-place updates.
Not to make it more confusing but we have to emphasize that the discussion is
only about the variable \emph{name} rather than the contents of the variable.
Having a closer look at \rref{JS-Split} will help to better understand our
design. Below the judgment of coercive merging, the generated variable names
($[[y1]]$ and $[[y2]]$ in the case) are used to generate the coercions (in
$\ottnt{J}_1$ and $\ottnt{J}_2$). The coercions are actually executed before the
coercive merging (in $\ottnt{J}_3$) in generated JavaScript. To avoid $[[y1]]$
and $[[y2]]$ from being initialized more than once, some extra checks are
performed when generating JavaScript code for
\rref{JS-Split,JM-Arrow,JM-All,JM-Rcd}.

{\small\ottdefncmerge{}}
