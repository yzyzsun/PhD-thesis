\section{Introduction}

Many programming language constructs are first-class. First-class functions are
a key construct of functional programming. Similarly, objects are first-class in
object-oriented programming (OOP). First-class constructs enable the
corresponding values to be abstracted by variables, passed as arguments, or
returned by functions or methods.

While classes are pervasive in most OOP languages, first-class classes are
\emph{much less} studied, and they are \emph{rarely} supported in mainstream
statically typed OOP languages. Languages such as Java, C\#, and Swift, just to
name a few, do not support first-class classes. In these languages, no variables
can abstract over classes, and thus a class cannot pick which class to inherit
from at run time. Nevertheless, some dynamically typed languages treat classes
as first-class constructs and allow dynamic inheritance. Taking
JavaScript\footnote{Although object-orientation in JavaScript is originally
prototype-based, newer standards (ECMAScript 6+) also support classes on top of
prototypes.} as an example, a base class can be passed as an argument, and the
inheritance hierarchy is determined at run time, after the application of the
function happens:
\begin{lstlisting}[language=TypeScript]
function Mixin(Base) {
  return class extends Base {
    greet() { alert('Hello, world!') }
  };
}
\end{lstlisting}
First-class classes offer powerful and flexible abstraction mechanisms for
programmers. For instance, \emph{mixins}~\citep{bracha1990mixin}, which are
class-like abstractions that can be mixed into other classes to add new
features, are encodable via first-class classes and dynamic inheritance. In our
example, the \lstinline{Mixin} function creates a class that inherits from
\lstinline{Base} and adds a \lstinline{greet} method. At run time, we can apply
\lstinline{Mixin} to different base classes that need \lstinline{greet}. Dynamic
inheritance rejects the common assumption that inheritance hierarchies are fixed
at compile time, providing a greater degree of flexibility compared to static
inheritance. Furthermore, first-class classes provide natural support for
\emph{nested classes}: classes defined within another class, or even inside
methods or functions as in the \lstinline{Mixin} example. Nested classes can
access definitions and methods in the surrounding lexical scope. In JavaScript,
nested classes are supported via first-class classes. Some other OOP languages,
such as Java, support nested classes without supporting first-class classes.

\begin{figure}
\centering
\begin{subfigure}{.33\textwidth}
\begin{lstlisting}[language=TypeScript]
class A {
  m() { return 48 }
}
\end{lstlisting}
\end{subfigure}
\begin{subfigure}{.33\textwidth}
\begin{lstlisting}[language=TypeScript]
class B extends A {
  m() { return "Hi" }
}
\end{lstlisting}
\end{subfigure}
\caption{JavaScript allows unconstrained overriding, whereas TypeScript's type
  system attempts to prevent type-unsafe overriding and statically rejects the
  example.} \label{fig:override}
\end{figure}

To ensure type-safe inheritance, an important concern is how to deal with
\emph{overriding} and, more generally, method conflicts. JavaScript deals with
method conflicts by employing \emph{implicit} overriding. That is, a method in a
subclass will override a method in the superclass if the superclass contains a
method with the \emph{same name}. Otherwise, a new method is defined in the
subclass if no method with the same name exists in the superclass. In JavaScript
or other dynamically typed languages, overriding is completely unconstrained,
allowing the overriding method to return a different type. An example is shown
in \autoref{fig:override}. Such overriding is not type-safe if an object of the
subclass \lstinline{B} is to be used in the place where the superclass
\lstinline{A} is expected, since the method \lstinline{m} in \lstinline{A} is
expected to return a number instead of a string.

Since TypeScript is a superset of JavaScript, it adopts the same implicit
overriding approach. However, like most statically typed OOP languages,
TypeScript places restrictions on overriding to ensure type safety. In
TypeScript, overriding methods must have types compatible with the overridden
ones, in order to allow for the safe use of a subclass in the place of a
superclass. Another possibility is to allow subclasses not to be subtypes of the
superclass~\citep{cook1990inheritance}, which is sometimes seen in structurally
typed OOP languages. In this case, a subclass may not always be used in place of
a superclass, and a type system can prevent the use of subclasses that are not
subtypes.  Nevertheless, this does not imply that overriding can be fully
unconstrained, as it is still possible to have type-safety issues even when
inheritance does not imply subtyping.

First-class classes and dynamic inheritance make type-safe overriding much
harder. Few statically typed languages attempt to support such features, and
some of the ones that do have type-unsound designs. For instance, in addition to
supporting conventional static inheritance idioms, TypeScript also supports
dynamic inheritance, but its type system cannot always ensure type-safe
overriding. With dynamic inheritance, the exact type of the superclass is
unknown statically, so it is hard to guarantee that no method is accidentally
overridden with an incompatible type at run time. We will illustrate this point
using examples in TypeScript later in this chapter.

Implicit overriding is not the only way to deal with method conflicts. Another
possibility is to detect and prevent conflicts, \emph{disallowing} any form of
implicit overriding. For instance, the \emph{trait}
model~\citep{ducasse2006traits} adopts an approach where implicit overriding is
disallowed. With traits overriding is still possible, but it must be
\emph{explicitly} triggered by the programmer, instead of being implicitly done
by the compiler. For instance, when composing two traits with conflicts, the
composition will be \emph{rejected}. To resolve conflicts, a programmer can, for
example, decide to take one of the implementations for the method, or provide a
new method implementation instead. 

Yet another possibility to deal with conflicts is what we call \emph{merging}.
Merging is not a new idea and has been used to a certain degree in existing
programming language designs. For instance, merging is central in programming
language designs with \emph{virtual
classes}~\citep{madsen1989virtual,ernst2006virtual,clarke2007tribe} and
\emph{family
polymorphism}~\citep{ernst2001family,saito2008lightweight,zhang2017familia}.
Virtual classes are a form of nested classes. However, the main feature of
virtual classes is that, when a virtual class conflicts with another virtual
class with the same name, the old class is \emph{not overridden}. Instead, the
behaviors of the two classes are \emph{merged}: the new class will contain all
the methods of the old class as well as the new methods. So, unlike overriding,
merging does not replace existing behaviors. Instead, it preserves existing
behaviors and adds some new ones.\footnote{Strictly speaking, most designs with
virtual classes will combine merging with overriding, in the case that the two
virtual classes have conflicting methods. In our discussion, when we describe
merging, we assume that the sets of methods in the two virtual classes are
disjoint and, consequently, no overriding takes place.}

The idea of merging can be extended to deal with conventional methods as well.
For example, in a language that adopts merging, code similar to that in
\autoref{fig:override} can be accepted. Class \lstinline{B} would contain
\emph{two} versions of the method \lstinline{m}: one returning a number and the
other returning a string. In other words, merging would act as a kind of
\emph{overloading} in this case, enabling two methods with the same name but
different types to coexist in the same class. Invocations of \lstinline{m} could
be disambiguated by the surrounding context or, if needed, by the programmer. Of
course, in the merging model, the combination of two methods with the same name
and \emph{related types} would still be problematic, as it would not be clear
how to choose and disambiguate between the two method implementations.

A solution to this problem is to adopt a trait-like model with merging. This is
the model adopted by the CP language and is the focus of this chapter. In a
trait-like model with merging, method conflicts are still forbidden, but methods
with the same name and \emph{disjoint types} (i.e. the types are unrelated) do
not create a conflict. In other words, if we compose two traits, each having a
method with the same name and related types, then we get an \emph{error} due to
a method conflict. However, if the methods have the same name but disjoint
types, then the composition is accepted, and the resulting object will retain
the two method implementations. By allowing merging in the disjoint case, we can
express the forms of composition that are required for virtual classes. In such
cases, virtual classes are modeled as fields, and two virtual classes with the
same name but disjoint interfaces (i.e. the types of the virtual class methods)
can be merged.

Such a model offers important advantages over designs that adopt implicit
overriding instead. A first advantage is that we can obtain flexible, powerful,
and type-safe inheritance models and avoid many of the restrictions imposed by
languages with static inheritance.  With merging it is possible to have a model
of inheritance that allows \emph{dynamic inheritance} and forms of
\emph{multiple inheritance} and \emph{family polymorphism} all at once. We are
aware of some designs with dynamic inheritance and first-class
classes~\citep{takikawa2012gradual,lee2015theory}, but without family
polymorphism.  We are also aware of some designs with both multiple inheritance
and family polymorphism~\citep{nystrom2006j,aracic2006overview,clarke2007tribe},
but without dynamic inheritance. Except for the CP language, the only statically
typed language we know that supports all three features is
\texttt{gbeta}~\citep{ernst2000gbeta}. However, \texttt{gbeta} cannot statically
guarantee that every use of dynamic inheritance is type-safe, although
\citet{ernst2002safe} proves a subset of use cases to be type-safe. Moreover,
separate compilation can only be supported with an inefficient linear search
through super-mixins for inherited attributes. To the best of our knowledge, CP
is the only language that supports all three features in a completely type-safe
manner, without compromising on modular type checking or separate compilation.
We believe that this absence in the design space is because it is hard to have
flexible and type-safe designs that support all these features at once.

Because our design is based on the trait model, we also inherit its advantages.
In particular, since merging extends behavior rather than replace behavior, it
is less prone to problems such as the \emph{fragile base class
problem}~\citep{mikhajlov1998study}. As we shall see, in the presence of dynamic
inheritance, designs based on implicit overriding, such as TypeScript,
exacerbate the fragile base class problem: not only can overriding break
invariants of the superclass, but it can also break type safety! Designs based
on a trait-model with merging preserve the behavior of inherited classes and
avoid the issues due to (implicit) overriding.
