\section{Dynamic Inheritance, Overriding, and Type Safety} \label{sec:background}

There are only a few statically typed languages that support first-class classes
and dynamic inheritance, among which are \texttt{gbeta}~\citep{ernst2000gbeta},
TypeScript~\citep{typescript}, Typed Racket~\citep{takikawa2012gradual}, and
Wyvern~\citep{lee2015theory}. Here we take the most popular one, TypeScript, as
the main example to illustrate the challenges of type-safe dynamic inheritance
and reveal significant limitations of TypeScript's type system. We will also
briefly mention JavaScript and Java to further illustrate concepts related to
first-class classes and dynamic inheritance. Discussions about \texttt{gbeta},
Typed Racket, and Wyvern can be found in \autoref{sec:first-class}.

\subsection{Class Inheritance and Structural Typing} \label{sec:bivariant}
Classes are the reusable building blocks in most OOP languages. They are reused
by inheritance, a mechanism to create a new class (called a \emph{subclass})
based on an existing class (called a \emph{superclass}). Inheritance enables the
reuse of implementations of methods or properties that are already provided in
the superclass. Furthermore, it is possible to override methods of the
superclass with new implementations that are more suitable for the subclass.

To make sure that instances of the subclass can be used in any context where its
superclass is expected, there is usually a requirement that the subclass has a
\emph{subtype} with respect to its superclass. While inheritance is related to
implementations, subtyping is a relation between types. In many programming
languages, class definitions
\lstinline[language=TypeScript]|class B extends A {...}| introduce both
relations between \lstinline{A} and \lstinline{B}: class \lstinline{B} inherits
the implementation from class \lstinline{A}, and it also introduces a subtyping
relation between the types of the two classes. For example, type \lstinline{B}
is required to be a subtype of type \lstinline{A} in TypeScript:

\noindent
\begin{minipage}{.5\textwidth}
\begin{lstlisting}[language=TypeScript]
class A {}
\end{lstlisting}
\end{minipage}%
\begin{minipage}{.5\textwidth}
\begin{lstlisting}[language=TypeScript]
class B extends A {
  m(): number { return 0; }
}
\end{lstlisting}
\end{minipage}

\noindent
Owing to the same reason, when a method in the superclass is overridden, the new
method in the subclass must have a subtype. For example, the method
\lstinline{f} in \lstinline{C} is overridden by the one in \lstinline{D} below:

\noindent
\begin{minipage}{.5\textwidth}
\begin{lstlisting}[language=TypeScript]
class C {
  f(x: B): number { return x.m(); }
}
\end{lstlisting}
\end{minipage}%
\begin{minipage}{.5\textwidth}
\begin{lstlisting}[language=TypeScript]
class D extends C {
  f(x: A): number { return 48; }
}
\end{lstlisting}
\end{minipage}

\noindent
The overriding is type-safe because the latter method has a subtype of the
former's. According to the standard subtyping rule for functions, the parameter
type is \emph{contravariant}, and the return type is \emph{covariant}. Since
\lstinline{A} is a supertype of \lstinline{B}, the function type
\lstinline[language=TypeScript]{(x: A) => number} is a subtype of
\lstinline[language=TypeScript]{(x: B) => number}.

\paragraph{Bivariant subtyping in TypeScript.}
Perhaps surprisingly, the following code also type-checks:

\noindent
\begin{minipage}{.5\textwidth}
\begin{lstlisting}[language=TypeScript]
class E {
  f(x: A): number { return 48; }
}
\end{lstlisting}
\end{minipage}%
\begin{minipage}{.5\textwidth}
\begin{lstlisting}[language=TypeScript]
class F extends E {
  f(x: B): number { return x.m(); }
}
\end{lstlisting}
\end{minipage}

\noindent
The parameter type of method \lstinline{f} becomes a subtype of that in the
superclass, but it still passes the subtyping check. In other words, TypeScript
does \emph{not} follow the standard type-theoretic treatment of function
subtyping. Instead, TypeScript allows \emph{bivariant} subtyping for method
parameters, where the type of method parameters being overridden can either be a
subtype or a supertype of the corresponding type in the superclass method.
Bivariant subtyping is a well-known source of type unsoundness. It would lead to
a runtime error that could have been prevented statically:
\begin{lstlisting}[language=TypeScript]
const o: E = new F;
o.f(new A)  // Runtime Error!
\end{lstlisting}
TypeScript developers are aware of this, but they motivate the use of bivariant
subtyping by large numbers of use cases in the libraries that require this
functionality.\footnote{\url{https://www.typescriptlang.org/tsconfig\#strictFunctionTypes}}
In essence, TypeScript trades type soundness for flexibility and thus supports a
more flexible model of inheritance in some cases.

\paragraph{A type-safe alternative model for structural typing.}
TypeScript's class model adopts the approach that subclasses always generate
subtypes of the superclass. Thus, it retains the familiar model that is common
in mainstream nominally typed languages like Java, C\#, or Scala, which can be
seen as an advantage for attracting programmers from those languages.

However, unlike these mainstream programming languages, TypeScript is
\emph{structurally} typed. With structural typing, there is a well-known
alternative that would enable the overriding in class \lstinline{F} to be
type-safe. As observed by \citet{cook1990inheritance}, inheritance is not
subtyping. In the context of a language of classes, this means that sometimes
subclasses may not be subtypes of the superclass. In particular, the parameter
of a \emph{binary method}~\citep{bruce1995binary} is supposed to be an object of
the class being defined. In this case, the subclass will covariantly refine the
type of the method parameters, and thus detaching inheritance from subtyping can
be helpful. Since there is no subtyping relation between subclasses and
superclasses in an inheritance-is-not-subtyping approach, the standard
contravariant subtyping rule, instead of bivariant subtyping, can be used for
function parameters, thus preventing type-safety issues that arise from
bivariant subtyping.

If TypeScript adopted an inheritance-is-not-subtyping approach instead, then the
code for \lstinline{F} could still type-check, but the subclass \lstinline{F}
would not be a subtype of its superclass \lstinline{E}. Therefore, the runtime
error would be prevented because the line would be rejected with a type error:
\begin{lstlisting}[language=TypeScript]
// Invalid upcast in an inheritance-is-not-subtyping approach!
const o: E = new F;
\end{lstlisting}
While type-safe, the inheritance-is-not-subtyping approach departs from the
conventional model adopted by mainstream languages. So, it could be harder for
programmers (especially those used to other mainstream OOP languages) to
understand that sometimes subclasses cannot be subtypes. This is perhaps a
reason (among others) for TypeScript not adopting this approach. Nevertheless,
we adopt a model based on inheritance-is-not-subtyping because it allows a more
\emph{flexible} but still \emph{type-safe} form of inheritance.

\subsection{Unsafe Overriding with Dynamic Inheritance} \label{sec:override}

TypeScript differs from other mainstream OOP languages in that it also supports
dynamic inheritance. Dynamic inheritance brings new type-safety considerations
with respect to overriding. These issues are not due to the use of bivariant
subtyping and appear to be unknown or undocumented by the TypeScript
implementers. Nevertheless, in order to obtain a type-safe design, we must be
able to address the type-safety issues that may arise from dynamic inheritance.
Thus, the purpose of this subsection is to identify such a problem in
TypeScript. We call this problem the \emph{inexact superclass problem}, because
it arises from a mismatch between the statically expected type of the superclass
and the actual (exact) type of the superclass. In \autoref{sec:dynamic}, we will
show how this problem can be addressed in a type-safe manner.

\paragraph{Dynamic inheritance in TypeScript.}
While JavaScript accepts the unsafe overriding in \autoref{fig:override},
TypeScript detects the type mismatch between the two methods and rejects the
code. For top-level classes and static inheritance, TypeScript's type system is
quite standard and rejects many unsafe examples. However, the checks that
TypeScript does are insufficient for \emph{dynamic inheritance}, which is
recommended by the TypeScript documentation to implement
mixins.\footnote{\url{https://www.typescriptlang.org/docs/handbook/mixins.html}}
We illustrate the issue in the program in \autoref{fig:dynamic}.

\begin{figure}
\begin{lstlisting}[language=TypeScript]
type Constructor = new (...args: any[]) => {};

function Mixin<TBase extends Constructor>(Base: TBase) {
  return class extends Base {
    // If m() exists in Base, that one will be overridden by this definition.
    m(): number { return 48; }
  };
}

class A {
  m(): string { return "foobar"; }
  n(): string { return this.m().toUpperCase(); }
}

// Below we use A as Base, which contains m() with a different type.
const B = Mixin(A);
(new B).n()  // Runtime Error!
\end{lstlisting}
\caption{\emph{Inexact Superclass Problem:} Dynamic inheritance is type-unsafe in TypeScript.} \label{fig:dynamic}
\end{figure}

Our example follows the guidelines in the TypeScript documentation to type
mixins and first-class classes. First of all, a type \lstinline{Constructor} is
declared to represent a class. Since its return type is an empty object type,
the type of every class is a subtype of \lstinline{Constructor}. In other words,
every class can be used as \lstinline{Base}. The function \lstinline{Mixin}
takes a base class of type \lstinline{TBase} and returns a new class that
extends (or overrides) the base class with the method \lstinline{m}. Then we
obtain class \lstinline{B} by applying \lstinline{Mixin} to class \lstinline{A}.
Note that \lstinline{A} already has a method \lstinline{m} with a different
type, and the other method \lstinline{n} relies on \lstinline{m} returning a
\emph{string}. However, the subclass returned by \lstinline{Mixin} overrides
\lstinline{m} with a method that returns a \emph{number} instead. Finally, we
instantiate the class \lstinline{B} and call the method \lstinline{n}. A runtime
error occurs because the method \lstinline{m} is unexpectedly overridden. In
essence, we cannot predict the exact type of the superclass at compile time, so
we cannot prevent the unsafe overriding as statically typed languages do for
second-class classes and static inheritance.

\paragraph{Constrained mixins.}
The TypeScript documentation also mentions \emph{constrained mixins}, which
provide finer control on the base class. In a constrained mixin, the base class
is known to have some methods, which is useful for the subclass to safely rely
on those methods being present in the superclass. Constrained mixins are modeled
with a generic version of \lstinline{Constructor}:
\begin{lstlisting}[language=TypeScript]
type GConstructor<T = {}> = new (...args: any[]) => T;
\end{lstlisting}
The generic parameter \lstinline{T} represents the interface of the base class
and defaults to an empty object type. For example, we can define another mixin
that relies on a method called \lstinline{pow}:
\begin{lstlisting}[language=TypeScript]
type Exponentiatable = GConstructor<{ pow: (x: number, y: number) => number }>;

function AnotherMixin<TBase extends Exponentiatable>(Base: TBase) {
  return class extends Base {
    cube(x: number) { return this.pow(x, 3); }
  };
}
\end{lstlisting}
In \lstinline{AnotherMixin}, the method \lstinline{cube} relies on
\lstinline[language=TypeScript]{this.pow}, which is declared to be present by
the interface \lstinline{Exponentiatable}. Similarly, in the definition of
\lstinline{Mixin} in \autoref{fig:dynamic}, we could declare that
\lstinline{TBase} extends some type like \lstinline|GConstructor<AInterface>|.
Although the base class is constrained by the interface now, it still does not
help with the issue of unsafe overriding. The problem here is that the base
class may contain more methods than the expected interface. For instance, the
base class could contain another method called \lstinline{cube} that would
return a \emph{string}, and would be called in the base class by some other
method \lstinline{rubik}. Then we could still run into the same problem, if
\lstinline{rubik} is called from an object that combines both classes. There is
no way in TypeScript to express the constraint that \lstinline{cube} is
\emph{absent} in the base class. Such absence constraints are key to preventing
unsafe overriding in dynamic inheritance while retaining flexibility.

\paragraph{From static to dynamic inheritance.}
The crucial point in our examples is that dynamic inheritance has the
flexibility to pass a class with a \emph{subtype} of the expected type for the
base class in \lstinline{Mixin}. Languages with static inheritance and
second-class classes, like Java or C\#, do not have this flexibility.
Subclassing is usually modeled with a construct like
\lstinline[language=TypeScript]|class B extends A {...}|. In languages with
first-class classes, \lstinline{A} can be an arbitrary expression; but in
languages with static inheritance, it can only be a concrete class name. In
Java, for instance, a class \lstinline{A} is associated both to a type
\lstinline{A}, which is the \emph{exact} interface (or type) of the class, and a
corresponding implementation of type \lstinline{A}. In other words, a class
declaration has two roles in these languages: declaring an interface and
providing an implementation with exactly that interface. Thus, we can never
inherit from an implementation that has a subtype of the superclass type. This
avoids the inexact superclass problem that we have to face with dynamic
inheritance in our TypeScript example, at the cost of flexibility.

\subsection{Nested Classes via First-Class Classes}

Both JavaScript and TypeScript support first-class classes: a class can be
defined in various places including within another class, or even a method.
Thus, \emph{nested classes} come (almost) for free once a language supports
first-class classes. In contrast, some other OOP languages, such as Java, do not
support first-class classes, but they still add support for nested classes as a
separate feature.

Nested classes are useful for encapsulation, and usually, they can make use of
the definitions from the outer class. For example, \autoref{fig:string_it} shows
how to model a string-specific iterator as a nested class in JavaScript. The
constructor for the \lstinline{Iterator} class is modeled by a \emph{factory}
method. The method \lstinline{print} relies on the nested iterator class to
iterate over the characters in the string.

\begin{figure}
\begin{lstlisting}[language=TypeScript]
class ANSIString {
  constructor(str) {
    this.length = str.length;
    this.chars = str.split('');
  }

  Iterator() {
    const outer = this;
    return class {
      index = 0;
      hasNext() { return this.index < outer.length; }
      next() { return outer.chars[this.index++]; }
    };
  }

  print() {
    const it = new (this.Iterator());  // Iterator is dynamically bound.
    while (it.hasNext()) alert(it.next());
  }
}
\end{lstlisting}
\caption{A string iterator in JavaScript using nested classes.}
\label{fig:string_it}
\end{figure}

\paragraph{Why not a class field?}
In JavaScript, the use of the factory method is important to provide access to
\lstinline[language=TypeScript]{this} of the outer class. If we declare a class
field directly with \lstinline[language=TypeScript]|Iterator = class {...}|, we
would not be able to access the properties and methods of \lstinline{ANSIString}
within \lstinline{Iterator}. JavaScript does not provide a direct way to refer
to the outer \lstinline[language=TypeScript]{this} from the nested class. That
is why we have to capture the reference to the outer
\lstinline[language=TypeScript]{this} in a variable \lstinline{outer} before
using it in the nested class. Then, the properties declared in the outer class,
\lstinline{length} for example, can be accessed via \lstinline{outer.length}.

The second reason for using a factory method is to make access to
\lstinline[language=TypeScript]{super.Iterator} possible in a subclass of
\lstinline{ANSIString}. In JavaScript, a class field defined by the superclass
is not accessible in the subclass via \lstinline[language=TypeScript]{super}.
Declaring \lstinline{Iterator} as a factory method bypasses the restriction.
Although we do not use \lstinline[language=TypeScript]{super.Iterator} in the
current example, \autoref{sec:family} will show some use cases.

\paragraph{Overriding nested classes.}
In JavaScript, the inheritance behavior for nested classes is consistent with
that for methods: they both employ an overriding semantics. This is partly
because nested classes must always be accessed via a property or a method, and
then we just use the default overriding semantics for them. The ability to
override nested classes allows some useful forms of family polymorphism, as we
shall discuss in \autoref{sec:family}. However, it is also a problem for type
safety since we can override a class with another class that has an entirely (or
partially) different set of methods. For example, the following code is allowed
in JavaScript:
\begin{lstlisting}[language=TypeScript]
class UTF8String extends ANSIString {
  Iterator() { return class {
    forEach(callback) { /* ... */ }
  }; }
}
(new UTF8String("Hi")).print();  // Runtime Error!
\end{lstlisting}
The class \lstinline{Iterator} nested in \lstinline{ANSIString} contains two
methods \lstinline{hasNext} and \lstinline{next}, while the one nested in
\lstinline{UTF8String} only contains a different method \lstinline{forEach}.
After overriding, \lstinline{print} triggers a runtime error since it depends on
the aforementioned two methods. Therefore, code relying on nested classes having
a certain interface can be completely broken by an override that replaces the
class with some other incompatible class.

\paragraph{Nested classes in TypeScript.}
TypeScript also attempts to prevent type-unsafe overriding for nested classes.
Similar code will be rejected by TypeScript because of the type incompatibility
between the two nested classes. However, with dynamic inheritance, the type
system still suffers from similar issues to those shown in
\autoref{fig:dynamic}:
\begin{lstlisting}[language=TypeScript]
function Mixin<TBase extends Constructor>(Base: TBase) {
  return class extends Base {
    Iterator() { return class {
      forEach(callback: (_: string) => void) { /* ... */ }
    }; }
  };
}
const UTF8String = Mixin(ANSIString);
(new UTF8String("Hi")).print();  // Runtime Error!
\end{lstlisting}
\noindent Therefore, TypeScript's support for nested classes is also affected by
the inexact superclass problem. Thus, nested classes can have type-soundness
issues as well.

\begin{figure}
\begin{lstlisting}[language=Java]
class ANSIString {
    int length;
    char[] chars;
    ANSIString(String str) {
        length = str.length();
        chars = str.toCharArray();
    }
    class Iterator {
        int index;
        boolean hasNext() { return index < length; }
        char next() { return chars[index++]; }
    }
    void print() {
        Iterator it = new Iterator();  // Iterator is statically bound.
        while (it.hasNext()) System.out.print(it.next());
    }
}
class UTF8String extends ANSIString {
    // We trivially call super's constructor.
    UTF8String(String str) { super(str); }

    class Iterator {  // This class shadows ANSIString.Iterator.
        void forEach(Consumer<? super Character> action) { /* ... */ }
    }
}
\end{lstlisting}
\caption{Nested classes in Java, with a shadowing semantics.}
\label{fig:nested_shadow}
\end{figure}

\paragraph{Nested classes with shadowing in Java.}
Finally, let us make a small digression to see how nested classes are treated in
Java. \autoref{fig:nested_shadow} illustrates a variant of our example in Java.
Similarly to the example in JavaScript, we create a new class
\lstinline{UTF8String} that inherits from \lstinline{ANSIString} and define a
different set of methods in the nested class \lstinline{Iterator}. The code
type-checks in Java and is still type-safe. The key to the type safety is that,
unlike methods, nested classes are not implicitly overridden in Java. Instead,
the \lstinline{Iterator} in \lstinline{UTF8String} \emph{shadows} the one in
\lstinline{ANSIString}. In other words, \lstinline{new Iterator()} in
\lstinline{print} is statically bound and is always instantiating
\lstinline{ANSIString.Iterator}. Nested classes are not dynamically dispatched
in Java, which is inconsistent with the inheritance behavior for methods. The
shadowing approach has the advantage of type safety, but this comes at the cost
of flexibility, since the ability to override and dynamically bind nested
classes is useful, as we shall see next.

\subsection{Virtual Classes and Family Polymorphism} \label{sec:family}

The ability to override or refine nested classes provides a considerable amount
of flexibility, and is a key idea behind concepts such as \emph{virtual
classes}~\citep{madsen1989virtual,ernst2006virtual,clarke2007tribe} and
\emph{family
polymorphism}~\citep{ernst2001family,saito2008lightweight,zhang2017familia}.
Thus, as we shall argue in this subsection, both JavaScript and TypeScript
support virtual classes to a large extent, which can be useful for writing
highly modular and reusable code.

\paragraph{Virtual classes.}
As we have seen before, a method in a superclass can be overridden in a subclass
to refine its behavior. A call to the method is dynamically dispatched according
to the runtime type of the object. Such a late-bound method is called a
\emph{virtual method}. In the same way, the power of dynamic dispatching can be
extended to nested classes. \emph{Virtual classes} are nested classes that can
be overridden (or rather refined) in subclasses, and the reference to the
virtual class is determined by the runtime type of the object of the outer
class. Virtual classes were originally introduced in the BETA programming
language~\citep{madsen1993object}, and they are also essentially supported in
JavaScript and TypeScript via first-class classes and the overriding semantics.


\paragraph{Family polymorphism.}
Virtual classes enable \emph{family polymorphism}, which naturally solves the
long-standing dilemma of modularity and extensibility -- \emph{the expression
problem}~\citep{wadler1998expression} -- in a Scandinavian
style~\citep{ernst2004expression}. In the expression problem, the challenge is
to provide various operations (evaluation, pretty-printing, etc.) over various
expressions (numbers, addition, negation, etc.) in a modular fashion. A
satisfactory solution should allow modular, type-safe extension to both
expressions and operations.

\begin{figure}[t]
\begin{subfigure}{.5\textwidth}
\begin{lstlisting}[language=TypeScript,basicstyle=\ttfamily\footnotesize]
type Eval = { eval: () => number };

class FamilyEval {
  Lit(n: number) {
    return class {
      eval() { return n; }
    };
  }
  Add(l: Eval, r: Eval) {
    return class {
      eval() { return l.eval() +
                      r.eval(); }
    };
  }
}
\end{lstlisting}
\caption{Initial family.}\label{fig:EP_initial}
\end{subfigure}%
\begin{subfigure}{.5\textwidth}
\begin{lstlisting}[language=TypeScript,basicstyle=\ttfamily\footnotesize]
type Print = { print: () => string };

class FamilyPrint extends FamilyEval {
  Lit(n: number) {
    return class extends super.Lit(n) {
      print() { return n.toString(); }
    };
  }
  Add(l: Eval&Print, r: Eval&Print) {
    return class extends super.Add(l, r) {
      print() { return l.print() + " + " +
                       r.print(); }
    };
  }
}
\end{lstlisting}
\caption{Adding a new operation.}\label{fig:EP_operation}
\end{subfigure}
\par\bigskip
\begin{subfigure}{\textwidth}
\begin{lstlisting}[language=TypeScript,xleftmargin=.25\textwidth,basicstyle=\ttfamily\footnotesize]
class FamilyNeg extends FamilyPrint {
  Neg(e: Eval&Print) {
    return class {
      eval()  { return -e.eval(); }
      print() { return "-(" + e.print() + ")"; }
    };
  }
}
\end{lstlisting}
\caption{Adding a new expression.}\label{fig:EP_expression}
\end{subfigure}
\caption{Expression Problem in TypeScript.}
\end{figure}

We start the example with numeric literals and addition, as well as the
evaluation operation in \autoref{fig:EP_initial}. \lstinline{Lit}, for numeric
literals, and \lstinline{Add}, for addition, form the initial class family
\lstinline{FamilyEval}. Since TypeScript is structurally typed, we do not need
to declare an abstract class or interface \lstinline{Exp} together with
\lstinline{Lit} and \lstinline{Add}. Instead, we can directly use type
\lstinline{Eval} to annotate the parameters of \lstinline{Add}.

To add a new operation, say pretty-printing, we can create a new class family
\lstinline{FamilyPrint} that inherits from \lstinline{FamilyEval}.
\autoref{fig:EP_operation} shows the code for the new family. In the new family,
\lstinline{Lit} and \lstinline{Add} also inherit from
\lstinline[language=TypeScript]{super.Lit} and
\lstinline[language=TypeScript]{super.Add}, and a new method \lstinline{print}
is added to both of them. The new operation is represented by type
\lstinline{Print}, and the parameters of \lstinline{Add} are refined to have
type \lstinline{Eval&Print}. As mentioned in \autoref{sec:bivariant}, TypeScript
allows bivariant subtyping for parameters of class members, so the unusual
refinement of \lstinline{Add} type-checks here. Note that the overriding of
nested classes is a special case here: \lstinline{Lit} and \lstinline{Add} are
simply extended with new methods, with no existing methods being overridden. In
other words, the nested classes are being \emph{merged}, instead of overriding
existing functionality.

Similarly, we create a new family \lstinline{FamilyNeg} for a new expression,
say negation in \autoref{fig:EP_expression}. Finally, we instantiate
\lstinline{FamilyNeg} and build an expression using all three constructors
(\lstinline{Lit}, \lstinline{Add}, and \lstinline{Neg}). Both operations
(\lstinline{eval} and \lstinline{print}) are available for the expression:
\begin{lstlisting}[language=TypeScript]
const fam = new FamilyNeg();
const e = new (fam.Add(new (fam.Lit(48)), new (fam.Neg(new (fam.Lit(2))))));
e.print() + " = " + e.eval()  // "48 + -(2) = 46"
\end{lstlisting}
In this way, we can solve the expression problem in TypeScript (modulo the
type-safety requirement). Although TypeScript does not fully ensure type safety,
its support for a rather minimal encoding of virtual classes allows a lot of
flexibility and reuse, which can be quite useful in practice.

One final remark is that the solution in TypeScript is still not completely
satisfactory because the order of extensions is fixed by the inheritance
hierarchy (from \lstinline{FamilyEval} to \lstinline{FamilyPrint} to
\lstinline{FamilyNeg}). In other words, the extension of \lstinline{Neg} is
coupled to the extension of \lstinline{Print}, and we cannot use the extension
of \lstinline{Neg} independently. This issue was not mentioned by
\citeauthor{wadler1998expression} in the original expression problem, but it was
later identified by \citet{zenger2005independently} as \emph{independent
extensibility}. In TypeScript, a possibility to address the coupling issue is to
adopt the mixin pattern, making class families such as \lstinline{FamilyPrint}
and \lstinline{FamilyNeg} functions parametrized by the family superclass. For
simplicity of presentation, we have just employed static inheritance here. We
will also address this issue in CP's solution in \autoref{sec:ep}, which
provides a simple and natural approach to avoid coupling and, additionally, is
type-safe.
