\section{Dynamic Inheritance in CP} \label{sec:overview}

CP is a statically typed language that supports dynamic inheritance via merging
and still guarantees type safety. In this section, we first show how inheritance
is modeled as merging in CP. We then demonstrate how CP resolves potential
conflicts in dynamic inheritance and addresses the inexact superclass problem.
Finally, we present a form of \emph{dynamic} family polymorphism in CP.

\subsection{From Merging to Inheritance} \label{sec:inheritance}

Let us first discuss how inheritance is modeled in CP. According to the
denotational semantics of inheritance~\citep{cook1989denotational}, an object is
essentially a record, and a class (or a trait in CP) is essentially a function
over records. Therefore, class \lstinline{A} in \autoref{fig:dynamic} can be
encoded as:
\begin{lstlisting}
type Rcd = { m: String; n: String };

-- class A
mkA = \(this: Rcd) -> { m = "foobar"; n = toUpperCase this.m };
\end{lstlisting}
The function parameter \lstinline{this} is a self-reference. With the
self-reference, we can refer to other fields like \lstinline{this.m} in the
\lstinline{n} field. In this model, the instantiation of a class is obtained by
taking a fixpoint of the function. Furthermore, class inheritance can be encoded
as record concatenation:
\begin{lstlisting}[deletekeywords=super]
-- class B extends A
mkB = \(this: Rcd) -> let super = mkA this in super , { m = 48 };
\end{lstlisting}
We first provide the new self-reference to \lstinline{mkA} to obtain
\lstinline[deletekeywords=super]{super}. Then we merge
\lstinline[deletekeywords=super]{super} with the body of class \lstinline{B} to
obtain the final object. After instantiating class \lstinline{B} with a
fixpoint, we can access the \lstinline{n} field:
\begin{lstlisting}
o = fix this: Rcd. mkB this;  --> { m = "foobar"; n = "FOOBAR"; m = 48 }
o.n  --> "FOOBAR"
\end{lstlisting}
Here we get the expected result instead of a runtime error. The key point is
that we allow duplicate labels as long as the fields have disjoint types.
Because of the merging semantics of CP, \lstinline{o} will have two
\lstinline{m} fields: one of type \lstinline{Int} and the other of type
\lstinline{String}. Thus, unlike TypeScript, no implicit (and type-unsafe)
overriding happens in this case. Instead, both \lstinline|{ m = "foobar" }| and
\lstinline|{ m = 48 }| are kept in the record \lstinline{o}, and
\lstinline{toUpperCase this.m} will automatically pick the former one.
Internally, \lstinline{o.m} has the intersection type \lstinline{String&Int},
which means it contains a merge of a string and an integer. Such behavior is a
kind of \emph{overloading by return type}, which is supported in some languages
such as Swift and Haskell (via type classes)~\citep{marntirosian2020resolution}.

\emph{Traits in CP} follow the aforementioned model of inheritance. Therefore,
the example above can be rewritten in the form of traits:
\begin{lstlisting}
mkA = trait [this: Rcd] => { m = "foobar"; n = toUpperCase this.m };
mkB = trait [this: Rcd] inherits mkA => { m = 48 };
\end{lstlisting}
The self-type annotation \lstinline{[this: Rcd]} corresponds to the function
parameter \lstinline{this} in the previous code. If there is no use of
\lstinline{this} in any field, the self-type annotation can be omitted. The
instantiation of a trait is more conveniently done by the \lstinline{new}
keyword:
\begin{lstlisting}
o = new mkB;  o.n  --> "FOOBAR"
\end{lstlisting}

\subsection{Dynamic Inheritance in CP} \label{sec:dynamic}

\begin{figure}
\begin{lstlisting}[xleftmargin=.2\textwidth]
mixin (TBase * { m: Int }) (base: Trait<TBase>) =
  trait [this: TBase] inherits base => { m = 48 };

mkA = trait [this: { m: String; n: String }] => {
  m = "foobar";
  n = toUpperCase this.m;
};

o = new mixin @{ m: String; n: String } mkA;
o.n  --> "FOOBAR"
\end{lstlisting}
\caption{Solving the inexact superclass problem in CP.}
\label{fig:inexactCP}
\end{figure}

\noindent
Now let us go back to the safety issue demonstrated in \autoref{fig:dynamic} and
see how it can be solved in CP. The code for a CP solution is shown in
\autoref{fig:inexactCP}. Here the function \lstinline{mixin} has two parameters:
\lstinline{TBase} is a type parameter, which is disjoint with
\lstinline|{ m: Int }|; and \lstinline{base} is a term parameter, which is a
trait that implements \lstinline{TBase}. Like first-class classes in TypeScript,
we can dynamically create a trait that inherits from \lstinline{base} in CP. The
difference here is that we can declare the absence of \lstinline|{ m: Int }| in
the trait \lstinline{base} to make sure that there is no conflict. As mentioned
in \autoref{sec:inheritance}, CP does a fine-grained disjointness check that
considers, not only the label name, but also the field type. Therefore,
\lstinline|{ m: String }| is disjoint with \lstinline|{ m: Int }|, and there is
no conflict in the dynamic inheritance. Since both versions of \lstinline{m}
fields are available in \lstinline{o}, the \lstinline{n} field can still rely on
the original \lstinline{m} field that contains a string. Together with
disjointness constraints, type safety is guaranteed in CP without sacrificing
the flexibility of dynamic inheritance.

Finally, note that if we apply \lstinline{mixin} to a different trait that
contains a \lstinline{m} field of type \lstinline{Int}:
\begin{lstlisting}
mkA' = trait => { m = 0; n = 0 };
o = new mixin @{ m: Int; n: Int } mkA';  -- Type Error!
\end{lstlisting}
We will get a type error because \lstinline|{ m: Int; n: Int }| is \emph{not}
disjoint with \lstinline|{ m: Int }|. In other words, the field \lstinline{m} in
\lstinline{mkA'} conflicts with \lstinline{m} in \lstinline{mixin}.


\subsection{Family Polymorphism in CP} \label{sec:ep}

Here we revisit the example of family polymorphism in \autoref{sec:family} and
show how it can be implemented in CP. As before, we start with the evaluation of
numeric literals and addition. The CP code is shown in
\autoref{fig:EP_cp_initial}. The compositional interface \lstinline{AddSig}
serves as the specification of expressions, while type \lstinline{Eval}
represents the evaluation operation. Note that \lstinline{<Exp>} is a special
type parameter called a \emph{sort} in CP. A sort is kept abstract until it is
instantiated with a concrete type like in \lstinline{AddSig<Eval>}. The
interface \lstinline{AddSig<Eval>} is implemented by trait
\lstinline{familyEval}, where syntactic sugar called \emph{method patterns} is
used to keep code compact. The desugared code is:
\begin{lstlisting}
familyEval = trait implements AddSig<Eval> => {
  Lit = \n   -> trait => { eval = n };
  Add = \l r -> trait => { eval = l.eval + r.eval };
};
\end{lstlisting}
Although the syntactic sugar makes it seem that \lstinline{eval} is defined by
pattern matching of constructors, \lstinline{(Lit n)} and \lstinline{(Add l r)}
are actually nested traits, which are virtual and can be refined in CP.

\begin{figure}
\begin{minipage}{.5\textwidth}
\begin{subfigure}{\textwidth}
\begin{lstlisting}[basicstyle=\ttfamily\footnotesize]
type AddSig<Exp> = {
  Lit: Int -> Exp;
  Add: Exp -> Exp -> Exp;
};

type Eval = { eval: Int };

familyEval =
  trait implements AddSig<Eval> => {
    (Lit   n).eval = n;
    (Add l r).eval = l.eval + r.eval;
  };
\end{lstlisting}
\caption{Initial family.}\label{fig:EP_cp_initial}
\end{subfigure}
\end{minipage}%
\begin{minipage}{.5\textwidth}
\begin{subfigure}{\textwidth}
\begin{lstlisting}[basicstyle=\ttfamily\footnotesize]
type Print = { print: String };

familyPrint =
  trait implements AddSig<Print> => {
    (Lit   n).print = toString n;
    (Add l r).print = l.print ++ " + "
                   ++ r.print;
  };
\end{lstlisting}
\caption{Adding a new operation.}\label{fig:EP_cp_operation}
\end{subfigure}
\par\bigskip
\begin{subfigure}{\textwidth}
\begin{lstlisting}[basicstyle=\ttfamily\footnotesize]
type NegSig<Exp> = { Neg: Exp -> Exp };

familyNeg =
  trait implements NegSig<Eval&Print> => {
    (Neg e).eval  = -e.eval;
    (Neg e).print = "-(" ++ e.print ++ ")";
  };
\end{lstlisting}
\caption{Adding a new expression.}\label{fig:EP_cp_expression}
\end{subfigure}
\end{minipage}
\caption{Expression Problem in CP.}
\end{figure}

The solution to the expression problem in CP is quite straightforward. To extend
operations, we instantiate the sort with another type and implement it with
another trait. For example, \autoref{fig:EP_cp_operation} shows how to add
support for pretty-printing. In the other dimension, we add negation to numeric
literals and addition. We define a new compositional interface and implement
both operations with a trait in \autoref{fig:EP_cp_expression}. This time we
instantiate the sort of \lstinline{NegSig} with the intersection type
\lstinline{Eval&Print}.

Finally, we can compose the two-dimensional extensions together by the merge
operator easily:
\begin{lstlisting}
fam = new familyEval , familyPrint , familyNeg
    : AddSig<Eval&Print> & NegSig<Eval&Print>;
\end{lstlisting}

\paragraph{Nested composition and distributive subtyping.}
The merge of the three traits seems simple from a syntactic perspective.
However, it requires a more sophisticated mechanism under the hood. Let us look
at the desugared code for the merge between \lstinline{familyEval} and
\lstinline{familyPrint}:
\begin{lstlisting}
trait implements AddSig<Eval> => {                 -- familyEval
  Lit = \n   -> trait => { eval = n };
  Add = \l r -> trait => { eval = l.eval + r.eval };
} ,
trait implements AddSig<Print> => {                -- familyPrint
  Lit = \n   -> trait => { print = toString n };
  Add = \l r -> trait => { print = l.print ++ " + " ++ r.print };
}
\end{lstlisting}
Our expectation is that the result of merging should contain, for example, a
single constructor \lstinline{Lit} that supports both the \lstinline{eval} and
\lstinline{print} operations. Therefore, the result should be equivalent to:
\begin{lstlisting}
trait implements AddSig<Eval&Print> => {
  Lit = \n   -> trait => { eval = n;
                           print = toString n };
  Add = \l r -> trait => { eval = l.eval + r.eval;
                           print = l.print ++ " + " ++ r.print };
}
\end{lstlisting}
To achieve this, CP employs \emph{nested composition}~\citep{bi2018essence} and
\emph{distributive subtyping}~\citep{barendregt1983filter}, where traits,
records, and functions distribute over intersections. In other words, merging
applies to the whole trait hierarchy, including nested traits. This example
showcases family polymorphism by the refinement of nested traits (i.e. CP's
version of virtual classes).

With these features available in CP, we can access the three constructors
(\lstinline{Lit}, \lstinline{Add}, and \lstinline{Neg}) as well as the two
operations (\lstinline{eval} and \lstinline{print}), similarly to the previous
TypeScript code:
\begin{lstlisting}
e = new fam.Add (new fam.Lit 48) (new fam.Neg (new fam.Lit 2));
e.print ++ " = " ++ toString e.eval  --> "48 + -(2) = 46"
\end{lstlisting}

\paragraph{Dynamic family polymorphism.}
Since merging generalizes dynamic inheritance, we can rewrite
\lstinline{familyNeg}, for instance, using a mixin style:
\begin{lstlisting}
familyNeg (TBase * NegSig<Eval&Print>) (base: Trait<TBase>) =
  trait [this: TBase] implements NegSig<Eval&Print> inherits base => {
    (Neg e).eval  = -e.eval;
    (Neg e).print = "-(" ++ e.print ++ ")";
  };
fam = new familyNeg @AddSig<Eval&Print> (familyEval,familyPrint)
    : AddSig<Eval&Print> & NegSig<Eval&Print>;
\end{lstlisting}
By applying \lstinline{familyNeg} to \lstinline{(familyEval,familyPrint)}, we
dynamically create a trait that inherits from the latter. Of course, we can
choose other traits as a base trait at run time, which is supported by dynamic
inheritance in CP.

Note that in \autoref{sec:family}, \lstinline{FamilyEval},
\lstinline{FamilyPrint}, and \lstinline{FamilyNeg} have a statically fixed
inheritance hierarchy. As a result, the negation expression cannot be separated
from the other two expressions because \lstinline{FamilyNeg} is a subclass of
\lstinline{FamilyPrint}. In contrast, the inheritance hierarchy can be
dynamically determined in CP, so \lstinline{familyEval},
\lstinline{familyPrint}, and \lstinline{familyNeg} can all be individually used
or composed with any other traits. In fact, CP's solution solves a dynamic
variant of the expression problem, which can be seen as the combination of the
expression product line~\citep{lopez2005evaluating} and dynamic software product
lines~\citep{hallsteinsen2008dynamic}.

\subsection{Discussion}

In this and the previous section, we have seen that both CP and
JavaScript/TypeScript support a powerful and expressive form of dynamic
inheritance. However, there are some important differences worth noting:

\begin{itemize}

\item \textbf{CP is type-safe.}
      While the three languages provide a high degree of flexibility, CP is the
      only language which combines flexibility and type safety.

\item \textbf{No implicit overriding in CP.}
      Unlike JavaScript/TypeScript, where implicit overriding is common, CP
      adopts a trait model, so implicit overriding can \emph{never} happen.

\item \textbf{Dealing with conflicts using disjoint types.}
      In JavaScript/TypeScript, method overriding is based on \emph{names}. So
      even when the method or field in the superclass has a different (or
      disjoint) type, overriding happens when the subclass has a method with the
      same name. As we have seen, this is the source of type unsoundness in the
      inexact superclass problem. In CP, methods with disjoint types can coexist
      in the same object. Thus, for the same situation, CP will not override but
      inherit the method from the superclass.
\end{itemize}

\noindent
These differences are important to obtain flexibility while preserving type
safety. However, these differences also mean that the dynamic semantics of CP
needs to be different from that of JavaScript/TypeScript. In particular, the
dynamic semantics of CP has to be aware of types, since types play a role in
determining whether conflicts exist or not, and in unambiguously performing
method lookup. This creates important challenges in obtaining an efficient
implementation, which have not been addressed in previous work.
